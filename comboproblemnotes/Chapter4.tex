\chapter{Algebraic Counting Techniques}
\section{The Principle of Inclusion and Exclusion}
\subsection{The size of a union of sets}  One of our  very first counting
principles was the {sum principle}\index{sum principle} which says that
the size of a union of disjoint sets is the sum of their sizes. 
Computing the size of overlapping sets requires, quite naturally,
information about how they overlap.  Taking such information into account
will allow us to develop a powerful extension of the sum principle known
as the ``principle of inclusion and exclusion."\index{principle of
inclusion and exclusion}\index{inclusion and exclusion principle}
  \bp
\itemm In a biology lab study of the effects of basic fertilizer
ingredients on plants, 16 plants are treated with potash, 16 plants are
treated with phosphate, and among these plants, eight are treated with
both phosphate and potash. No other treatments are used.  How many plants
receive at least one treatment?  If 32 plants are studied, how many
receive no treatment?\label{fertilizer2}
\solution{The number of plants receiving treatment was $16+16-8 = 24$.
The number of plants receiving no treatment was eight.}
\items  Give a formula for the size of the union $A\cup B$ of two
sets
$A$ in terms of the sizes $|A|$ of $A$, $|B|$ of $B$, and $|A\cap B|$ of
$A\cap B$. If $A$ and $B$ are subsets of some ``universal" set $U$,
express the size of the complement $U-(A\cup B)$ in terms of the sizes
$|U|$ of $U$, $|A|$ of $A$, $|B|$ of $B$, and $|A\cap B|$ of
$A\cap B$.
\solution{$|A\cup B|=|A| + |B| - |A\cap B|$.\\  $|U-(A\cup B)| = |U|-|A|-|B| + |A
\cap B|$.}

\itemm In Problem \ref{fertilizer2}, there were just two fertilizers used
to treat the sample plants.  Now suppose there are three fertilizer
treatments, and 15 plants are treated with nitrates, 16 with potash, 16
with phosphate, 7 with nitrate and potash, 9 with nitrate and phosphate,
8 with  potash and phosphate and 4 with all three.  Now how many plants
have been treated?  If 32 plants were studied, how many received no
treatment at all?
\solution{$15+16+16-7-9-8+4=27$ plants were treated and five received
no treatment.}

\iteme Give a formula for the size of $A_1\cup A_2\cup A_3$ in terms of
the sizes of $A_1$, $A_2$, $A_3$ and the intersections of these
sets.\label{threesetintersection}
\solution{\\$|A\cup B\cup C|=|A|+|B|+|C|-|A\cap B|- |A\cap C| - |B\cap
C| +|A\cap B\cap C|$.}

\iteme Conjecture a formula for the size of a union of sets $$A_1\cup
A_2\cup \cdots\cup A_n = \bigcup_{i=1}^n A_i$$ in terms of the sizes of
the sets $A_i$ and their intersections.\label{nsetintersection}
\solution{
$$\left|\bigcup_{i=1}^n A_i\right| = \sum_{i=1}^n|A_i| -
\sum_{i=1}^n\sum_{j=i+1}^n |A_i \cap A_j| + \cdots +(-1)^{n-1} |A_1\cap
A_2\cap\cdots \cap A_n|.$$}
\ep
The difficulty of generalizing Problem \ref{threesetintersection} to
Problem \ref{nsetintersection} is not likely to be one of being able to
see what the right conjecture is, but of finding a good notation to
express your conjecture.  In fact, it would be easier for some people to
express the conjecture in words than to express it in a notation. Here
is some notation that will make your task easier.  Let us define
$$\bigcap_{i:i\in I}A_i$$ to mean the intersection over all elements $i$
in the set $I$ of
$A_i$.  Thus
  \begin{equation}
\bigcap_{i:i\in
\{1,3,4,6\}} = A_1\cap A_3\cap A_4 \cap A_6.\label{intersectionnotation}
  \end{equation}
This kind of notation, consisting of an operator with a description
underneath of the values of a dummy variable of interest to us, can be
extended in many ways.  For example
\begin{eqnarray}\sum_{I:I \subseteq \{1,2,3,4\}, \ |I|=2} |\cap_{i\in I}
A_i| &=& |A_1\cap A_2|+ |A_1\cap A_3|
 +|A_1\cap A_4|\nonumber\\ &+& |A_2\cap A_3|+
|A_2\cap A_4| +|A_3\cap A_4|.\label{notationsolution}
\end{eqnarray}
  \bp
 \iteme Use notation something like that of Equation
\ref{intersectionnotation} and Equation \ref{notationsolution} to express
the answer to Problem \ref{nsetintersection}.  Note there are many
different correct ways to do this problem.  Try to write down  more
than one and choose the nicest one you can.  Say why you chose it
(because your view of what makes a formula nice may be different from
somebody else's).  The nicest formula won't necessarily involve all the
elements of Equations
\ref{intersectionnotation} and 
\ref{notationsolution}.\label{inclusion-exclusionunion}
\solution{$$\left|\bigcup_{i=1}^n A_i\right| = \sum_{S:S\subseteq [n],
S\not=\emptyset}(-1)^{|S|-1}|\bigcap_{i: i\in S}A_i|$$ I chose this
way of writing the formula partly because it is efficient with symbols;
for example, it uses only one sum sign. But more importantly I chose it
because it captures what I would want to say in words: ``You sum, over
all ways of choosing an intersection of the sets $A_i$, the size of the
intersection times a sign factor that is -1 if you are intersecting
an even number of sets and 1 if you are intersecting an odd number." 
If I were writing my solution out in words, I would probably assume
that nobody would think about the possibility of an intersection of the
empty set of the
$A_i$s, but I had to put the $S\not=\emptyset$ in my formula because
otherwise the formula would have had us consider the possibility that
$S$ was empty.}

\iteme  A group of $n$ students goes to a restaurant carrying backpacks. 
The manager invites everyone to check their backpack at the check desk and
everyone does.  While they are eating, a child playing in the check room
randomly moves around the claim check stubs on the backpacks.  What is
the probability that, at the end of the meal, at least one student
receives his or her own backpack?  In other words, in what fraction of
the total number of ways to pass the backpacks back does at least one
student get his or her own backpack back?  (Hint: For each student, how
big is the set of backpack distributions in which that student gets the
correct backpack? It might be a good idea to
first consider cases with $n=3$, 4, and 5.) 
\label{hatcheck}  What is the probability that no student gets his or her
own backpack?
\solution{First we compute the number of ways to pass out the
backpacks so that at least one student gets the correct one, then we
divide by $n!$ the total number of ways to pass out the backpacks to
get the probability that at least one student gets the correct one. 
If we let $A_i$ be the set of backpack distributions in which student
$i$ gets the correct backpack, then the number we want to compute is
the size of the union of the sets $A_i$.  For this purpose we need to
know, for every subset $S\subseteq [n]$ the size of $\cap_{i\in S}A_i$.
That is, we need to know the number of ways to pass out the backpacks
so that student $i$ gets the correct one for each $i$ in $S$.  It
won't matter whether or not other students get the correct backpacks,
so we can just assume that for each $i\in S$, student $i$ gets the
correct backpack and then hand out the remaining $n-|S|$ backpacks to
the remaining $n-|S|$ students in $(n-|S|)!$ ways.  Thus $(n-|S|)!$ is
$|\cap_{i:i\in S}A_i|$.  Using our formula from Problem
\ref{inclusion-exclusionunion} we get
\begin{eqnarray*}\left|\bigcup_{i=1}^n A_i \right| &=& \sum_{S:
S\subseteq [n], S\not=\emptyset}
(-1)^{|S|-1}\left|\bigcap_{i:i\in S} A_i \right|\\ &=&\sum_{S:
S\subseteq [n], S\not=\emptyset}(-1)^{|S|-1} (n-|S|)!\\ &=&
\sum_{s=1}^n {n\choose
s}(-1)^{s-1}(n-s)!\\&=&\sum_{s=1}^n
(-1)^{s-1}{n!\over s!}\end{eqnarray*} Dividing this by
$n!$, the total number of ways to pass back the backpacks, we get
$\displaystyle \sum_{s=1}^n {(-1)^{s-1}\over s!}$ for the probability
that at least one student gets the correct backpack.  Subtracting from
1 to get the probability that no student gets the correct backpack
gives us $$1- \sum_{s=1}^n {(-1)^{s-1}\over
s!}=\sum_{s=0}^n{(-1)^s\over s!}.$$}
\itemi As the number of students becomes large, what does the probability
that no student gets the correct backpack approach? 
\solution{From calculus, we know that $e^x=\sum_{j=0}^\infty {x^j\over
j!}$.  Substituting $x=-1$ gives us $e^{-1}=\sum_{j=0}^\infty
{(-1)^j\over j!}$ which is the limit as $n$ becomes infinite of the
probability in the solution to Problem \ref{hatcheck}.  Thus the
probability approaches $1/e$.}
  \ep
The formula you have given in Problem \ref{inclusion-exclusionunion} is
often called {\bf the principle of inclusion and
exclusion}\index{inclusion and exclusion principle
!for unions of sets}\index{principle of inclusion and exclusion!for
unions of sets} for unions of sets.  The reason is the pattern in which
the formula first adds (includes) all the sizes of the sets, then
subtracts (excludes) all the sizes of the intersections of two sets, then
adds (includes) all the sizes of the intersections of three sets, and so
on.  Notice that we haven't yet proved the principle.  We will first
describe the principle in an apparently more general situation that
doesn't require us to translate each application into the language of
sets.  While this new version of the principle might seem more general
than the principle for unions of sets; it is equivalent.  However once one
understands the notation used to express it, it is more convenient to
apply.

Problem \ref{hatcheck} is ``classically'' called the {\em hatcheck
problem}\index{hatcheck problem}; the name comes from substituting hats
for backpacks.  If is also sometimes called the {\em derangement
problem}\index{derangement problem}.  A {\em
derangement}\index{derangement} of an
$n$-element set is a permutation of that set (thought of as a bijection)
that maps no element of the set to itself.  One can think of a way of
handing back the backpacks as a permutation $f$ of the students: $f(i)$ is
the owner of the backpack that student $i$ receives.  Then a derangement
is a way to pass back the backpacks so that no student gets his or her
own.

\subsection{The hatcheck problem restated}  The last question in Problem
\ref{hatcheck} requires that we compute the number of ways to hand back
the backpacks so that nobody gets his or her own backpack.  We can think
of the set of ways to hand back the backpacks so that student $i$ gets
the correct one as the set of permutations of the backpacks with the
property that student $i$ gets his or her own backpack.  Since there are
$n-1$ other students and they can receive any of the remaining $n-1$
backpacks in $(n-1)$ ways, the number of permutations with the property
that student $i$ gets the correct backpack is $(n-1)!$. How many
permutations are there with the properties that student $i$ gets the
correct backpack {\em and} student $j$ gets the correct backpack?  (Let's
call these properties $i$ and $j$ for short.)  Since there are $n-2$
remaining students and $n-2$ remaining backpacks, the number of
permutations with properties $i$ and $j$ is $(n-2)!$.  Similarly, the
number of permutations with properties $i_1,i_2,\ldots,i_k$ is $(n-k)!$. 
Thus when we compute the size of the union of the sets
$$S_i=\{f: f\mbox{ is a permutation with property }i\},$$ we are
computing the number of ways to pass back the backpacks so that at least
one student gets the correct backpack.  This answers the first question
in Problem \ref{hatcheck}.  The last question in Problem \ref{hatcheck}
is asking us for the number of ways to pass back the backpacks that have
{\em none} of the properties.  To say this in a different way, the
question is asking us to compute the number of ways of passing back the
backpacks that have exactly the {\em empty set}, $\emptyset$, of
properties.  

\subsection{Basic counting functions: $N_{\mbox{at least}}$ and
$N_{\mbox{exactly}}$}
Notice that the quantities that we were able to count easily were the
number of ways to pass back the backpacks so that we satisfy a certain
subset
$K= \{i_1,i_2,\ldots,i_k\}$ of our properties.  In fact, among the $(n-k)$
ways to pass back the backpacks with this particular set $K$ of
properties is the permutation that gives each student the correct
backpack, and has not just the properties in $K$, but the whole set of
properties.  Similarly, for any set $M$ of properties with $K\subseteq
M$, the permutations that have all the properties in $M$ are among the
$(n-k)!$ permutations that have the properties in the set $K$.  Thus we
can think of $(n-k)!$ as counting the number of permutations that have
{\em at least} the properties in $K$.  In particular, $n!$ is the number
of ways to pass back the backpacks that have at least the empty set of
properties.  We thus write $N_{\mbox{at least}}(\emptyset)=n!$, or
$N_{\mbox{a}}(\emptyset)=n!$ for short.  For a $k$-element subset $K$ of
the properties, we write $N_{\mbox{at least}}(K) =(n-k)!$ or
$N_{\mbox{a}}(K) =(n-k)!$ for short.

The question we are trying to answer is ``How many of the
distributions of backpacks have exactly the empty set of properties?" 
For this purpose we introduce one more piece of notation.  We use
$N_{\mbox{exactly}}(\emptyset)$ or $N_{\mbox{e}}(\emptyset)$ to stand for
the number of backpack distributions with exactly the empty set of
properties, and for any set $K$ of properties we use
$N_{\mbox{exactly}}(K)$ or $N_{\mbox{e}}(K)$ to stand for the number of
backpack distributions with exactly the set $K$ of properties.  Thus
$N_{\mbox{e}}(K)$ is the number of distributions in which the students
represented by the set $K$ of properties get the correct backpacks back
and no other students do.  

\subsection{The principle of inclusion and exclusion for properties}
For the principle of inclusion and exclusion for properties, suppose we
have a set of arrangements (like backpack distributions) and a set $P$ of
properties (like student $i$ gets the correct backpack) that the
arrangements might or might not have.  We suppose that we know (or can
easily compute) the numbers $N_{\mbox{a}}(K)$ for every subset $K$ of
$P$.  We are most interested in computing $N_{\mbox{e}}(\emptyset)$, the
number of arrangements with none of the properties, but it will turn out
that with no more work we can compute $N_{\mbox{e}}(K)$ for every subset
$K$ of $P$.  Based on our answer to Problem \ref{hatcheck} we expect that
\begin{equation}
N_{\mbox{e}}(\emptyset) = \sum_{S: S\subseteq P} (-1)^{|S|}N_{\mbox{a}}(S)
\label{incexempty}\end{equation}
and it is a natural guess that, for every subset $K$ of $S$,
\begin{equation}
N_{\mbox{e}}(K) = \sum_{S: K\subseteq S\subseteq P}
(-1)^{|S|-|K|}N_{\mbox{a}}(S).\label{incexnonempty}\end{equation}
Equations  \ref{incexempty} and \ref{incexnonempty} are called {\bf the
principle of inclusion and exclusion for properties}.\index{inclusion and
exclusion principle!for properties}\index{principle of inclusion and
exclusion!for properties}
\bp
\itemm Verify that the formula for the number of ways to pass back the
backpacks in Problem
\ref{hatcheck} so that nobody gets the correct backpack has the form of
Equation
\ref{incexempty}.
\solution{Suppose we take property $i$ to be the property that student $i$
gets the correct backpack.  In our solution to Problem \ref{hatcheck} we
thought of $S$ as a subset of $[n]$.  Since we are numbering our
properties from $1$ to $n$, each set $S$ determines a corresponding set
$S'$ of properties: those properties numbered by the elements of $S$.  We
will use $P$ to stand for the set of all properties, that is property 1
through property $n$. The number of ways to pass back the hats so that
nobody gets the correct hat is $n!$ minus the number of ways to pass back
the hats so that somebody does get the correct hat back.  From our
solution to Problem
\ref{hatcheck}, we see that one way to express this is as
   \begin{eqnarray*}
n!-\sum_{S:
S\subseteq [n], S\not=\emptyset}(-1)^{|S|-1} (n-|S|)!&=&
n!+\sum_{S:
S\subseteq [n], S\not=\emptyset}(-1)^{|S|} (n-|S|)!\\&=&
\sum_{S:
S\subseteq [n]}(-1)^{|S|} (n-|S|)!\\
&=&\sum_{S:
S\subseteq [n]}(-1)^{|S|} N_{\mbox{a}}(S')\\&=&
\sum_{S':
S'\subseteq P}(-1)^{|S|} N_{\mbox{a}}(S')
  \end{eqnarray*}
because $N_{\mbox{a}}(S')$ is the number of ways to pass back the backpacks
so that at least the students in $S$ get the correct backpacks, which is
the same as the size of $\bigcap_{i:i\in S} A_i$, and which we computed to
be $(n-|S|)!$.}

\iteme Find a way to express $N_{\mbox{a}}(S)$ in terms of
$N_{\mbox{e}}(J)$ for subsets $J$ of $P$ containing $S$.  In particular,
what is the equation that expresses $N_{\mbox{a}}(\emptyset)$ in terms of
$N_{\mbox{e}}(J)$ for subsets $J$ of $P$?\label{incexsystemeq}
\solution{$N_{\mbox{a}}(S) = \sum_{J: S\subseteq
J\subseteq P}N_{\mbox{e}}(J)$.  In particular, we get by substitution  that
$N_{\mbox{a}}(\emptyset)=\sum_{J:J\subseteq P} N_{\mbox{e}}(J)$.}

\iteme Substitute the formula for $N_{\mbox{a}}$ from Problem
\ref{incexsystemeq} into the right hand sides of the formulas of Equations
\ref{incexempty} and
\ref{incexnonempty} and simplify what you get to show that for Equations
\ref{incexempty} and \ref{incexnonempty} the right-hand sides are indeed
equal to the left-hand sides.  This will prove that those equations are
true. (Hint:  You will get a double sum.  If you can figure out how to
reverse the order of the two summations, the binomial theorem may help you
simplify the formulas you get.)
\solution{Since Equation \ref{incexempty} is a special case of Equation
\ref{incexnonempty}, we will just substitute the formula from Problem
\ref{incexsystemeq} into the right-hand side of Equation
\ref{incexnonempty}.  This gives us
\begin{eqnarray*}\hspace*{-.25 in}\sum_{S: K\subseteq S\subseteq P}
(-1)^{|S|-|K|}N_{\mbox{a}}(S)&=&\sum_{S: K\subseteq S\subseteq P}
(-1)^{|S|-|K|}\sum_{J: S\subseteq
J\subseteq P}N_{\mbox{e}}(J)\\
&=&\sum_{S: K\subseteq S\subseteq P}
\sum_{J: S\subseteq
J\subseteq P}(-1)^{|S|-|K|}N_{\mbox{e}}(J)\\
&=&\sum_{J:K\subseteq J\subseteq P}\sum_{S:K\subseteq S\subseteq J}
(-1)^{|S|-|K|}N_{\mbox{e}}(J)\\
&=&\sum_{J:K\subseteq J\subseteq P}N_{\mbox{e}}(J)\sum_{S:K\subseteq
S\subseteq J} (-1)^{|S|-|K|}\\
&=&\sum_{J:K\subseteq J\subseteq P}N_{\mbox{e}}(J)\sum_{s=|K|}^{|J|}
{|J|-|K|
\choose s-|K|}(-1)^{s-|K|}\\
&=&\sum_{J:K\subseteq J\subseteq P}N_{\mbox{e}}(J)\sum_{i=0}^{|J|-|K|}
{|J|-|K|
\choose i}(-1)^{i}\\
&=&\sum_{J:K\subseteq J\subseteq
P}N_{\mbox{e}}(J)(1-1)^{|J|-|K|}\\&=&N_{\mbox{e}}(K).
\end{eqnarray*}
In the fourth-to-last line of our equations, we used the fact that the
number of subsets $S$ of $J$ that contain $K$ is the number of ways to
choose the elements of the set $S-K$ of elements in $S$ but not $K$ from
the elements of $J-K$, the set of elements of $J$ that are not in $K$.  In
the second-to-last line of the equations, we recognized that the second
sum in the third-to-last line is the kind of sum we would get by applying
the binomial theorem to something of the form $(x+y)^{|J|-|K|}$ for a
suitable $x$ and $y$, and then saw that $x=1$ and $y=-1$ would give us
exactly the second sum in the third-to-last line.  In going from the
second-to-last line to the last line we used the facts that $0^n=o$ if $n>0$
and $0^0=1$.  This proves that Equation
\ref{incexnonempty} is true for all
$K$, and in particular when
$K=\emptyset$, which also proves Equation \ref{incexempty}.  We have just
proved the Principle of inclusion and exclusion.}

\item In how many ways may we distribute $k$ identical apples to $n$
children so that no child gets more than three apples?
\solution{Let property $i$ be ``Child $i$ gets four or more apples."  Then
we are asking for the number of ways to pass out the apples that have none
of the properties, so we are asking for $N_{\mbox{e}}(\emptyset)$.  From
Equation \ref{incexempty} we see that to find this number we need to know
$N_{\mbox{a}}(S)$ for every subset $S$ of our set of properties.  But if
$S$ has size $s$, then having the properties in $S$ means that all the
children in a particular set of size $s$ will get four or more apples.  We
already know how to pass out the apples so that the children in a
particular set
$\hat S$ of children get at least four apples:  we give everyone in $\hat
S$ four apples, and then pass out the remaining $k-4s$ apples to the
children in ${k-4s+n-1\choose k-4s}= {k-4s+n-1\choose n-1}$ ways. This
counts the number of ways to give at least four apples to every child in
$\hat S$, and maybe give four apples to some other children as well. Thus
$N_{\mbox{a}}(S) = {k-4|S|+n-1\choose n-1}$.  Applying Equation
\ref{incexempty} gives us
\begin{eqnarray*} N_{\mbox{e}}(\emptyset)&=&\sum_{S:S\subseteq
P}(-1)^{|S|}N_{\mbox{a}}(S)\\
&=&\sum_{s=0}^k {k\choose s}(-1)^s {k-4s+n-1\choose n-1}\\
&=&\sum_{s=0}^k (-1)^s{k\choose s}{k-4s+n-1\choose n-1}
\end{eqnarray*}}

\itemi A group of $n$ married couples comes to a group discussion session
where they all sit around a round table.  In how many ways can they sit
so that no person is next to his or her spouse? 
(Note that two people of the same sex can sit next to
each other.)\label{relaxedmenage}
\solution{Let Property $i$ be that couple $i$ sits together.  We are
interested in the number of seating arrangements that have none of the
properties.  Thus for a set $S$ of properties, we need to compute
$N_{\mbox{a}}(S)$.  If we let each couple described by $S$ sit together,
we will seat $|S|$ couples and $2n-2|S|$ individuals around the table.  We
can do this in $2^{|S|}(|S| + 2n-2 |S|-1)!$ ways, because once we chose a
place for a couple (i.e. two adjacent seats) there are two ways the couple
can sit down.  Thus $N_{\mbox{a}}(S) =2^{|S|}(2n-|S|-1)!$. 
Substituting this into Equation \ref{incexempty} with $P$ as the set of
all properties gives us
\begin{eqnarray*}N_{\mbox{e}}(\emptyset) &=& \sum_{S:S\subseteq P}
(-1)^{|S|} N_{\mbox{a}}(S)\\
&=&\sum_{s=0}^n(-1)^s{n\choose s}2^{s}(2n- s-1)!
\end{eqnarray*}}

\itemih A group of $n$ married couples comes to a group
discussion session where they all sit around a round table.  In how many
ways can they sit so that no person is next to his or her spouse or a
person of the same sex?  This problem is called the {\em menage
problem}.\index{menage problem}  (Hint:  Reason somewhat as you did in Problem
\ref{relaxedmenage}, noting that if the set of couples who do sit
side-by-side is nonempty, then the sex of the person at each place at the
table is determined once we seat one couple in that set.)
\solution{We are going to consider arrangements of the couples alternating
sex around the table.  Property $i$ of an arrangement is that couple $i$
sits together.  We are interested in the number of arrangements that have
exactly none of the properties.  Thus for each subset $S$ of the
properties, we consider the number of arrangements that has at least the
set $S$ of properties.  We distinguish the case that $S$ is empty from the
others.  $N_{\mbox{a}}(\emptyset)$ is just the number of ways to seat $2n$
couples around the table, alternating sex, but with no other
restrictions.  We can arrange one of the sexes in a circle in $n-1!$ ways
and then assign the members of the opposite sex to the places between them
in  $n!$ ways, so $N_{\mbox{a}}(\emptyset) = (n-1)!n!$. (Another way to
get this result is to let one person sit down.  This determines the sex
of the person at each place of the table, so there are $(n-1)!$ ways to
assign the people of the same sex of the first person, and $n!$ ways to
assign the people of the opposite.  It appears that there are $2n$
choices for where the first person sits, but we can break the seating
charts up into blocks of $2n$ seating charts, each of which gives the
same circular arrangement.  Thus there are $(n-1)!n!$ inequivalent ways
to seat the people with at least the empty set of properties.)

   Now if
$S$ is nonempty, and has $s$ members, we seat one of the couples that must
sit together (say the first in alphabetical order), and this determines the
sex of the person that must sit at each other place.  There are $2n$ pairs
of adjacent seats where we can seat that couple and two ways they can sit
in the pair of adjacent seats that we choose.  Then we have $s-1$ couples,
$n-s$ men and $n-s$ women to seat in the remaining places.  First we
arrange the $s-1$ couples and $2n-2s$ identical empty chairs in places at
the table in $(2n-2s+s-1)!/(2n-2s)!=(2n-s-1)!/(2n-2s)!$ ways.  Each couple
can sit in only one way in the places they have chosen, because the sex of
the person in a given place has been determined by how the first couple
sits.  The sex of the person in each of the remaining chairs has been
determined, so we assign the men to their seats in $(n-s)!$ ways and we
assign the women to their seats in $(n-s)!$ ways.  Thus we have
$2\cdot2n(2n-s-1)!(n-s)!^2/(2n-2s)!$ ways to place the people.  But we can
partition the placements into blocks of $2n$ equivalent placements,
because shifting everyone the same number of places to the right or left
gives an equivalent placement.  Thus the number of inequivalent seating
arrangements is
   \begin{eqnarray*}
{2(2n-s-1)!(n-s)!^2\over(2n-2s)!}&=&{2(2n-s-1)!(n-s)!^2\over
2(n-s)(2n-2s-1)!}\\&=&{(2n-s-1)!(n-s)!(n-s-1)!\over(2n-2s-1)!}.
    \end{eqnarray*}
 Notice that
if we take
$s=0$, this formula reduces to $(n-1)!n!$.  Thus for all sets $S$
$$N_{\mbox{a}}(S)={(2n-s-1)!(n-s)!(n-s-1)!\over(2n-2s-1)!}.$$
Then from Equation \ref{incexempty}
\begin{eqnarray*}
N_{\mbox{e}} &=& \sum_{S:S\subseteq P} (-1)^{|S|}{(2n-|S|-1)!(n-|S|)!(n-|S|-1)!
\over(2n-2|S|-1)!}\\
&=&\sum_{s=0}^n(-1)^s{n\choose s}{(2n-s-1)!(n-s)!(n-s-1)!\over(2n-2s-1)!}\\
&=&\sum_{s=0}^n(-1)^s{n!\over
s!(n-s)!}{(2n-s-1)!(n-s)!(n-s-1)!\over(2n-2s-1)!}\\
&=&\sum_{s=0}^n(-1)^s {n!(2n-s-1)!(n-s-1)!\over s!(2n-2s-1)!}\\
&=&\sum_{s=0}^n(-1)^s{2n-s-1\choose s}n!(n-s-1)!
\end{eqnarray*} is the number of ways to seat the people, alternating sex, so
that no couple sits together.}

\ep

\subsection{Counting onto functions}
\bp
\iteme Given a function $f$ from the $k$-element set $K$ to the
$n$-element set $[n]$, we say $f$ has property $i$ if $f(x)\not= i$ for
every $x$ in
$K$.  How many of these properties does an onto function have?  What is
the number of functions from a $k$-element set onto an $n$-element
set?\index{onto functions!number of}\index{surjections!number
 of}\index{functions!onto!number of}\label{numontofun}
\solution{An onto function has none of these properties.  Thus, using this set
$P$ of properties, the number of onto functions is
$N_{\mbox{e}}(\emptyset)$.  For a set $S$ of these properties,
$N_{\mbox{a}}(S)$ is the number of functions from $K$ to $[n]-S'$, where $S'$
is the set of all $i$ such that property $i$ is in $S$.  This number of
functions is $(n-|S|)^k$.  Thus by Equation
\ref{incexempty} 
    \begin{eqnarray*}
N_{\mbox{e}}(\emptyset) &=& \sum_{S:s\subseteq P} (-1)^{|S|}
(n-|S|)^k\\
&=&\sum_{s=0}^n (-1)^s{n\choose s}(n-s)^k
   \end{eqnarray*}
is the number of functions from $K$ onto $[n]$.}

\itemi  Find a formula for the Stirling number (of the second kind)
$S(k,n)$.\index{Stirling Number!second kind} 
\solution{Since the number of functions from $[k]$ onto $[n]$ is $S(k,n)n!$,
we get from the solution to Problem \ref{numontofun}
$$S(k,n) = {1\over n!}\sum_{s=0}^n (-1)^s{n\choose s}(n-s)^k.$$}
\ep

\subsection{The chromatic polynomial of a graph}
We defined a graph to consist of set $V$ of elements called
vertices and a set $E$ of elements called edges such that each edge joins
two vertices.  A {\em coloring}\index{coloring of a
graph}\index{graph!coloring of} of a graph by the elements of a set
$C$ (of colors) is an assignment of an element of $C$ to each vertex of
the graph; that is, a function from the vertex set $V$ of the graph to
$C$.  A coloring is called {\em proper}\index{coloring of a
graph!proper}\index{graph!coloring of!proper}\index{proper coloring of 
a graph} if for each edge joining two distinct vertices\footnote{If a
graph had a loop connecting a vertex to itself, that loop would connect a
vertex to a vertex of the same color.  It is because of this that we only
consider edges with two distinct vertices in our definition.}, the two
vertices it joins have different colors. You may have heard of the famous
four color theorem of graph theory that says if a graph may be drawn in
the plane so that no two edges cross (though they may touch at a vertex),
then the graph has a proper coloring with four colors.  Here we are
interested in a different, though related, problem: namely, in how many
ways may we properly color a graph (regardless of whether it can be drawn
in the plane or not) using
$k$ or fewer colors?  When we studied trees, we restricted ourselves to
connected graphs. (Recall that a graph is connected if, for each pair of
vertices, there is a walk between them.)  Here, disconnected graphs will
also be important to us.  Given a graph which might or might not be
connected, we partition its vertices into blocks called {\em connected
components}\index{connected component of a graph}\index{graph!connected
component of} as follows.  For each vertex
$v$ we put all vertices connected to it by a walk into a block together. 
Clearly each vertex is in at least one block, because vertex
$v$ is connected to vertex $v$ by the trivial walk consisting of the
single vertex $v$ and no edges.  To have a partition, each vertex must be
in one and only one block.  To prove that we have defined a
partition, suppose that vertex
$v$ is in the blocks
$B_1$ and $B_2$.  Then $B_1$ is the set of all
vertices connected by walks to some vertex $v_1$ and $B_2$ is the set of
all vertices connected by walks to some vertex $v_2$.  
\bp
\itemes (Relevant in Appendix \ref{expogenfun} as well as this section.) Show
that
$B_1=B_2$.\label{conncomp}
\solution{Since $v$ is in $B_1$, there is a walk from $v_1$ to $v$.  Since
there is a walk from every  vertex in $B_1$ to $v_1$, there is a walk from
every vertex in in $B_1$ to $v$.  But there is a walk from $v$ to $v_2$ since
$v\in B_2$.  Thus there is a walk from every vertex in $B_1$ to $v_2$.  Then
by our description of $B_2$ just before the problem, every vertex in $B_1$ is
also in $B_2$.  A similar argument shows that every vertex in $B_2$ is also in
$B_1$.  Thus $B_1=B_2$.}
\ep
Since $B_1=B_2$, these two sets are the same block, and thus all blocks
containing $v$ are identical, so $v$ is in only one block.  Thus we have
a partition of the vertex set, and the blocks of the partition are the
connected components of the graph.  Notice that the connected components
depend on the edge set of the graph.  If we have a graph on the vertex
set $V$ with edge set
$E$ and another graph on the vertex set
$V$ with edge set $E'$, then these two graphs could have different
connected components.  It is traditional to use the Greek letter
$\gamma$ (gamma)\footnote{The greek  letter gamma is pronounced gam-uh,
where gam rhymes with ham.} to stand for the number of connected
components of a graph; in particular,  $\gamma(V,E)$ stands for the
number of connected components of the graph with vertex set $V$ and edge
set $E$.  We are going to show how the principle of inclusion and
exclusion may be used to compute the number of ways to properly color a
graph using colors from a set $C$ of $c$ colors.  
\bp  
\itemes Suppose  we have a graph G with vertex set V and edge set $E$. 
Suppose $F$ is a subset of $E$.  Suppose we have a set $C$ of $c$ colors
with which to color the vertices.
\begin{enumerate}
  \item In terms of $\gamma(V,F)$, in how many
ways may we color the vertices of $G$ so that each edge in $F$ connects
two vertices of the same color?
\solution{For each edge in $F$ to connect two vertices of the same color, we
must have all the vertices in a connected component of the graph with vertex
set $V$ and edge set $F$ colored the same color.  Thus the number of such
colorings is
$c^{\gamma(V,F)}$.}
  \item Given a coloring of $G$, for each edge $e$ in $E$, let us consider
the property that the endpoints of $e$ are colored the same color.  Let us
call this property ``property $e$."  In this way each set of properties
can be thought of as a subset of $E$.  What set of properties does a
proper coloring have?  
\solution{A proper coloring has none of the properties.}
\item Find a formula (which may involve summing over all subsets $F$ of
the edge set of the graph and using the number $\gamma(V,F)$ of connected
components of the graph with vertex set $V$ and edge set $F$) for the
number of proper colorings of
$G$ using colors in the set $C$.\label{chromaticpoly}
\solution{$N_{\mbox{e}}(\emptyset)=\sum_{F:F\subseteq E}
 (-1)^{|F|}c^{\gamma(V,F)}.$}
\end{enumerate}

\ep

The formula you found in Problem \ref{chromaticpoly} is a formula that
involves powers of $c$, and so it is a polynomial function of $c$. Thus
it is called the ``chromatic polynomial\index{graph!chromatic
polynomial of}\index{chromatic polynomial of a graph} of the graph
$G$.    Since we like to think about polynomials as having a
variable $x$ and we like to think of $c$ as standing for some constant,
people often use $x$ as the notation for the number of colors we are
using to color $G$. Frequently people will use
$\Chi_G(x)$ to stand for the number of way to color
$G$ with $x$ colors, and call $\Chi_G(x)$ the {\em chromatic polynomial} of
$G$.  

\bp

\itemi In Chapter 2 we introduced the deletion-contraction
recurrence\index{deletion-contraction recurrence} for counting spanning
trees of a graph.  Figure out how the chromatic polynomial of a graph is
related to those resulting from deletion of an edge $e$ and from
contraction of that same edge $e$.  Try to find a recurrence like the one
for counting spanning trees that expresses the chromatic polynomial of a
graph in terms of the chromatic polynomials of $G-e$ and $G/e$ for an
arbitrary edge $e$.  Use this recurrence to give another proof that the
number of ways to color a graph with $x$ colors is a polynomial function
of $x$.  \label{chrompolydel/cont}
\solution{The number of colorings of $G-e$ is equal to the number of proper
colorings of $G$ plus the number of colorings of $G$ that are proper except
for giving both ends of $e$ the same color.  But the number of colorings of
$G$ that are proper except for giving both ends of $e$ the same color is the
number of proper colorings of $G/e$.  Therefore $\Chi_{G-e}(x) =\Chi_G(x)
-\Chi_{G/e}(x)$.  This gives us $\chi_G(x) = \Chi_{G-e}(x)
-\Chi_{G/e}(x)$. We can use this to prove inductively that $\Chi_G(x)$ is a
polynomial in $x$.  If $G$ has one vertex, then the number of ways to color
$G$ properly with $x$ colors is $x$.  This is a polynomial in $x$.  Now
suppose inductively that $G$ has more than one vertex and whenever a graph $H$
has fewer vertices than
$G$, the function $\Chi_H(x)$ is a polynomial function in $x$.  Then
$\Chi_G(x)= \chi_{G-e}(x)-\Chi_{G/e}(x)$, is a difference of of two polynomial
functions in $x$, so it is a polynomial function in $x$.  Therefore by the
principle of mathematical induction, for all graphs $G$ on a finite vertex
set, the number of ways to properly color $G$ in $x$ colors is a polynomial in
$x$.}



\item Use the deletion-contraction recurrence to compute the
chromatic polynomial of the graph in Figure \ref{del-cont}.  (You can
simplify your computations by thinking about the effect on the chromatic
polynomial of deleting an edge that is a loop, or deleting one of several
edges between the same two vertices.)
\solution{If a graph has a loop it has no proper colorings.  The graph in
Figure \ref{del-cont} has no loops and no multiple edges between two
vertices.  The only way we could get a loop is by contracting one of several
multiple edges between two vertices, and the resulting graph would have no
contribution to the chromatic polynomial of the original graph.  Thus whenever
a contraction gives us a graph with multiple edges between two vertices, we
can replace the multiple edges by one edge and go on with our computation from
there.  The graphs we get when we delete and contract the edges $\{1,2\}$ and
$\{2,3\}$ are $(G-\{1,2\})-\{2,3\}$, $(G-\{1,2\})/\{2,3\}$,
$(G-\{2,3\})/\{1,2\}$,  and $(G/\{2,3\})/\{1,2\}$.  These are shown in the
following picture.
\begin{center}\mbox{\psfig{figure=spantreeexerciseresult.eps%,height=1.0in
}}\end{center}
The chromatic polynomial of a triangle is $x(x-1)(x-2)$ because
for one vertex we have $x$ colors, for a second we have $x-1$, and for the
third vertex, because it is adjacent to both of the other vertices,  we have
$x-2$ choices of colors. For a vertex of degree 1 there are $x-1$ choices of
colors, those colors not used on the one vertex it is adjacent to.  As we
mentioned, the extra edges do not change the chromatic polynomial, so we have
that the chromatic polynomial of $(G-\{1,2\})-\{2,3\}$ is $x(x-1)^3(x-2)$, the
chromatic polynomial of $(G-\{1,2\})/\{2,3\}$ is $x(x-1)^2(x-2)$, as is that of
$(G-\{2,3\})/\{1,2\}$, and the chromatic polynomial of $(G/\{2,3\})/\{1,2\}$ is
$(x-1)(x-2)(x-3)$.  Using the deletion-contraction recurrence, we get that
\begin{eqnarray*}\Chi_G(x) &=& \Chi_{G-\{1,2\}}(x) - \Chi_{G/\{1,2\}}(x)\\&=&
\Chi_{G-\{1,2\}-\{2,3\}}(x)-\Chi_{(G-\{1,2\})/\{2,3\}}(x)-
\Chi_{(G/\{1,2\})-\{2,3\}}(x)\\ &+& \Chi_{G/\{1,2\}/\{2,3\}}(x)\\
&=& x(x-1)^3(x-2)-2x(x-1)^2(x-2) + x(x-1)(x-2)\\
&=& x(x-1)(x-2)(x^2-2x+1 +2x-2 +1)\\
&=& x^3(x-1)(x-2)
\end{eqnarray*}
for the chromatic polynomial of $G$.}



\begin{figure}[htb]\caption{A
graph.}\label{del-cont}\vglue-1in
\begin{center}\mbox{\psfig{figure=spantreeexercise.eps%,height=1.0in
}}\end{center}\end{figure}

\itemi In how many ways may you properly color the vertices of a path on
$n$ vertices with
$x$ colors? Describe any dependence of the chromatic polynomial of a path
on the number of vertices. In how many ways may you properly color the
vertices of a cycle on $n$ vertices with
$x$ colors?  Describe any dependence of the chromatic polynomial of a
cycle on the number of vertices. 
\solution{To color the vertices of a path, start at one end.  There are $x$
colors for that vertex, and $x-1$ colors for each of the next $n-1$, since
each of them must be different from the preceding one.  Thus the chromatic
polynomial of a path on $n$ vertices is $x(x-1)^{n-1}$.  The dependence on the
number of vertices appears in the exponent on $x-1$.  If we use $C_n$ to stand
for a path on $n$ vertices and $P_n$ to stand for a path on $n$ vertices, then
by the deletion-contraction recurrence, we may write 
\begin{eqnarray*}\hspace*{-.4 in}\Chi_{C_n}(x) &=&
\Chi_{P_n}(x)-\Chi_{C_{n-1}}(x)\\
&=&\Chi_{P_n}(x)-\Chi_{P_{n-1}}(x)+\Chi_{C_{n-2}}(x)\\&=&
\Chi_{P_n}(x)-\Chi_{P_{n-1}}(x)+\Chi_{P_{n-3}}(x)-
\cdots+(-1)^{n-3}(\Chi_{P_3}(x)-\Chi_{C_2}(x))\\
&=&x(x-1)^{n-1}-x(x-1)^{n-2}+x(x-1)^{n-3}\cdots\\
&+&(-1)^{n-3}[x(x-1)^2-x(x-1)]\\
&=&x(x-1)\sum_{i=0}^{n-2}(x-1)^i(-1)^{n-2-i}\\
&=&x(x-1)(-1)^{n-2}\sum_{i=0}^{n-2}(1-x)^i\\
&=&x(x-1)(-1)^{n-2}{1-(1-x)^{n-1}\over 1-(1-x)}\\
&=&(x-1)[(x-1)^{n-1}+(-1)^n].\end{eqnarray*} Here the dependence on $n$ is
interesting; effectively, we are taking $(x-1)$ times the result of dropping
the constant term from $(x-1)^{n-1}$.}

\item In how many ways may you properly color the vertices of a tree on
$n$ vertices with
$x$ colors?
\solution{Color an arbitrary vertex; you have $x$ choices for the color of
that vertex. No two vertices adjacent to it are adjacent (otherwise we'd have
a cycle), so for for each of them you have $x-1$ choices of colors.  No two
vertices adjacent to colored vertices  are adjacent to each other, nor is
one of them adjacent to two colored vertices (in either case you'd have a
cycle), so for each of them you'd have $x-1$ colors.  You can continue this
argument until all vertices are colored, so you have $x(x-1)^{n-1}$ ways to
color the vertices.

You can also prove by induction that the chromatic polynomial of a tree is
$x(x-1)^{n-1}$.  This is clearly true if there is one vertex.  Otherwise,
choose a vertex of degree 1 in an $n$-vertex tree and remove it. You may
inductively assume that the chromatic polynomial of the remaining tree is
$x(x-1)^{n-2}$. Now there are $x-1$ choices for the color of the vertex you
removed since it has degree 1, and so the chromatic polynomial of the tree is
$x(x-1)^{n-1}$.  There is also an inductive argument in which you delete and
contract an arbitrary edge.}

\itemi What do you observe about the signs of the coefficients of the
chromatic polynomial of the graph in Figure \ref{del-cont}?  What about
the signs of the coefficients of the chromatic polynomial of a path?  Of
a cycle?  Of a tree?  Make a conjecture about the signs of the
coefficients of a chromatic polynomial and prove it.
\solution{not all powers of $x$ appear,  but the signs alternate as the power
of
$x$ increases; that is, the sign of
$x^i$ is opposite that of $x^{i+1}$. More precisely, if $c_i$ is the
coefficient of $x^i$, then $(-1)^{n-i}c_i\ge 0$. To prove this, note it is
trivially true for a graph with no edges.  Choose an edge $e$ of
$G$.  Then
$\Chi_G(x) =
\Chi_{G-e}(x)-\Chi_{G/e}(x)$.  In $G-e$, we may assume inductively that
$(-1)^{n-i}c'_i\ge0$ and in $G-/e$ we can assume inductively that
$c''_i(-1)^{n-1-i}\ge0$, where we use $c'_i$ and $c''_i$ as the coefficient of
$x^i$ in $\Chi_{G-e}(x)$ and $\Chi_{G/e}(x)$, respectively.  Then $c_i=c'_i
-c''_i$, and
$$c_i(-1)^{n-i}=c'_i(-1)^{n-i}-c''_i(-1)^{n-i}=c'_i(-1)^{n-i}+c''_i(-1)^{n-1-i}
\ge0.$$ Therefore by the principle of mathematical induction, $c_i(-1)^i\ge0$
for all finite graphs.}
\ep 


\section{The Idea of Generating Functions} Suppose you are going to choose
three pieces of fruit from among apples, pears and bananas for a snack. 
We can symbolically represent all your choices as
$$\ap\ap\ap+\pe\pe\pe+\ba\ba\ba+\ap\ap\pe+\ap\ap\ba+\ap\pe\pe +\pe\pe\ba
+\ap\ba\ba+\pe\ba\ba+\ap\pe\ba.$$
Here we are using a picture of a piece of fruit to stand for taking a
piece of that fruit.  Thus $\ap$ stands for taking an apple, $\ap\pe$ for
taking an apple and a pear, and
$\ap\ap$ for taking two apples.  
You can think of the plus sign as standing for the ``exclusive or," that
is,
$\ap+\ba$ would stand for ``I take an apple or a banana but not both."  To
say ``I take both an apple and a banana," we would write $\ap\ba$.  We can
extend the analogy to mathematical notation by condensing our statement
that we take three pieces of fruit to
$$\ap^3+\pe^3+\ba^3+\ap^2\pe+\ap^2\ba +\ap\pe^2+\pe^2\ba+
\ap\ba^2+\pe\ba^2 +\ap\pe\ba.$$
In this notation $\ap^3$ stands for taking a multiset of three apples,
while $\ap^2\ba$ stands for taking a multiset of two apples and a banana,
and so on.  What our notation is really doing is giving us a convenient
way to list all three element multisets chosen from the
set $\{\ap,\pe,\ba\}$.\footnote{This approach was inspired by George
P\'olya's paper ``Picture Writing," in the December, 1956 issue of the
{\em American Mathematical Monthly}, page 689. While we are taking a
somewhat more formal approach than P\'olya, it is still completely in the
spirit of his work.}

Suppose now that we plan to choose between one and three apples, between
one and two pears, and between one and two bananas.  In a somewhat clumsy
way we could describe our fruit selections as
\begin{equation} \ap\pe\ba+\ap^2\pe\ba+\cdots+\ap^2\pe^2\ba+\cdots +
\ap^2\pe^2\ba^2+\ap^3\pe\ba+
\cdots +\ap^3\pe^2\ba+\cdots +
\ap^3\pe^2\ba^2.\label{uptothreefruits}\end{equation}

\bp \iteme Using an $A$ in place of the picture of an apple, a $P$ in
place of the picture of a pear, and a $B$ in place of the picture of a
banana, write out the formula similar to Formula \ref{uptothreefruits}
without any dots for left out terms.  (You may use pictures instead of
letters if you prefer, but it gets tedious quite quickly!)  Now expand
the product
$(A+A^2+A^3)(P+P^2)(B+B^2)$ and compare the result with your formula.
\label{twopiecesoffruit}
\solution{$APB+APB^2 +AP^2B+ AP^2B^2+ A^2PB+A^2PB^2+ A^2P^2B+ A^2P^2B^2+
A^3PB+A^3PB^2 +A^3P^2B+ A^3P^2B^2$
\begin{eqnarray*}&&(A+A^2+A^3)(P+P^2)(B+B^2)\\
&=&APB+APB^2+AP^2B+AP^2B^2+A^2PB+A^2PB^2+A^2P^2B\\&+&A^2P^2B^2+
A^3PB+A^3PB^2+A^3P^2B+A^3P^2B^2.
\end{eqnarray*} We get the same expression in both cases.}

\iteme Substitute 
$x$ for all of $A$, $P$ and $B$ (or for the corresponding pictures) in
the formula you got in Problem \ref{twopiecesoffruit} and expand the
result in powers of $x$.  Give an interpretation of the coefficient
of $x^n$.
\solution{$x^3+3x^4+4x^5+3x^6+x^7$. There is one way to choose three pieces of
fruit, there are three ways to choose four pieces, four ways to chose 5
pieces, three ways to choose 6 pieces , and there is one way to choose 7
pieces of fruit.  The coefficient of $x^n$ is the number of ways to choose $n$
pieces of fruit.}
\ep

If we were to expand the formula
\begin{equation}(\ap+\ap^2+\ap^3)(\pe+\pe^2)(\ba+\ba^2).
\label{threefruitsagain}
\end{equation}
we would get Formula \ref{uptothreefruits}.  Thus formula
\ref{uptothreefruits} and formula \ref{threefruitsagain} each describe
the number of multisets we can choose from the set ${\ap,\pe,\ba}$ in
which \ap\ appears  between 1 and three times and \pe, and \ba\ each
appear once or twice.  We interpret Formula \ref{uptothreefruits} as
describing each individual multiset we can choose, and we interpret
Formula
\ref{threefruitsagain} as saying that we first decide how many apples to
take, and then decide how many pears to take, and then decide how many
bananas to take.  At this stage it might seem a bit magical that doing
ordinary algebra with the second formula yields the first, but in fact we
could define addition and multiplication with these pictures more
formally so we could explain in detail why things work out.  However
since the pictures are for motivation, and are actually difficult to
write out on paper, it doesn't make much sense to work out these
details.  We will see an explanation in another context later on.

\subsection{Picture functions}  As you've seen, in our descriptions of
ways of choosing fruits, we've treated the pictures of the fruit as if
they are variables.  You've also likely noticed that it is much easier to
do algebraic manipulations with letters rather than pictures, simply
because it is time consuming to draw the same picture over and over
again, while we are used to writing letters quickly.  In the theory of
generating functions, we associate variables or polynomials or even power
series with members of a set.  There is no standard language describing
how we associate variables with members of a set, so we shall invent
some.  By a {\em picture} of a member of a set we will mean a variable, or
perhaps a product of powers of variables (or even a sum of products of
powers of variables).  A function that assigns a picture $P(s)$ to each
member $s$ of a set $S$ will be called a {\em picture
function}\label{picturefunction}.  The {\bf picture
enumerator}\index{picture enumerator} for a picture function
$P$ defined on  a set
$S$ will be 

$$E_P(S) = \sum_{s: s\in S}  P(s).$$

We choose this language because the picture enumerator lists, or
enumerates, all the elements of $S$ according to their pictures.  Thus
Formula \ref{uptothreefruits} is the picture enumerator the set of all
multisets of fruit with between one and three apples, one and two pears,
and one and two bananas.
\bp
\itemm How would you write down a polynomial in the variable $A$ that says
you should take between zero and three apples?
\label{zerotothreeapples}\solution{$A^0+A^1+A^2+A^3$.}

\iteme How would you write down a picture enumerator that says we take
between zero and three apples, between zero and three pears, and between
zero and three bananas?\label{zerotothreefruits} 
\solution{\\
$(A^0+A^1+A^2+A^3)(P^0+P^1+P^2+P^3)(B^0+B^1+B^2+B^3).$}

\itemes (Used in Chapter \ref{groupsonsets}.) Notice that when we used $A^2$ to
stand for taking two apples, and
$P^3$ to stand for taking three pears, then we used the product
$A^2P^3$ to stand for taking two apples and three pears.  Thus we have
chosen the picture of the ordered pair (2 apples, 3 pears) to be the
product of the pictures of a multiset of two apples and a multiset of
three pears.  Show that if $S_1$ and $S_2$ are sets with picture
functions $P_1$ and $P_2$ defined on them, and if we define the picture
of an ordered pair $(x_1,x_2)\in S_1\times S_2$ to be
$P((x_1,x_2))= P_1(x_1)P_2(x_2)$, then the picture enumerator of $P$ on
the set $S_1\times S_2$ is $E_{P_1}(S_1)E_{P_2}(S_2)$.  We call this the
{\bf product principle for picture enumerators}.\index{product
principle!picture enumerators}\index{picture enumerators!product
principle for}
\solution{\begin{eqnarray*}E_P(S_1\times S_2)&=&\sum_{(x_1,x_2)\in S_1\times
S_2} P(x_1)P(x_2)\\ &=&
\sum_{x_1:x_1\in S_1}\sum_{x_2:x_2\in S_2} P(x_1)P(x_2)\\
&=&\sum_{x_1\in S_1}P(x_1)\sum_{x_2\in S_2}P(x_2)\\
&=&\sum_{x_1\in S_1} P(x_1) E_{P_2}(S_2)\\
&=&\left(\sum_{x_1\in S_1} P(x_1)\right)E_{P_2}(S_2)\\
&=&E_{P_1}(S_1)E_{P_2}(S_2)
\end{eqnarray*}}
\ep
\subsection{Generating functions}
\bp 
\iteme Suppose you are going to choose a snack of between zero and three
apples, between zero and three pears, and between zero and three
bananas.  Write down a polynomial in one variable $x$ such that the
coefficient of $x^n$ is the number of ways to choose a snack with $n$
pieces of fruit.  Hint:  substitute something for $A$, $P$ and $B$ in
your formula from Problem \ref{zerotothreefruits}.
\solution{$(1+x+x^2+x^3)^3$}
\itemm Suppose an apple costs 20 cents, a banana costs 25 cents, and a
pear costs 30 cents.  What should you substitute for $A$, $P$, and $B$ in
Problem \ref{zerotothreefruits} in order to get a polynomial in which the
coefficient of $x^n$ is the number of ways to choose a selection of fruit
that costs $n$ cents?
\solution{Substitute $x^{20}$ for $A$, $x^{25}$ for $B$ and $x^{30}$ for $P$.}
\iteme Suppose an apple has 40 calories, a pear has 60 calories, and a
banana has 80 calories.  What should you substitute for $A$, $P$, and $B$ 
in Problem \ref{zerotothreefruits} in order to get a polynomial in which
the coefficient of $x^n$ is the number of ways to choose a selection of
fruit with a total of $n$ calories?
\solution{Substitute $x^{40}$ for $A$, $x^{60}$ for $P$, and $x^{80}$ for $B$.}
\iteme We are going to choose a subset of the set $[n]=\{1,2,\ldots,
n\}$.  Suppose we use $x_1$ to be the picture of choosing 1 to be in our
subset.  What is the picture enumerator for either choosing 1 or not
choosing 1?  Suppose that for each $i$ between 1 and $n$, we use $x_i$ to
be the picture of choosing
$i$ to be in our subset.  What is the picture enumerator for either
choosing $i$ or not choosing $i$ to be in our subset?  What is the
picture enumerator for all possible choices of subsets of $[n]$?  What
should we substitute for $x_i$ in order to get a polynomial in $x$ such
that the coefficient of $x^k$ is the number of ways to choose a
$k$-element subset of $n$?  What theorem have we just
reproved (a special case of)?\label{reprovingbinomialtheorem}
\solution{The picture enumerator for choosing $1$ or not choosing 1 is
$x_1+1$.  The picture enumerator for choosing or not choosing $i$ is $x_i+1$. 
The picture enumerator for choosing all possible subsets of $[n]$ is 
$(x_1+1)(x_2+1)\cdots(x_n+1).$
We should substitute $x$ for $x_i$, thus getting $(1+x)^n$.  Since the number
of ways to choose an $n$-element subset is $n\choose k$, we have just proved
the version of the binomial theorem that says
$$(x+1)^n=\sum_{i=0}^n {n\choose i}x^i.$$}
\ep
In Problem \ref{reprovingbinomialtheorem} we see that we can think of the
process of expanding the polynomial $(1+x)^n$ as a way of ``generating"
the binomial coefficients $n\choose k$ as the coefficients of $x^k$ in the
expansion of $(1+x)^n$.  For this reason, we say that $(1+x)^n$ is the
``generating function" for the binomial coefficients $n\choose k$.  More
generally, the {\bf generating function} for a sequence $a_i$, defined for
$i$ with $0\le i\le n$ is the expression $\sum_{i=0}^n a_ix^i$, and the
{\bf generating function}\index{generating function} for the sequence
$a_i$ with $i\ge 0$ is the expression $\sum_{i=0}^\infty a_ix^i$.  This
last expression is an example of a power series.  In calculus it is
important to think about whether a power series converges in order to
determine whether or not it represents a function.  In a nice twist of
language, even though we use the phrase generating function as the name
of a power series in combinatorics, we don't require the power series to
actually represent a function in the usual sense, and so we don't have to
worry about convergence.\footnote{In the evolution of our current
mathematical terminology, the word function evolved through several
meanings, starting with  very imprecise meanings and ending with our
current rather precise meaning.  The terminology ``generating function''
may be thought of as an example of one of the earlier usages of the term
function.}  Instead we think of a power series as a convenient way of
representing the terms of a sequence of numbers of interest to us.  The
only justification for saying that such a representation is convenient is
because of the way algebraic properties of power series capture some of
the important properties of some sequences that are of combinatorial
importance.  The remainder of this chapter
is devoted to giving examples of how the algebra of power series reflects
combinatorial ideas.  

Because we choose to think of power series  as strings of symbols that we
manipulate by using the ordinary rules of algebra and we choose to ignore
issues of convergence, we have to avoid manipulating power series in a way that
would require us to add infinitely many real numbers.  For example, we
cannot make the substitution of $y+1$ for
$x$ in the power series $\sum_{i=0}^\infty x^i$, because in order to
interpret $\sum_{i=0}^\infty (y+1)^i$ as a power series we would have to
apply the binomial theorem to each of the $(y+1)^i$ terms, and then
collect like terms, giving us infinitely many ones added together as the
coefficient of
$y^0$, and in fact infinitely many numbers added together for the
coefficient of any~$y^i$.  (On the other hand, it would be fine to
substitute $y+y^2$ for $x$.  Can you see why?)

\subsection{Power series}  For now, most of our uses of power series will
involve just simple algebra.  Since we use power series in a different
way in combinatorics than we do in calculus, we should review a bit of
the algebra of power series. 
\bp
\iteme In the polynomial $(a_0 +a_1x+a_2x^2)(b_0+b_1x+b_2x^2+b_3x^3)$,
what is the coefficient of $x^2$?  What is the coefficient of $x^4$?
\label{coeffinproduct} 
\solution{$a_0b^2+a_1b_1+a_2b_0$ is the coefficient of $x^2$.  $a_1b_3+a_2b_2$
is the coefficient of $x^4$.}

\iteme In Problem \ref{coeffinproduct} why is there a $b_0$ and a $b_1$ in
your expression for the coefficient of $x^2$ but there is not a $b_0$
or a $b_1$ in your  expression for the coefficient of  $x^4$?  What is the
coefficient of $x^4$ in $$(a_0+a_1x+a_2x^2+a_3x^3+a_4x^4)(b_0+b_1x+b_2x^2
+b_3x^3+b_4x^4)?$$
Express this coefficient in the form
$$\sum_{i=0}^4 \mbox{ something},$$
where the something is an expression you need to figure out.  Now suppose
that $a_3=0$, $a_4=0$ and $b_4=0$.  To what is your expression equal
after you substitute these values?  In particular, what does this have to
do with Problem \ref{coeffinproduct}?\label{coeffinproduct1}
\solution{There is a $b_0$ because it can be paired with an $a_2$ to give the
term $a_2b_0x^4$.  Similarly there is a $b_1$ because it can be paired with
$a_1$ for the same purpose.  However there is no $a_i$ that we can pair with
$b_0$ to get a coefficient of $x^4$ and no $a_i$ that we can pair with $b_3$
to get a coefficient of $x^4$.  

The coefficient of $x^4$ in
$$(a_0+a_1x+a_2x^2+a_3x^3+a_4x^4)(b_0+b_1x+b_2x^2
+b_3x^3+b_4x^4)$$
is $\sum_{i=0}^4 a_ib_{4-i}$.  If we substitute $a_3=0$, $a_4=0$,
and
$b_4 =0$, we get the coefficient of $x^4$ in $(a_0
+a_1x+a_2x^2)(b_0+b_1x+b_2x^2+b_3x^3)$.  This exemplifies the idea that we can
get a uniform formula for the coefficient of $x^i$ (namely, sum all
$a_jb_{i-j}$ from $j=0$ to $i$) in a product of two polynomials if we are
willing to say that the coefficient of a power of
$x$ that does not appear in a polynomial is 0.}


\iteme The point of the Problems \ref{coeffinproduct} and
\ref{coeffinproduct1} is that so long as we are willing to assume $a_i=0$
for $i>n$ and $b_j =0$
 for $j>m$, then there is a very nice formula for the coefficient of
$x^k$ in the product
$$\left(\sum_{i=0}^n a_ix^i\right)\left(\sum_{j=0}^m b_jx^j\right).$$ 
Write down this formula explicitly.
\solution{$\sum_{i=0}^k a_ib_{k-i}$.}

\iteme Assuming that the rules you use to do arithmetic with polynomials
apply to power series, write down a formula for the coefficient of $x^k$
in the product \label{coeffinpowerseries}
$$\left(\sum_{i=0}^\infty a_ix^i\right)\left(\sum_{j=0}^\infty
b_jx^j\right).$$
\solution{$\sum_{i=0}^k a_ib_{k-i}$.}
\ep

We use the expression you obtained in Problem \ref{coeffinpowerseries} to
{\em define} the product of power series.  That is, we define the product
$$\left(\sum_{i=0}^\infty a_ix^i\right)\left(\sum_{j=0}^\infty
b_jx^j\right)$$ to be the power series $\sum_{k=0}^\infty c_k x^k$, where
$c_k$ is the expression you found in Problem \ref{coeffinpowerseries}.
Since you derived this expression by using the usual rules of algebra for
polynomials, it should not be surprising that the product of power series
satisfies these rules.\footnote{Technically we should explicitly state
these rules and prove that they are all valid for power series
multiplication, but it seems like overkill at this point to do so!}
\subsection{Product principle for generating functions}



 Each time that
we converted a picture function to a generating function by substituting
$x$ or some power of $x$ for each picture, the coefficient of $x$ had a
meaning that was significant to us.  For example, with the
picture enumerator for selecting between zero and three each of apples,
pears, and bananas, when we substituted $x$ for each of our pictures, the
exponent $i$ in the power $x^i$ is the number of pieces of fruit in the
fruit selection that led us to $x^i$.  After we simplify our product by
collecting together all like powers of $x$, the coefficient of $x^i$ is
the number of fruit selections that use $i$ pieces of fruit.  In the same
way, if we substitute $x^c$ for a picture, where $c$ is the number of
calories in that particular kind of fruit, then the $i$ in an $x^i$ term
in our generating function stands for the number of calories in a fruit
selection that gave rise to $x^i$, and the coefficient of $x^i$ in our
generating function is the number of fruit selections with $i$ calories. 
The product principle of picture enumerators translates directly into a
product principle for generating functions.

\bp 
\iteme Suppose that we have two sets $S_1$ and $S_2$.  Let $v_1$ ($v$
stands for value) be a function from $S_1$ to the nonnegative integers
and let $v_2$ be a function from $S_2$ to the nonnegative integers. 
Define a new function $v$ on the set $S_1 \times S_2$ by $v(x_1,x_2) =
v_1(x_1) +v_2(x_2)$.  Suppose further that $\sum_{i=0}^\infty a_ix^i$ is
the generating function for the number of elements $x_1$ of $S_1$ of value
$i$, that is with $v_1(x_1)=i$.  Suppose also
that
$\sum_{j=0}^\infty b_j x^j$ is the generating function for the number of
elements of $x_2$ of $S_2$ of value $j$, that is with $v_2(x_2) = j$. 
Prove that the coefficient of $x^k$ in  
$$\left(\sum_{i=0}^\infty a_ix^i\right)\left(\sum_{j=0}^\infty
b_jx^j\right)$$ is the number of ordered pairs $(x_1,x_2)$ in $S_1\times
S_2$ with total value $k$, that is with $v_1(x_1) +v_2(x_2) =k$.  This is
called the {\bf product principle for generating functions}.\index{product
principle for generating functions}\index{generating function!product
principle for}\label{ProductPrincipleOGF}
\solution{The generating function for ordered pairs of total value $k$ will
have the number of ordered pairs of total value $k$ as the coefficient of
$x^k$.  But we get a total value $k$ by taking something of value $i$ in $S_1$
and something of value $k-i$ in $j$.  And since values cannot be negative, the
only $i$s available to us are the ones between $0$ and $k$.  By the product
principle for pairs, the number of ordered pairs $(x,y)$ with $v_1(x)=i$ and
$v_2(y)=k-i$ is $a_ib_{k-i}$.  To get the number of pairs of total value $k$,
we have to sum over all possible pairs $(i,k-i)$ of values, that is, we have
to take the sum $\sum_{i=0}^k a_ib_{k-i}$.  And this is the coefficient of
$x^k$ in the product $$\left(\sum_{i=0}^\infty a_ix^i\right)\left(\sum_{j=0}^\infty
b_jx^j\right).$$  This proves the product principle for generating functions.}

\iteme Let $i$ denote an integer between 1 and $n$. 
\begin{enumerate}\item
What is the generating function for the number of subsets of $\{i\}$ of
each possible size? (Notice that the only subsets of $\{i\}$ are
$\emptyset$ and $\{i\}$.)
\solution{$1+x$.}
\item Use the product principle for generating functions to prove the
binomial theorem. \end{enumerate}
\solution{The generating function for number of $n$-tuples of 0s and 1s with
$k$ ones is, by the product principle for generating functions (extended to
$n$ factors) $(1+x)^n$.  However the number of $n$-tuples of 0s and 1s with
$k$ ones is the number of $k$-element subsets of an $n$-element set. 
Therefore $(1+x)^n =\sum_{i=0}^n {n\choose i}x^i$.  To get the $(x+y)^n$ form
of the binomial theorem, note that $x+y = y(1+{x\over y})$, so that
\begin{eqnarray*}(x+y)^n &=& y^n\left({x\over y}+1\right)^n\\
&=&y^n\sum_{i=0}^n {n\choose i}{x^i\over y^i}\\
&=& \sum_{i=0}^n{n\choose i}x^iy^{n-i}.\end{eqnarray*}}


\ep


\subsection{The extended binomial theorem and multisets}
\bp \iteme Suppose once again that $i$ is an integer between 1 and $n$.
\begin{enumerate}
\item What is the generating function in which the coefficient
of
$x^k$ is the number of multisets of size $k$ chosen from $\{i\}$?  This
series is an example of what is called an {\em infinite geometric
series}.\index{geometric series}\index{series!geometric}
\solution{$1+x+x^2+\cdots+x^i+\cdots=\sum_{i=0}^\infty x^i$.}
\item Express the generating function in which the coefficient of $x^k$
is the number of multisets chosen from $[n]$ as a power of a power
series.  What does Problem \ref{multiset} (in which your answer could
be expressed as a binomial coefficient) tell you about what this
generating function equals?  
\solution{The generating function is $\left(\sum_{i=0}^\infty x^i\right)^n$. 
Problem \ref{multiset} tells us that this equals
$\sum_{i=0}^\infty{n+i-1\choose i}x^i$.} 
\end{enumerate}

 \itemm  What is the product $$(1-x)\sum_{k=0}^\infty x^k?$$
\solution{$$(1-x)\sum_{k=0}^n x^k=1-x+x-x^2+\cdots+x^{n-1}-x^n+x^n-x^{n+1} =
1-x^{n+1}.$$
$$(1-x)\sum_{k=0}^\infty x^k=\sum_{i=0}^\infty x^i-\sum_{i=0}^\infty
x^{i+1}=\sum_{i=0}^\infty x^i-\sum_{i=1}^\infty x^i = 1.$$}

\itemei Express the generating function for the number of multisets of
size
$k$ chosen from $[n]$ (where $n$ is fixed but $k$ can be any nonnegative
integer) as a 1 over something relatively simple.
\solution{Since $(1-x)\sum_{k=0}^\infty x^k=1$, we have that
$$\sum_{k=0}^\infty x^k={1\over 1-x}.$$  Therefore $\left(\sum_{k=0}^\infty
x^k\right)^n= {1\over(1-x)^n}$ is the generating function for multisets of
size $k$ chosen from an $n$ element set.}

\iteme Find a formula for $(1+x)^{-n}$ as a power series whose coefficients
involve binomial coefficients.   What does this formula tell you about
how we should define
$-n\choose k$ when $n$ is positive? \label{negnchoosek} 
\solution{$$\hspace*{-.25 in}(1+x)^{-n}=(1-(-x))^{-n}=\sum_{i=0}^\infty
{n+i-1\choose i}(-x)^i=\sum_{i=0}^\infty (-1)^i{n+i-1\choose i}x^i.$$  We want
the coefficient of
$x^k$ in
$(1+x)^{-n}$ to be $-n\choose k$, so we want ${-n \choose k}=
(-1)^k{n+k-1\choose k}$.}

\itemei  If you define $-n \choose k$ in the way you described in Problem
\ref{negnchoosek}, you can write down a version of the binomial theorem
for $(x+y)^n$ that is valid for both nonnegative and negative values of
$n$.  Do so.  This is called the {\em extended binomial
theorem}\index{binomial theorem! extended}\index{extended binomial
theorem}.
\solution{$(x+y)^n=\sum_{i=0}^\infty{n\choose i}x^i$.  The proof consists of
writing $(x+y)=y({x\over y}+1)$ and applying the Problem \ref{negnchoosek} when
$n$ is negative. When $n$ is positive, we recall that $n\choose k$ is zero
when $k>n$, so replacing the upper limit of $n$ in the standard version of the
binomial theorem by $\infty$ doesn't change the value of the sum.}

\itemei Write down the generating function for the number of ways to
distribute identical pieces of candy to three children so that no
child gets more than 4 pieces.  Write this generating function as a
quotient of polynomials.  Using both the extended binomial theorem and
the original binomial theorem, find out in how many ways we can pass out
exactly ten pieces.  Use one of our earlier counting techniques to verify
your answer.
\solution{$(1+x+x^2+x^3+x^4)^3$.  We can write
\begin{eqnarray*} &&(1+x+x^2+x^3+x^4)^3 \>=\>\left({1-x^5\over 1-x}\right)^3\\
&=&(1-x^5)^3(1-x)^{-3}\\
&=&(1-3x^5+3x^{10}-x^{15})\sum_{i=0}^\infty {3+i-1\choose i}x^i\\
&=&(1-3x^5+3x^{10}-x^{15})\sum_{i=0}^\infty {2+i\choose i}x^i
%&=&\sum_{i=0}^\infty \left({2+i\choose i} -3{2+i-5\choose i-5}
%+3{2-i-10\choose i-10} -{2+i-15\choose i-15}\right)x^i\\
%&=&\sum_{i=0}^\infty \left({2+i\choose i} -3{i-3\choose i-5} + 3{i-8\choose
%i-10} - {i-13\choose i-15}\right)x^i
\end{eqnarray*}
The coefficient of $x^{10}$ is the number of ways to pass out ten pieces of
candy, and is ${12\choose 10}-3{7\choose 5} +3{2\choose 0}$. 
 Thus the number of ways to pass out ten pieces of candy is
$66-3\cdot21+3=6$.)

A second solution technique that is appropriate is inclusion-exclusion
counting.  We let property $i$ be that child $i$ gets more than four pieces of
fruit.  The number of ways to pass out $n$ pieces of fruit with a set $J$ of
size
$j$ properties is ${n-5j+3-1\choose n-5j}={n-5j+2\choose 2}$.  By
Equation \ref{incexempty} we have 
$N_{\mbox{e}}(\emptyset) = \sum_{j=0}^3
(-1)^j{3\choose j}{n-5j+2\choose 2}$.  Using $n=10$ gives us
${12\choose2}-3{7\choose2} +3{2\choose2} =6.$  (You can either note that
${-3\choose 2}$ is 0 or note that with 10 pieces of candy, there is no way to
pass out five pieces to each of three children.  In this second case we would
revise our statement about the number of ways to pass out $n$ pieces of fruit
to three children with the properties in $J$ to note that it is zero if
$5j>n$, and otherwise is $n-5j+2\choose2$.)}

\iteme  What is the generating function for the number of multisets chosen
from an $n$-element set so that each element appears at least $j$ times
and less than $m$ times.  Write this generating function as a quotient of
polynomials, then as a product of a polynomial and a power series. 
\solution{
\begin{eqnarray*}(x^j+x^{j+1}+\cdots+x^{m-1})&=&x^j
\left(\sum_{i=0}^{m-j-1}x^i\right)^n\\&=&
\left(x^j{1-x^{m-j}\over 1-x}\right)^n\\
&=&\left({x^j-x^m\over1-x}\right)^n\\
&=& (x^j-x^m)^n\sum_{i=0}^\infty {n+i-1\choose i}x^i.
\end{eqnarray*}}
\ep

\subsection{Generating functions for integer partitions}
\bp
\iteme If we have five identical pennies, five identical nickels, five
identical dimes, and five identical quarters, give the picture enumerator
for the combinations of coins we can form and convert it to a generating
function for the number of ways to make $k$ cents with the coins we
have.  Do the same thing assuming we have an unlimited supply of pennies,
nickels, dimes, and quarters.\label{change-making}
\solution{$(1+P+P^2+P^3+P^4+P^5)(1+N+N^2+N^3+N^4+N^5)(1+D+D^2+D^3+D^4+D^5)
(1+Q+Q^2+Q^3+Q^4+Q^5)$.  Substituting $x$ for $P$, $x^5$ for $N$, $x^{10}$ for
$D$ and $x^{25}$ for $Q$ gives us
$$\sum_{i=0}^5x^i\sum_{i=0}^5x^{5i}\sum_{i=0}^5 x^{10i} \sum_{i=0}^5
x^{25i}={1-x^6\over1-x}\cdot{1-x^{30}\over 1-x^5}\cdot{1-x^{60}\over
1-x^{10}}\cdot{1-x^{150}\over 1-x^{25}}.$$ Although we could write this as a
polynomial times a product of four power series, doing so would not
significantly increase our understanding, though it would let us make some
painful computations of the number of ways to make a certain number of
cents. If we actually wanted such numbers we would be better off asking a
computer algebra package to expand the product of the polynomials on the
left.  With unlimited supplies the generating function becomes
$${1\over1-x}\cdot{1\over 1-x^5}\cdot{1\over 1-x^{10}}\cdot{1\over 1-x^{25}}.$$ 
Again, we could write this as a product of power series, and that would allow us
to painfully compute the number of ways to create a certain number of cents.  If
we actually wanted to know the number of ways to make up 200 cents, say, it
would be more sensible to ask a computer algebra package to extract the
coefficient of $x^{200}$ in the product of the four quotients.}

\iteme  Recall that a partition of an integer $k$ is a multiset of numbers
that adds to $k$.  In Problem \ref{change-making} we found the
generating function for the number of partitions of an integer into parts
of size 1, 5, 10, and 25.   Give the generating function for the number
partitions of an integer into parts of size one through ten.  Give the
generating function for the number of partitions of an integer $k$ into
parts of size at most $m$. (Where $m$ is fixed but $k$ may vary.)  Notice this
is the generating function for partitions whose Young diagram fits into the space
between the line
$x=0$ and the line $x=m$ in a coordinate plane.  (We assume the boxes in the
Young diagram are one unit by one unit.)\label{largestpartatmostm}  When
working with generating functions for partitions, it is becoming standard
to use $q$ rather than $x$ as the variable in the generating function. 
Write your answers in this notation.\footnote{The reason for this change
in the notation relates to the subject of finite fields in abstract
algebra, where $q$ is the standard notation for the size of a finite
field.  While we will make no use of this connection, it will be easier
for you to read more advanced work if you get used to the different
notation.}
\solution{$\prod_{i=1}^{10}{1\over1-q^i}$, $\prod_{i=1}^m{1\over 1-q^i}$.}

\iteme In Problem \ref{largestpartatmostm} you gave the generating
function for the number of partitions of an integer into parts of size at
most
$m$.  Explain why this is also the generating function for partitions of
an integer into at most $m$ parts.  Notice that this is the generating
function for the number of partitions whose Young diagram fits into the
space between the line $y=0$ and the line $y=m$.\label{atmostmparts}
\solution{Conjugation is a bijection between partitions with largest part at
most
$m$ and partitions with at most $m$ parts. Thus the coefficient of $q^i$ (the
number of partitions of $i$ into parts of size at most $m$) in the generating
function for the number of partitions of integers into parts of size at most
$m$ will be the coefficient of
$q^i$ (the number of partitions of $i$ with at most $m$ parts) in the
generating function for the number of partitions of integers into parts of
size at most
$m$.  Thus the two generating functions are the same.}



\iteme Give the
generating function for the number of partitions of an integer into parts
of any size. Don't forget to use $q$ rather than $x$ as your variable.
This generating function involves an infinite product.  Describe the kinds
of terms you actually multiply and add together to
get the last generating function.  Rewrite any
power series that appear in your product as
quotients of polynomials or as integers divided
by polynomials.\label{genfunpartitions}
\solution{We start with 
$$(1+q+q^2+\cdots)(1+q^2+q^4+\cdots)\cdots(1+q^i+q^{2i})\cdots,$$
which we can write more precisely as 
$$\prod_{i=1}^\infty \sum_{j=0}^\infty q^{ij}.$$  To get the coefficient of
$q^k$ in this product, we look at all ways of choosing one summand from each
of the infinite series and multiplying them together to get $q^k$, and add all
these products up.  That is, the coefficient of $q^k$ is the number of ways of
making these choices of one summand from each series so that the product of
our choices is $q^k$.  We can rewrite the infinite product as
$\displaystyle\prod_{i=0}^\infty{1\over 1-q^i}$.}

\itemi In Problem \ref{genfunpartitions}, we multiplied together
infinitely many power series.  Here are two notations for infinite
products that look rather similar:
$$\prod_{i=1}^\infty 1 + x + x^2 +\cdots+ x^i\quad\mbox{and}\quad
\prod_{i=1}^\infty 1 +x^i +x^{2i} +\cdots + x^{i^2}. $$
 However, one makes sense and one doesn't.  Figure
out which one makes sense and explain why it makes sense and the other one
doesn't.  If we want a product of the form
$$\prod_{i=1}^\infty 1 +p_i(x),$$
where each $p_i(x)$ is a nonzero polynomial in $x$
to make sense, describe a relatively simple assumption
about the polynomials $p_i(x)$ that will make the product make
sense.  If we assumed the terms $p_i(x)$ were nonzero power series, is
there a relatively simple assumption we could make about them in order to
make the product make sense?  (Describe such a condition or explain why
you think there couldn't be one.)
\solution{$\prod_{i=1}^\infty 1 +x^i +x^{2i} +\cdots + x^{i^2}$ makes sense
because when we look for ways of choosing one summand from each factor so that
the summands multiply together to give us $x^k$, we will find only finitely
many ways of making those choices, so the coefficient of $x^k$ can be taken to
be the number of such choices.  On the other hand, in the expression
$\prod_{i=1}^\infty 1 + x + x^2 +\cdots+ x^i$, there are infinitely many ways
to choose $x$ from one term and $1$ from all the rest of the terms so that the
product of these summands is $x$.  Thus we can't even specify what the
coefficient of $x$ is in the product.  On the basis of this analysis, we see
that for $\prod_{i=1}^\infty 1 +p_i(x)$ to make sense, we need to assume that
for each possible positive integer $n$, there are only a finite number of
polynomials $p_i$ whose lowest degree term has degree less than
or equal to
$n$.  In that way, for each positive integer, there will be only finitely many
ways to chose a summand from each factor so that the product of the summands
is a multiple of $x^k$. The same assumption works when the $p_i$ are power
series, for the same reason.} 

\iteme What is the generating function (using $q$ for the variable) for
the number of partitions of an integer in which each part is even?
\solution{$(1+q^2+q^4+\cdots)(1+q^4+q^8+\cdots)(1+q^6+q^{12}+\cdots)\cdots$,
which can be written more precisely as $\displaystyle\prod_{i=1}^\infty
\sum_{j=0}^\infty q^{2ij}$.}

\iteme  What is the generating function (using $q$ as the variable) for
the number of partitions of an integer into distinct parts, that is, in
which each part is used at most once?
\solution{$\displaystyle (1+q)(1+q^2)(1+q^3)\cdots=
\prod_{i=1}^\infty(1+q^i)$.}


\iteme  Use generating functions to explain why the number of partitions
of an integer in which each part is used an even number of times equals
the generating function for the number of partitions of an integer in
which each part is even.
\solution{In the generating function $\displaystyle\prod_{i=1}^\infty
\sum_{j=0}^\infty q^{2ij}$, we may interpret the $2ij$ in $q^{2ji}$ the value
of using $2i$ as a part $j$ times or as the value of using $i$ as a part $2j$
times.  Therefore this is the generating function both for the number of
partitions of integers into parts that are even and the number of partitions
into parts that are used an even number of times. Therefore the number of
partitions of $n$ in which each part is even equals the number of partitions
of $n$ in which each part is used an even number of times.}

\itemei Use the fact that $${1-q^{2i}\over 1-q^i}= 1+q^i$$ and the
generating function for the number of partitions of an integer into
distinct parts to show how the number of partitions of an integer $k$
into distinct parts is related to the number of partitions of an integer
$k$ into odd parts.
\solution{$\displaystyle\prod_{i=1}^\infty 1+q^i=\prod_{i=1}^n {1-q^{2i} \over
1-q^i}={\prod_{i=1}^\infty1-q^{2i}\over\prod_{j=1}^\infty 1-q^j}
=\prod_{i=j}^\infty {1\over1-q^{2j-1}}$, because all the terms in the numerator
cancel with alternate terms in the denominator leaving only terms with odd
powers of $q$. But \begin{eqnarray*}&&\prod_{j=1}^\infty{1\over
1-q^{2j-1}}\>=\>\prod_{j=1}^\infty
\sum_{i=0}^\infty
(q^{2j-1})^i\\&=&(1+q+q^{2\cdot1}+\cdots)(1+q^3+q^{2\cdot3}+\cdots)(1
+q^5+q^{2\cdot5}+\cdots)\cdots,
\end{eqnarray*}
which is the generating function for partitions of integers into parts that
are odd numbers.}

\item Write down the generating function for the number of ways to
partition an integer into parts of size no more than $m$, each used an
odd number of times.  Write down the generating function for the number
of partitions of an integer into parts of size no more than $m$, each
used an even number of times.  Use these two generating functions to get
a relationship between the two sequences for which you wrote down the
generating functions. 
\solution{$$\displaystyle\prod_{i=1}^m q^i+q^{3i}+q^{5i}= q^{1+2+\cdots
m}\prod_{i=1}^m 1+q^{2i}+q^{4i}+\cdots=q^{n+1\choose 2}\prod_{i=1}^m{1\over
1-q^2}$$
is the generating function for the number of ways to partition an integer into
parts of size at most $m$, each used an odd number of times. $$\displaystyle
\prod_{i=1}^m 1 +q^{2i}+q^{4i}+q^{6i}+\cdots=\prod_{i=1}^m {1\over1-q^{2i}}$$
is the generating function for the number of partitions of an integer into
parts of size no more than $m$, each used an even number of times. Therefore
the number of partitions of $k$ into parts of size no more than $m$, each used
an even number of times is the number of partitions of $k+{m+1\choose 2}$,
into parts of size no more than $m$, each used an odd number of times.}



\itemi In Problem \ref{largestpartatmostm} and Problem \ref{atmostmparts}
you gave the generating functions for, respectively, the number of
partitions of $k$ into parts the largest of which is at most $m$ and for
the number of partitions of $k$ into at most $m$ parts.  In this problem
we will give the generating function for the number of partitions of $k$
into at most $n$ parts, the largest of which is at most $m$.  That is we
will analyze $\sum_{i=0}^\infty a_kq^k$ where $a_k$ is the number of
partitions of $k$ into at most $n$ parts, the largest of which is at most
$m$.  Geometrically, it is the generating function for partitions whose
Young diagram fits into an $m$ by $n$ rectangle, as in Problem
\ref{rectanglecomplement}.  This generating function has significant
analogs to the binomial coefficient
$m+n\choose n$, and so it is denoted by $\qchoose{m+n}{n}$.  It is called
a {\em $q$-binomial coefficient}.\index{$q$-binomial
coefficient}\index{binomial coefficient!$q$-binomial}
\begin{enumerate}
\item Compute $\qchoose{4}{2}=\qchoose{2+2}{2}$.
\solution{A partition with up to two parts of size up to two can have no
parts, one part of size 1, which makes it a partition of 1, one part of
size 2 which makes it a partition of 2, two parts of size 1, which makes
it a partition of 2, a part of size 2 and a part of size 1, which makes
it a partition of 3, or two parts of size 2, which makes it a partition
of 4.  Thus $\qchoose{4}{2}=1+q+2q^2+q^3+q^4$.}
\item Find  explicit formulas for $\qchoose{n}{1}$ and
$\qchoose{n}{n-1}$.
\solution{Both are $1+q+q^2+\cdots+q^{n-1} = {1-q^{n-1}\over 1-q}$,
because they are the generating function for the number of partitions
whose Young diagram fits into a rectangle $n-1$ units wide and 1 unit
deep or into a rectangle 1 unit wide and $n-1$ units deep respectively.}
\item How are $\qchoose{m+n}{n}$ and $\qchoose{m+n}{n}$ related?  Prove
it.  (Note this is the same as asking how $\qchoose{r}{s}$ and
$\qchoose{r}{r-s}$ are related.)
\solution{By conjugation, $\qchoose{m+n}{n}=\qchoose{m+n}{m}$.}
\item So far the analogy to $m+n\choose n$ is rather thin!  If we had a
recurrence like the Pascal recurrence, that would demonstrate a real
analogy.  Is $\qchoose{m+n}{n}=
\qchoose{m+n-1}{n-1}+\qchoose{m+n-1}{n}$?
\solution{No. $\qchoose{4}{2}=1+q+2q^2+q^3+q^4$, but
$\qchoose{3}{1}=1+q+q^2$ and
$\qchoose{3}{2}=1+q+q^2$.}
\item Recall the two operations we studied in Problem
\ref{numberpartitionrecurrence}.  
\begin{enumerate}
\item The largest part of a partition counted
by $\qchoose{m+n}{n}$ is either $m$ or is less than or equal to $m-1$. 
In the second case, the partition fits into a rectangle that is at most 
$m-1$ units wide and at most $n$ units deep.  What is the generating
function for partitions of this type?  In the first case, what kind of
rectangle does the partition we get by removing the largest part sit in? 
What is the generating function for partitions that sit in this kind of
rectangle?  What is the generating function for partitions that sit in
this kind of rectangle after we remove a largest part of size $m$?  What
recurrence relation does this give you?
\solution{The generating function for partitions that arise in the second
case is $\qchoose{m+n-1}{m-1}=\qchoose{m+n-1}{n}$.  In the first case,
after we delete the largest part, the Young diagram sits in a rectangle
of width $m$ and depth $n-1$.  The generating function for partitions that
arise in the first case {\em after} we have deleted a part of size
$m$ is $\qchoose{m+n-1}{n-1}$.  Thus the generating function for
partitions that arise in the first case ({\em before} we delete the part
of size $m$) is $q^m\qchoose{m+n-1}{n-1}$.  Thus by the sum principle the
generating function for all partitions that fit into a rectangle of width
$m$ and depth $n$ is given by
$$\qchoose{m+n}{n}= q^m\qchoose{m+n-1}{n-1}+\qchoose{m+n-1}{n}.$$
If you don't use the symmetry $\qchoose{m+n-1}{m-1}=\qchoose{m+n-1}{n}$,
you get
$$\qchoose{m+n}{n}= q^m\qchoose{m+n-1}{n-1}+\qchoose{m+n-1}{m-1}.$$ 
While this doesn't {quite} look like the Pascal recurrence, it is still a
correct answer.}
\item What recurrence do you get from the other operation we studied in
Problem \ref{numberpartitionrecurrence}?
\solution{
We either have exactly $k$ parts or fewer than $k$ parts.  In
the first case, removing one from each part gives us a partition whose
Young diagram fits into a $m-1$ by $n$ box, while in the second case
doing nothing gives us a partition that fits into a $m$ by $n-1$ box.  In
the first case the partition we get partitions $k-n$, and in the second
case it still partitions $k$.  Thus we get
$$\qchoose{m+n}{n}= q^n\qchoose{m+n-1}{n}+\qchoose{m+n-1}{n-1}.$$}
\item It is quite likely that the two recurrences you got are different. 
One would expect that they might give different values for
$\qchoose{m+n}{n}$.  Can you resolve this potential conflict?
\solution{Yes, by conjugation.}
\end{enumerate}
\item Define $[n]_q$ to be $1+q+\cdots+q^{n-1}$ for $n>0$ and $[0]_q =1$. 
We read this simply as
$n$-sub-$q$.  Define $[n]!_q$ to be $[n]_q[n-1]_q\cdots
[3]_q[2]_q[1]_q$.  We read this as $n$ cue-torial, and refer to it as a
$q$-ary factorial.\index{factorial!$q$-ary}\index{$q$-ary factorial}  Show
that\label{qtorialformula}
$$\qchoose{m+n}{n} = {[m+n]!_q \over [m]!_q[n]!_q}.$$
\solution{
Note that $\qchoose{n}{0}= \qchoose{n}{n} = 1$, because only
the partition with no parts sits in a rectangle of width or depth 0. But 
${[m+0]!_q \over [m]!_q[0]!_q} =1$ and ${[0+n]!_q \over [0]!_q[n]!_q}
=1$.  Thus the formula holds when $m=0$ or $n=0$.  But 
\begin{eqnarray*} 
&&q^m{[m+n-1]!_q\over [n-1]!_q[m]!_q}+{[m+n-1]!_q\over [n]!_q[m-1]!_q}\\
&=&\relax[m+n-1]!_q\left({q^m\over[n-1]!_q[m]!_q}
+{1\over[n]!_q[m-1]!_q}\right)\\
&&\relax[m+n-1]!_q\left({[n]_qq^m\over [n]!_q[m]_q}+{[m]_q\over
[n]!_q[m]!_q}\right)\\
&=&\relax{[m+n-1]!_q\over [n]!_q[m]!_q}\left((1+q+\cdots+q^{n-1})q^m +
1+q+\cdots+q^{m-1}\right)\\
&=&\relax{[m+n-1]!_q[m+n]_q\over [n]!_q[m]!_q}\\
&=&\qchoose{m+n}{n}.
\end{eqnarray*}
Thus $[m+n]!_q \over [m]!_q[n]!_q$ satisfies our recurrence and so by the
principle of mathematical induction, $\qchoose{m+n}{n} = {[m+n]!_q \over
[m]!_q[n]!_q}.$
}
\item Now think of $q$ as a variable that we will let approach $1$. Find
an explicit formula for 
\begin{enumerate}
\item $\displaystyle\lim_{q\rightarrow 1} [n]_q$.
\solution{$n$}
\item  $\displaystyle\lim_{q\rightarrow 1} [n]!_q$.
\solution{$n!$}
\item  $\displaystyle\lim_{q\rightarrow 1}
\qchoose{m+n}{n}$.\label{partiii}
\solution{
$m+n\choose n$}
\end{enumerate}  Why is the limit in Part iii %\ref{partiii} 
equal to the
number of partitions (of any number) with at most $n$ parts all of size
most
$m$?  Can you explain bijectively why this quantity equals the formula
you got?
\solution{
Since the generating function is a finite sum (we are talking
about partitions whose Young diagram into a finite rectangle), the limit
is obtained by setting $q=1$, and this sums the number of partitions of
each possible number $k$ that have at most $n$ parts all of size at most
$m$.  We want a bijection between such partitions and the  $n$
element subsets of an $m+n$ element set.  Recall that there is a
bijection between subsets of an $n$ element set and lattice paths from
$(0,0)$ to $m,n$ in a coordinate plane.  If we draw our rectangle of
width $m$ and depth $n$ with its lower left corner at $(0,0)$, then each
Young diagram gives us such a lattice path and each such lattice path
gives us a Young diagram.}
\itemitemh What happens to $\qchoose{m+n}{n}$ if we let $q$
approach -1?
\solution{$\qchoose{m+n}{n}$ is the generating function for the
number of partitions whose Young diagram fits into an $m$ by $n$
rectangle. That is, 
$$\qchoose{m+n}{n}=\sum_{k=0}^\infty a_kq^k,$$
where $a_k$ is the number of partitions of $k$ whose Young diagram fits into an
$m$ by $n$ rectangle.  In particular $a_k=0$ if $k>mn$, because the Young
diagram of a partition of a number larger than $mn$ certainly cannot fit into
an $m$ by $n$ rectangle.  Thus
$$\qchoose{m+n}{n}=\sum_{k=0}^{mn} a_kq^k.$$
Furthermore, $a_k=a_{mn-k}$, because the complement in an $m$ by $n$ rectangle
of a partition of $k$ is a partition of $mn-k$, and complementation in an $m$
by $n$ rectangle is a bijection between partitions of $k$ that fit into the
rectangle and partitions of $mn-k$ that fit into the rectangle.  Thus
$\qchoose{m+n}{n}$ is a polynomial of degree $mn$ in which the coefficient of
$q_k$ equals the coefficient of $q^{mn-k}$.  For this reason we can compute the
limit as $q$ approaches $-1$ simply by substituting $-1$ for $q$ in the
polynomial; the only trouble being that the formula we know for
$\qchoose{m+n}{n}$ is quotient of polynomials that has zero in both the
numerator and denominator when we substitute in $q=-1$.  Nonetheless, when we
substitute $-1$ for $q$ we get the alternating sum $\sum_{i=0}^{mn} (-1)^ia_i$. 
Thus if $a_i$ and $a_{mn-i}$ have opposite sign, they will cancel out.  If $mn$
is even, $i$ is even exactly when $mn-i$ is even, and so $a_i$ and $a_{mn-i}$
have the same sign.  However when $mn$ is odd, $a_i$ and $a_{mn-i}$ have
opposite signs in the sum $\sum_{i=0}^{mn} (-1)^ia_i$, and so the sum is zero. 
Thus the polynomial
$\qchoose{m+n}{n}$ is zero at $q=-1$ whenever $mn$ is odd.

Our experience with binomial coefficients might lead us to hope that the
alternating sum of the coefficients $a_k$ will always be zero.  However, we
computed that
$\qchoose{4}{2}=1+q+2q^2+q^3+q^4$, so 
$$\lim_{q\rightarrow -1}\qchoose{4}{2}=1-1+2-1+1=1.$$ 
\iffalse Drawing a few Young
diagrams shows us that
$$\qchoose{5}{2}=\qchoose{5}{3}=1+1q+2q^2+2q^3+2q^4+1q^5+1q^6$$ and
$$\qchoose{6}{2}=\qchoose{6}{4}=1+q+2q^2+2q^3+3q^4+2q^5+2q^6+1q^7+1q^8.$$
Thus $\lim_{q\rightarrow-1}\qchoose{5}{2}=2$ and
$\lim_{q\rightarrow-1}\qchoose{6}{2}=3$.  It seems unlikely that we will be
able to work out enough cases to make a good guess by drawing Young diagrams. 
Part \ref{qtorialformula} tells us that
\begin{eqnarray*}&&\qchoose{m+n}{n} = {[m+n]!_q \over
[m]!_q[n]!_q}\\
&=&{[m+n]_q[m+n-1]_q\cdots[m+1]_q\over [n]_q\cdot[n-1]_q\cdots [2]_q[1]_q}\\
&=&{(1+q+\cdots+q^{m+n-1})\cdots(1+q+\cdots+q^{m})\over (1+q+\cdots
q^{n-1})\cdots(1+q)(1)}\\
&=&{(1+q+\cdots+q^{m+n-1})\cdots(1+q+\cdots+q^{m})\over (1+q+\cdots
q^{n-1})\cdots(1+q)(1)}\cdot{(1-q)^n\over (1-q)^n}\\
&=&{(1-q^{m+n})\cdots(1-q^{m+1})\over (1-
q^{n})\cdots(1-q^2)(1-q)}.
\end{eqnarray*}

Since $\qchoose{m+n}{n}$ is a polynomial in $q$, the factors in the denominator
should divide into the factors of the numerator to give us that polynomial. 
For example,
$$\qchoose{7}{2}={(1-q^7)(1-q^6)\over(1-q^2)(1-q)} =(1+q+q^2+q^3+q^4+q^4+q^6)
(1+q^2+q^4)$$
so $\neg1choose{7}{2}=1\cdot3=3$.  Unfortunately this breaks the pattern we had
with $\neg1choose{4}{2}=1$, $\neg1choose{5}{2}=2$, $\neg1choose{6}{2}=3$. 
However, this method of calculation is sufficiently straightforward to allow us
to compute more complicated values of $\neg1choose{m+n}{n}$.  It is also
reasonable for someone moderately comfortable with a symbolic algebra program
to implement on a computer.   Either these computations or the recurrence gives
us the following table.\fi
We could compute some more values of the limit  by going back to the
definition in this way, but it seems unlikely that we could get enough data to
make a good conjecture.  However we have the recurrence, and setting
$q=-1$ in the recurrence gives us the table

\begin{tabular}{c|c|c|c|c|c|c|c|c|c|c}
$m+n\backslash n$&0&1&2&3&4&5&6&7&8&9\\
\hline
0&1&0&0&0&0&0&0&0&0&0\\
1&1&1&0&0&0&0&0&0&0&0\\
2&1&0&1&0&0&0&0&0&0&0\\
3&1&1&1&1&0&0&0&0&0&0\\
4&1&0&2&0&1&0&0&0&0&0\\
5&1&1&2&2&1&1&0&0&0&0\\
6&1&0&3&0&3&0&1&0&0&0\\
7&1&1&3&3&3&3&1&1&0&0\\
8&1&0&4&0&6&0&4&0&1&0\\
9&1&1&4&4&6&6&4&4&1&1
\end{tabular}

From the table, it is clear that we get binomial coefficients interspersed with
0s in the even numbered rows of our table and repeated binomial coefficients in
the odd numbered rows of our table.  In particular when $m+n$ and $n$ are both
even, $\neg1choose{m+n}{n}={(m+n)/2\choose n/2}$, and if $m+n$ is even and $n$
is odd, $\neg1choose{m+n}{n}=0$.  In our table, at least, if $m+n$ is odd, it
appears that we get $\neg1choose{m+n}{n}={\lfloor(m+n)/2\rfloor\choose
\lfloor n/2\rfloor}$.  In fact, using the recurrence just as we used it to
construct the table, we can prove by induction that
$$\neg1choose{m+n}{n}=\left\{ 
\begin{array}{ll}0&\mbox{if $mn$ is odd}\\
{\lfloor(m+n)/2\rfloor\choose
\lfloor n/2\rfloor}&\mbox{otherwise.}
\end{array}\right.$$
(Note that $mn$ is odd if and only if $m+n$ is even and $n$ is odd.)  It would
be fascinating to know what we are counting here!}

\iffalse
 \solution $ \qchoose{n}{k} with q=-1 equals

              ( 0                               if k(n-k) is odd
              <
              ( (floor(n/2) choose floor(k/2))  if k(n-k) is even$

Note that the limit _always_ exists, since [n choose k]_q is actually
a _polynomial_ in q (via either of the two q-Pascal recurrences).  This
means that in the explicit product formula for [n choose k]_q, which
makes it look a priori like a rational function in q, there must be
cancellation of all the 1+q factors in the denominator by 1+q factors
in the numerator, so that L'Hopital's rule has to work (even if it is
messy).

Probably better than L'Hopital would be to just guess the above
formula empirically, by checking enough special cases via a computer
algebra package, and then using either q-Pascal recurrence (with q=-1)
to verify it by induction (making use of the usual Pascal recurrence
for binomial coefficients).  Note that the case where k(n-k) is odd
giving 0 is trivial by the symmetry of the coefficients in the polynomial
[n choose k]_q, since k(n-k) is the degree of this polynomial
(a symmetric polynomial of odd degree evaluated at -1 is always 0).

So in terms of stating a problem for the students, perhaps it's better
to

  (a) ask them to explain why you get 0 if k(n-k) is odd
      (hint about symmetry?)

  (b) then ask them to use a computer algebra package
      or pencil and paper to get data allowing them to guess
      the formula when k(n-k) is even,

  (c) then see if they can prove it
      (hint about the q-Pascal recurrences?).}\fi

\end{enumerate}
\ep

\section{Generating Functions and Recurrence Relations} Recall that a
recurrence relation for a sequence $a_n$ expresses $a_n$ in terms of
values $a_i$ for $i<n$.  For example, the equation $a_i=3a_{i-1} +2^i$ is
a first order linear constant coefficient recurrence.
\subsection{How generating functions are relevant}  Algebraic
manipulations with generating functions can sometimes reveal the
solutions to a recurrence relation. 

\bp
\iteme Suppose that $a_i=3a_{i-1} + 3^i$.  \label{substituteandsolve}
\begin{enumerate}
\item Multiply both sides by $x^i$ and
sum both the left hand side and right hand side from $i=1$ to infinity. 
In the left-hand side use the fact that 
 $$\sum_{i=1}^\infty a_ix^i = (\sum_{i=0}^\infty x^i) -a_0$$ and in the
right hand side, use the fact that 
 $$\sum_{i=1}^\infty a_{i-1}x^i = x\sum_{i=1}^\infty a_ix^{i-1}
=x\sum_{j=0}^\infty a_jx^j =x\sum_{i=0}^\infty a_ix^i$$ (where we substituted
$j$ for $i-1$ to see explicitly how to change the limits of summation, a
surprisingly useful trick) to rewrite the equation in terms of the power series
$\sum_{i=0}^\infty a_ix^i$.  Solve the resulting equation for the power series
$\sum_{i=0}^\infty a_ix^i$.
\solution{\begin{eqnarray*}\sum_{i=1}^\infty a_ix^i &=&3\sum_{i=1}^\infty
a_{i-1}x^i+\sum_{i=1}^i3^ix^i\\
\sum_{i=1}^\infty a_ix^i&=&3x\sum_{i=1}^\infty a_{i-1}x^{i-1}+
\sum_{i=0}^\infty 3^ix^i-3^0x^0\\
\sum_{i=0}^\infty a_ix^i -a_0&=&3x\sum_{i=0}^\infty a_{i}x^{i}+
{1\over 1-3x}-1\\
(1-3x)\sum_{i=0}^\infty a_ix^i &=&a_0+{1\over 1-3x} -1\\
\sum_{i=0}^\infty a_ix^i &=&{a_0-1\over 1-3x}+{1\over (1-3x)^2} 
\end{eqnarray*}}

\item Use the previous part to get a formula for $a_i$ in terms of $a_0$.
\solution{\begin{eqnarray*}\sum_{i=0}^\infty a_i x^i &=& {a_0-1\over 1-3x} +
{1\over(1-3x)^2} \\
&=&(a_0-1)\sum_{i=0}^\infty 3^ix^i +\sum_{i=0}^\infty {i+1\choose i}3^ix^i,
\end{eqnarray*}
which gives us $a_n=(a_0-1) 3^i + (i+1)3^i=(a_0+i)3^i$.}

\item Now suppose that $a_i=3a_{i-1} + 2^i$.  Repeat the previous two
steps for this recurrence relation.  (There is a way to do this part
using what you already know.  Later on we shall introduce yet another way
to deal with the kind of generating function that arises here.)
\solution{
\begin{eqnarray*}\sum_{i=1}^\infty a_ix^i&=&3\sum_{i=1}^\infty
a_{i-1}x^i +\sum_{i=1}^\infty 2^ix^i\\
\sum_{i=0}^\infty a_ix^i -a_0 &=&3x\sum_{i=1}^\infty a_{i-1}x^{i-1}
+\sum_{i=0}^\infty (2x)^i -1\\
\sum_{i=0}^\infty a_ix^i -a_0 &=&3x\sum_{i=0}^\infty a_ix^i +{1\over 1-2x} -1\\
(1-3x)\sum_{i=0}^\infty a_ix^i&=&a_0 +{1\over 1-2x}-1\\
\sum_{i=0}^\infty a_ix^i&=& {a_0-1\over 1-3x}+{1\over(1-2x)(1-3x)} \\
\sum_{i=0}^\infty a_ix^i&=&(a_0-1)\sum_{i=0}^\infty 3^ix^i +\sum_{i=0}^\infty
2^ix^i\sum_{j=0}^\infty 3^jx^j \\
\end{eqnarray*} 
But\ \  $\displaystyle\sum_{i=0}^\infty
2^ix^i\sum_{j=0}^\infty 3^jx^j=
\sum_{k=0}^\infty
\sum_{i=0}^k 2^i3^{k-i}x^k=
\sum_{k=0}^\infty3^kx^k
\sum_{i=0}^k {2^i\over3^i}=\\ \sum_{k=0}^\infty{1-\left({2\over
3}\right)^{k+1}\over 1-{2\over 3}}3^kx^k=\sum_{k=0}^\infty
(3^{k+1}-2^{k+1})x^k$. Substituting this into the equation for
$\sum_{i=0}^\infty a_ix^i$ gives us
$a_i =(a_0+2)3^i -2^{i+1}$.}
\end{enumerate}


\itemm Suppose we deposit \$5000 in a savings certificate that pays ten
percent interest and also participate in a program to add \$1000 to the
certificate at the end of each year (from the end of the first year on)
that follows (also subject to interest.)  Assuming we make the \$5000
deposit at the end of year 0, and letting $a_i$ be the amount of money in
the account at the end of year
$i$, write a recurrence for the amount of money the certificate is worth
at the end of year $n$.  Solve this recurrence.  How much money do we
have in the account (after our year-end deposit) at the end of ten years? 
At the end of 20 years?
\solution{$a_n=1.1a_{n-1}+1000$, and $a_0=5000$.  
\begin{eqnarray*}\sum_{i=1}^\infty
a_ix^i&=&1.1x\sum_{i=1}^\infty a_{i-1}x^{i-1} + 1000\sum_{i=1}^\infty x^i\\
\hspace*{-.25 in}\sum_{i=0}^\infty a_ix^i -a_0&=&
1.1x\sum_{i=0}^\infty a_ix^i +1000(\sum_{i=0}^\infty x^i-1)\\ 
\hspace*{-.5
in}(1-1.1x)\sum_{i=0}^\infty a_ix^i &=& a_0 +1000\sum_{i=10}^\infty x^i-1000\\
\sum_{i=0}^\infty a_i x^i&=& a_0\sum_{i=0}^\infty (1.1)^ix^i
+1000\sum_{i=0}^\infty (1.1)^ix^i\sum_{i=0}^\infty x^i -1000\\
\sum_{i=0}^\infty a_i x^i&=& a_0\sum_{i=0}^\infty (1.1)^ix^i
+1000\sum_{k=0}^\infty \left(\sum_{i=0}^k(1.1)^i\right)x^k
- 1000\sum_{i=0}^\infty (1.1)^ix^i
\end{eqnarray*} This gives us that $a_n= 4000(1.1)^n +1000{1-(1.1)^{n+1}\over
-.1}$.  This simplifies to $a_n= 4000(1.1)^n+10000(1.1)^{n+1}-10000 =
15,000(1.1)^n-10000$. Courtesy of Maple, after ten years we have \$28,906.14
and after 20 years we have \$90,912.50.}


\ep 

\subsection{Fibonacci Numbers}

The sequence of problems that follows describes a number of hypotheses we
might make about a fictional population of rabbits.  We use the example
of a rabbit population for historic reasons; our goal is a classical
sequence of numbers called Fibonacci numbers.  When
Fibonacci\footnote{Apparently Leanardo de Pisa was given the name Fibonacci
posthumously} introduced them, he did so with a fictional population of rabbits.

\subsection{Second order linear recurrence relations}
\bp
\iteme Suppose we start (at the end of month 0) with 10 pairs of baby rabbits,
and that after baby rabbits mature for one month they begin to reproduce, with
each pair producing two new pairs at the end of each month afterwards.  Suppose
further that over the time we observe the rabbits, none die. Let $a_n$ be the
number of rabbits we have at the end of month $n$. Show that
$a_n=a_{n-1} + 2a_{n-2}$.  This is an example of a {\em second
order}\index{recurrence!second order}\index{second order recurrence}
{\em linear}\index{recurrence!linear}\index{linear recurrence!second
order} recurrence with constant coefficients.\index{recurrence!constant
coefficient}  Using a method similar to that  of Problem
\ref{substituteandsolve}, show that 
 $$\sum_{i=0}^\infty a_ix^i = {10\over 1-x-2x^2}.$$
This gives us the generating function for the sequence $a_i$ giving the
population in month $i$; shortly we shall see a method for converting
this to a solution to the recurrence. \label{secondorderintroduction}
\solution{
\begin{eqnarray*}\sum_{n=2}^\infty a_nx^n&=&\sum_{n=2}^\infty a_{n-1}x^n +
2\sum_{n=2}^\infty a_{n-2}x^n\\
\sum_{n=0}^\infty a_nx^n -a_0-a_1x &=&x\sum_{n=2}^\infty a_{n-1}x^{n-1} +
2x^2\sum_{n=2}^\infty a_{n-2}x^{n-2}\\
\sum_{n=0}^\infty a_nx^n -a_0-a_1x&=&x\sum_{n=1}^\infty a_{n}x^{n} +
2x^2\sum_{n=0}^\infty a_{n}x^n\\
\sum_{n=0}^\infty a_nx^n -a_0-a_1x&=&x\left(\sum_{n=0}^\infty
a_{n}x^{n}-a_0\right) + 2x^2\sum_{n=0}^\infty a_{n}x^n\\
(1-x-2x^2)\sum_{n=0}^\infty a_nx^n&=&a_0+a_1x-a_0x\\
\sum_{n=0}^\infty a_nx^n&=&{a_0+a_1x-a_0x\over (1-x-2x^2)} 
\end{eqnarray*}
In this problem
$a_0=a_1=10$ because we start with ten pairs of baby rabbits, so they have to
mature for a month before they begin reproducing.  Thus
$\displaystyle\sum_{n=0}^\infty a_nx^n = {10\over (1-x-2x^2)}$}

\iteme In Fibonacci's original problem, each pair of mature rabbits
produces one new pair at the end of each month, but otherwise the
situation is the same as in Problem \ref{secondorderintroduction}. 
Assuming that we start with one pair of baby rabbits (at the end of month
0), find the generating function for the number of pairs of rabbits we
have at the end on $n$ months.\label{originalFibonacci}
\solution{Our recurrence becomes $a_n=a_{n-1}+a_{n-2}$, and following the
pattern of Problem \ref{secondorderintroduction} we get
\begin{eqnarray*}
\sum_{n=2}^\infty a_nx^n&=&\sum_{n=2}^\infty a_{n-1}x^n +
\sum_{n=2}^\infty a_{n-2}x^n\\
\sum_{n=0}^\infty a_nx^n-a_0-a_1x &=&x\left(\sum_{n=0}^\infty
a_{n}x^{n}-a_0\right) + x^2\sum_{n=0}^\infty a_{n}x^n\\
(1-x-x^2)\sum_{n=0}^\infty a_nx^n&=&a_0+a_1x-a_0x\\
\sum_{n=0}^\infty a_nx^n&=&{a_0+a_1x-a_0x\over (1-x-x^2)}
\end{eqnarray*} 
Since now $a_0=a_1=1$, we have $\displaystyle \sum_{n=0}^\infty a_nx^n=
{1\over 1-x-x^2}$.}

\itemi Find the generating function for the solutions to the recurrence
$$a_i=5a_{i-1}-6a_{i-2} + 2^i.$$\label{secondordernonhomo}
\solution{\begin{eqnarray*}\sum_{i=2}^\infty a_ix^i &=& \sum_{i=2}^\infty
5a_{i-1}x^i-\sum_{i=2}^\infty 6a_{i-2}x^i + \sum_{i=2}^\infty 2^ix^i\\
\hspace*{-.3in}\sum_{i=0}^\infty a_ix^i -a_0-a_1x &=&
5x\left(\sum_{i=0}^\infty
a_{i}x^i-a_0\right)-6x^2\sum_{i=0}^\infty a_{i}x^i + \sum_{n=0}^\infty
2^ix^i-1-2x\\
\hspace*{-.55in}(1-5x+6x^2)\sum_{i=0}^\infty a_ix^i&=&a_0-1+(a_1-5a_0-2)x +
\sum_{n=0}^\infty 2^ix^i\\
\sum_{i=0}^\infty a_ix^i&=& {a_0-1+(a_1-5a_0-2)x\over 1-5x+6x^2}+ {1\over
1-5x+6x^2}\cdot{1\over 1-2x}
\end{eqnarray*}}



\ep

The recurrence relations we have seen in this section are called
{\em second order}\index{recurrence!second order} because they specify
$a_i$ in terms of
$a_{i-1}$ and
$a_{i-2}$, they are called {\em
linear}\index{recurrence!linear}\index{linear recurrence} because
$a_{i-1}$ and
$a_{i-2}$ each appear to the first power, and they are called {\em
constant coefficient recurrences}\index{recurrence!constant coefficient}
because the coefficients in front of
$a_{i-1}$ and $a_{i-2}$ are constants.
\subsection{Partial fractions}  The generating functions you found in the
previous section all can be expressed in terms of the reciprocal of a
quadratic polynomial.  However without a power series representation, the
generating function doesn't tell us what the sequence is.  It turns out
that whenever you can factor a polynomial into linear factors (and over
the complex numbers such a factorization always exists) you can use that
factorization to express the reciprocal in terms of power series.

\bp
\iteme Express ${1\over x-3 } + {2\over x-2}$ as a single
fraction.\label{simplifysumoffractions}
\solution{${x-2 +2(x-3)\over(x-3)(x-2)}={3x-8\over x^2-5x+6}$}

\itemm In Problem \ref{simplifysumoffractions} you see that when we added
numerical multiples of the reciprocals of first degree polynomials we got
a fraction in which the denominator is a quadratic polynomial.  This will
always happen unless the two denominators are multiples of each other,
because their least common multiple will simply be their product, a
quadratic polynomial.  This leads us to ask whether a fraction whose
denominator is a quadratic polynomial can always be expressed as a sum of
fractions whose denominators are first degree polynomials.  Find numbers
$c$ and $d$ so that
$${5x+1\over(x-3)(x+5)} = {c\over x-3} + {d\over
x+5}.$$\label{partialfractionsintro}
\solution{
\begin{eqnarray*}{5x+1\over(x-3)(x+5)} &=& {c\over x-3} + {d\over
x+5}\\
{5x+1\over(x-3)(x-5)} &=&cx+5c+dx-3d
\end{eqnarray*}
gives us 
\begin{eqnarray*}
5x&=&cx+dx\\
1&=&5c-3d\\
\tallstrut{18}5&=&c+d\\
1&=&5c-3d\\
\end{eqnarray*} Which gives us $16=8c$ so that $c=2$ and then $d=3$.}


\iteme In Problem \ref{partialfractionsintro} you may have simply guessed
at values of
$c$ and $d$, or you may have solved a system of equations in the two
unknowns
$c$ and $d$.  Given constants $a$, $b$, $r_1$, and $r_2$ (with $r_1\not=
r_2$), write down a system of equations we can solve for
$c$ and
$d$ to write 
$${ax+b\over (x-r_1)(x-r_2)} = {c\over x-r_1} + {d\over
x-r_2}.$$\label{partialfractions1}\solution{To have
$$
{ax+b\over (x-r_1)(x-r_2)} =  {c\over x-r_1} + {d\over
x-r_2}$$
we must have
$$cx-r_2c+dx-r_2d = ax+b.
$$
This gives us the equations $cx+dx=ax$ and $-r_2c-r_1d=b$. Since $x$ can
be any value, in particular it can be nonzero, so we can divide by it.  This
gives us the equations $c+d=a$ and $r_2c+r_1d=-b$.}


\ep
Writing down the equations in Problem \ref{partialfractions1} and solving
them is called the {\em method of partial fractions}.\index{partial
fractions!method of}  This method will let you find power series
expansions for generating functions of the type you found in Problems
\ref{secondorderintroduction} to
\ref{secondordernonhomo}.  However you have to be able to factor the
quadratic polynomials that are in the denominators of your generating
functions.  

\bp
\iteme Use the method of partial fractions to convert the generating
function of Problem \ref{secondorderintroduction} into the form
$${c\over x-r_1} + {d\over x-r_2}.$$  Use this
to find a formula for $a_n$.
\solution{${10\over (1-x-2x^2)}={10\over (1-2x)(1+x)} = {c\over1-2x} +{d\over
1+x}$.  This gives us the equations $c+d=10$ and  $c-2d=0$. Thus $3d=10$, so
$d={10\over3}$, and $c=2d$ so $c={20\over3}$.  Thus
\begin{eqnarray*}\sum_{i=0}^\infty a_ix^i &=& {10\over 1-x-2x^2}\\
&=&{20/3\over 1-2x} + {10/3\over 1+x}\\
&=&{20\over 3}\sum_{i=0}^\infty (2x)^i + {10\over 3}\sum_{i=0}^\infty (-1)^ix^i
\end{eqnarray*}
Thus $a_i={20\over3}2^i +{10\over3}(-1)^i$.}

\iteme Use the quadratic formula to find the solutions to $x^2+x-1=0$, and
use that information to factor $x^2+x-1$.\label{factorFibonacci}
\solution{$r_1={-1+\sqrt{5}\over2}$, $r_2 = {-1-\sqrt{5}\over 2}$.  These roots
give us the equation 
$x^2+x-1=(x-{-1+\sqrt{5}\over2})(x-{-1-\sqrt{5}\over 2})$.}

\iteme Use the factors you found in Problem \ref{factorFibonacci}  to
write $$1\over x^2+x-1$$
in the form 
 $${c\over x-r_1} + {d\over x-r_2}.$$
(Hint:  You can save yourself a tremendous amount of frustrating algebra
if you arbitrarily choose one of the solutions and call it $r_1$ and
call the other solution $r_2$ and solve the problem using
these algebraic symbols in place of the actual roots.\footnote{We use the
words roots and solutions interchangeably.}  Not only will you save
yourself some work, but you will get a formula you could use in other
problems.  When you are done, substitute in the actual values of the
solutions and simplify.\label{fractionFibonacci})
\solution{${1\over x^2+x-1}={c\over x-r_1}+{d\over x-r_2}$ gives us
$cx-cr_2+dx-dr_1=1$.  Thus $c+d=0$, and $cr_2+dr_1 =-1$. This gives us $d=-c$
and so $cr_2-cr_1=-1$, which yields $c={1\over r_1-r_2}$, and $d={1\over
r_2-r_1}
$.  By substitution, $c=1/\sqrt{5}$ and $d=-1/\sqrt{5}$.   This gives
us $${1\over x^2+x-1} = {1/\sqrt{5}\over x-{-1+\sqrt{5}\over2}}
+{-1/\sqrt{5}\over x- {-1-\sqrt{5}\over2}}.$$}

\iteme
\begin{enumerate}
\item Use the partial fractions decomposition you found in
Problem
\ref{factorFibonacci} to write the generating function you found in
Problem \ref{originalFibonacci} in the form $$\sum_{n=0}^\infty a_nx^i$$
and use this to give an explicit formula for $a_n$.  (Hint: once again it
will save a lot of tedious algebra if you use the symbols $r_1$ and $r_2$
for the solutions as in Problem \ref{fractionFibonacci} and substitute
the actual values of the solutions once you have a formula for $a_n$ in
terms of $r_1$ and $r_2$.)  
\begin{eqnarray*}\sum_{n=0}^\infty a_nx^n &=&{1\over 1-x-x^2}\>=\>-{1\over
x^2+x-1}\\ 
&=&{1\over\sqrt{5}}\cdot{1\over r_1-x} -{1\over\sqrt{5}}\cdot{1\over r_2-x}\\
&=&{1\over r_1\sqrt{5}}\cdot{1\over1-x/r_1} -{1\over r_2\sqrt{5}}\cdot{1\over
1-x/r_2}\\ 
&=&{1\over r_1\sqrt{5}}\sum_{n=0}^\infty\left({x\over r_1}\right)^n
-{1\over r_2\sqrt{5}}\sum_{n=0}^\infty  \left({x\over r_2}\right)^n
\end{eqnarray*}
This gives us that \begin{eqnarray*}a_n&=&{1\over\sqrt{5}\cdot r_1^{n+1}}
+{1\over\sqrt{5}\cdot r_2^{n+1}}\\
&=&{2^{n+1}\over\sqrt{5}(-1+\sqrt{5})^{n+1}}
+{2^{n+1}\over\sqrt{5}(-1-\sqrt{5})^{n+1}}\\&=&{2^{n+1}(1+\sqrt{5})^{n+1}\over
\sqrt{5}\cdot
4^{n+1}}-{2^{n+1}(1-\sqrt{5})^{n+1}\over\sqrt{5}\cdot4^{n+1}}\\
&=&{1\over \sqrt{5}}\left({1+\sqrt{5}\over 2}\right)^{n+1}-
{1\over \sqrt{5}}\left({1-\sqrt{5} \over2}\right)^{n+1}.\end{eqnarray*}

\item When we have
$a_0=1$ and $a_1=1$, i.e. when we start with one pair of baby rabbits, the
numbers
$a_n$ are called {\em Fibonacci Numbers}\index{Fibonacci numbers}.  Use
either the recurrence or your final formula to find $a_2$ through $a_8$. 
Are you amazed that your general formula produces integers, or for that
matter produces rational numbers?  Why does the recurrence equation tell
you that the Fibonacci numbers are all integers?  
\solution{Using the recurrence, the Fibonacci numbers from $a_0$ to $a_8$ are
1, 1, 2, 3, 5, 8, 13, 21, 34.  The recurrence says each term is the sum of the
two preceding terms, and since the first two terms are integers, all the sums
must be integers.}

\itemitemi Find an algebraic
explanation (not using the recurrence equation) of what happens to make the
square roots of five go away. Explain why there is a real number
$b$ such that, for large values of $n$, the value of the
$n$th Fibonacci number is almost exactly (but not quite) some constant
times
$b^n$.  (Find $b$ and the constant.)
\solution{From the binomial theorem, 
\begin{eqnarray*} &&{1\over \sqrt{5}}\left({1+\sqrt{5}\over
2}\right)^{n+1}- {1\over \sqrt{5}}\left({1-\sqrt{5}
\over2}\right)^{n+1}\\&=&
{1\over2^{n+1}\sqrt{5}}\left[\sum_{i=0}^{n+1}{n+1\choose
i}\left(\sqrt{5}\right)^i -\sum_{i=0}^{n+1} {n+1\choose
i}(-1)^i\left(\sqrt{5}\right)^i\right]\\ &=&{1\over2^{n+1}\sqrt{5}}\sum_{i:i\in
[n+1],\ i\mbox{\scriptsize~is odd}}{n+1\choose i}\left(\left(\sqrt{5}\right)^i
-(-1)^i\left(\sqrt{5}\right)^i\right)\\
&=&{1\over2^{n+1}\sqrt{5}}\sum_{i:i\in[n+1],\ i\mbox{\scriptsize~is
odd}}	2{n+1\choose
i}\left(\sqrt{5}\right)^i \\
&=&{1\over2^{n}}\sum_{i:i\in[n+1],\ i\mbox{\scriptsize~is
odd}}	{n+1\choose
i}\left(\sqrt{5}\right)^{i-1}\\
&=&{1\over2^{n}}\sum_{i:i\in[n],\ i\mbox{\scriptsize~is
even}}	{n+1\choose
i+1}5^{i/2} \\
&=&{1\over 2^n}\sum_{k=0}^{\lfloor n/2 \rfloor}{n+1\choose 2k+1}5^k,
\end{eqnarray*} which makes it clear that $a_n$ is at least a rational
number. It is not clear from this new formula why the result is always an
integer. Since
$\left|{1-\sqrt{5}\over 2}\right|<1$,
${1\over\sqrt{5}}\left({1-\sqrt{5}\over 2}\right)^{n-1}$ approaches 0 as $n$
becomes large.  Therefore if we take $b$ to be $\left({1+\sqrt{5}\over
2}\right)^n$ and we take $c$ to be ${1+\sqrt{5}\over 2\sqrt{5}}$ then then
$n$th Fibonacci number is almost exactly $cb^n$ when $n$ is large.  In
particular, it is the nearest integer to $cb^n$.}
\itemitemih As a challenge (which the author has not yet done), see if you can
find a way to show algebraically (not using the recurrence relation, but rather
the formula you get by removing the square roots of five) that the formula for
the Fibonacci numbers yields integers.
\solution{None is yet available.} 
\end{enumerate}

\item  Solve the recurrence $a_n= 4a_{n-1} - 4a_{n-2}$.
\solution{\begin{eqnarray*}
\sum_{n=2}^\infty a_nx^n &=& 4\sum_{n=2}^\infty a_{n-1}x^n -
4\sum_{n=2}^\infty a_{n-2}x^n\\
\sum_{n=0}^\infty a_nx^n -a_0-a_1x &=& 4x(\sum_{n=0}^\infty a_{n}x^n
-a_0) - 4x^2\sum_{n=0}^\infty a_{n}x^n\\
(1-4x+4x^2)\sum_{n=0}^\infty a_nx^n  &=& 
a_0+a_1x-4a_0x \\
\sum_{n=0}^\infty a_nx^n&=&{a_0+a_1x-4a_0x\over (1-4x+4x^2)}\\
\sum_{n=0}^\infty a_nx^n&=&{a_0+a_1x-4a_0x\over (1-2x)^2}\\
\sum_{n=0}^\infty a_nx^n&=&(a_0+a_1x-4a_0x)\sum_{n=0}^\infty {n+2-1\choose n}
2^nx^n\\
\sum_{n=0}^\infty a_nx^n&=&(a_0+a_1x-4a_0x)\sum_{n=0}^\infty (n+1)
2^nx^n
\end{eqnarray*}Thus $a_n=a_0(n+1)2^n
+(a_1-4a_0)n2^{n-1}=a_02^n+(a_1-2a_0)n2^{n-1}$.}

\ep


\subsection{Catalan Numbers}

\bp
\itemi Using either lattice paths or diagonal lattice paths, explain why
the Catalan Number\index{Catalan Number!recurrence for} $c_n$ satisfies
the recurrence

$$C_n = \sum_{i=1}^{n-1} C_{i-1}C_{n-i}.$$

Show that if we use $y$ to stand for the power series $\sum_{i=0}^\infty
c_nx^n$, then we can find $y$ by solving a quadratic equation.  Solve for
$y$.  Taylor's theorem from calculus tells us that the extended binomial
theorem  $$(1+x)^r = \sum_{i=0}^\infty {r\choose i}x^i$$
holds for any number real number $r$, where $r\choose i$ is defined to be
$${r^{\underline{i}}\over i!} = {r(r-1)\cdots(r-i+1)\over i!}.$$
Use this and your solution for $y$ (note that of the two possible values
for $y$ that you get from the quadratic formula, only one gives an actual
power series) to get a formula for the Catalan numbers.\index{Catalan
number!generating function for}\label{CatalanRecurrence}
\solution{Recall that  a Dyck Path is a diagonal lattice path that never goes
lower than its starting point and a Catalan Path of length $2n$ is a Dyck path
that goes from
$(0,0)$ to $(2n,0)$  The Catalan number $C_n$ is the number of Catalan paths
of length $2n$.  We take $C_0=1$.  A Catalan path could touch the $x$-axis
several times before it reaches $(2n,0)$.  Its first touch can be any point
$(2i,0)$ between
$(2,0)$ and
$(2n,0)$.  For the path to touch first at $(0,2i)$, the path must start with
an upstep and then proceed as a Dyck path from $(1,1)$ to $(2i-1,1)$.  From
there it must take a downstep.  But the number of Dyck paths from $(1,1)$ to
$(2i-1,1)$ is the same as the number of Catalan paths from $(0,0)$ to
$(2i-2,0)$.  The number of Catalan paths is the number whose first touch of the
$x$-axis is at $x=2$ plus the number whose first touch is at $x=4$,\ldots,
through the number whose first touch is at $2n$.  After the first touch at
$x=2i$, the path then behaves as a Catalan path from $(2i,0)$ to $(2n,0)$.  The
number of such paths is $C_{n-i}$.  By the product principle, the number of
Catalan paths whose first touch is at $x=2i$ is $C_{i-1}C_{n-i}$.  Then by the
sum principle, the number of Catalan paths of length 1 or more is
$$C_n=\sum_{i=1}^nC_{i-1}C_{n-i}.$$

To solve for $C_n$, write
\begin{eqnarray*}
\sum_{n=0}^\infty C_nx^n &=&1+\sum_{n=1}^\infty\sum_{i=1}^nC_{i-1}C_{n-i}x^n\\
\sum_{n=0}^\infty C_nx^n
&=&1+x\sum_{n=1}^\infty\sum_{i=1}^nC_{i-1}x^{i-1}C_{n-i}x^{n-i}\\
\sum_{n=0}^\infty C_nx^n
&=&1+x\sum_{i=1}^\infty C_{i-1}x^{i-1}\sum_{j=0}^\infty C_{j}x^{j}\\
y&=&1 + xy^2
\end{eqnarray*}
This gives us $xy^2-y+1=0$, and solving for $y$ by the quadratic formula gives us
$y={1\pm \sqrt{1-4x}\over 2x}$.  By the extended binomial theorem,
$$\sqrt{1-4x}=(1-4x)^{1/2} = \sum_{i=0}^\infty {1/2\choose i}(-4x)^i=
\sum_{i=0}^\infty {(1/2)^{\underline{i}}\over i!}(-1)^i4^ix^i.$$
The first term of this power series is 1, so to get a power series for $y$, we must
take the negative square root so that the $x$ in the denominator will cancel
out.  Thus $y=-{1\over2}\sum_{i=1}^\infty {(1/2)^{\underline{i}}\over
i!}(-1)^i4^ix^i$. But
$(1/2)^{\underline{i}}=({1\over2})({-1\over2})({-3\over2})({-5\over2})\cdots
({-2i+3\over 2})$, so 
\begin{eqnarray*}y&=&-{1\over2}\sum_{i=1}^\infty{1\cdot3\cdot5\cdots(2i-3)\over
i!}(-1)^{2i-1}2^ix^{i-1}\\
&=& \sum_{i=1}^\infty {(2i-2)!\over
(i-1)!2^{i}i!}2^ix^{i-1}\\&=& \sum_{i=1}^\infty{2i-2!\over i!(i-1)!}x^{i-1}\\
&=&\sum_{j=0}^\infty{2j!\over (j+1)!j!}x^j
\end{eqnarray*}
giving us $C_j={1\over j+1}{2j\choose j}$, which is our earlier formula for the
Catalan Number $C_j$.}


\ep

\section{Supplementary Problems}
\begin{enumerate}
\item Each person attending a party has been asked to bring a prize.  The
person planning the party has arranged to give out exactly as many prizes
as there are guests, but any person may win any number of prizes.  If
there are $n$ guests, in how many ways may the prizes be given out so
that nobody gets the prize that he or she brought?
\solution{We use inclusion and exclusion.  Property $P_i$ is that person $i$
gets the prize he or she brought.  We are interested in
$N_{\mbox{e}}(\emptyset)$.  If
$P$ is the set of all properties, we need to compute $N_{\mbox{a}}(S)$ for every
subset
$S$ of
$P$. But $N_{\mbox{a}}(S)$ is the number of
functions from the prizes to the people that give each prize represented by a
property in
$S$ to the person who brought it.  Think in terms of distributing those prizes
first. Then there are
$n-|S|$ other prizes that we may pass out to the $n$ people as we please, so we
may do that in $n^{n-|S|}$ ways.  Thus $N_{\mbox{a}}(S)=n^{n-|S|}$.  Applying
Equation \ref{incexempty}, we get
$$N_{\mbox{e}}(\emptyset)=\sum_{S:s\subseteq P}(-1)^{|S|} n^{n-|S|} =
\sum_{s=0}^n (-1)^{|S|}{n\choose s}n^{n-s}.$$}
 
\item There are $m$ students attending a seminar in a room with $n$
seats.  The seminar is a long one, and in the middle the group takes a
break.  In how many ways may the students return to the room and sit down
so that nobody is in the same seat as before?
\solution{We use inclusion and exclusion.  We let property $P_i$ be the
property that student $i$ sits in the same seat as before.  We are interested
in $N_{\mbox{e}}(\emptyset)$. For this purpose, for each subset $S$ of the set
$P$ of all properties, we need to  compute $N_{\mbox{a}}(S)$, the number of
ways for the students to return so that every student represented by a property
in $S$ sits in his or her previous seat.  This leaves us with $n-|S|$ seats to
be filled in a one-to-one fashion in $m-|S|$ students. There are
$(n-|S|)^{\underline{m-|S|}}$ such seating arrangements, so $N_{\mbox{a}}(S)=
(n-|S|)^{\underline{m-|S|}}$.  Thus we have
$$N_{\mbox{e}}(\emptyset)=\sum_{S:S\subseteq P}
(-1)^{|S|}(n-|S|)^{\underline{m-|S|}} = \sum_{s=0}^m (-1)^s{m\choose
s}(n-s)^{\underline{m-s}}$$ ways for the students to return so that nobody sits
in his or her previous place.}


\itemi In how many ways may $k$ distinct books be arranged on $n$ shelves
so that no shelf gets more than $m$ books?
\solution{We use inclusion and exclusion.  Let property $P_i$ be that shelf $i$
gets more than $m$ books.  Then the number of arrangements of books with at
least the properties in a subset $S$ of the set $P$ of all properties is
$N_{\mbox{a}}(S) = k^{\underline{(m+1)|S|}}n^{\underline{k-(m+1)|S|}}$, because
in order to have at least the properties in $S$ we may choose $(m+1)|S|$ books
and arrange $m+1$ of them on each of the shelves represented by the properties in
$S$, after which we arrange the remainder of the books.  Thus
$$N_{\mbox{e}}(\emptyset)=\sum_{S:S\subseteq P} (-1)^{|S|}
k^{\underline{(m+1)|S|}}n^{\underline{k-(m+1)|S|}}=\sum_{s=0}^n
(-1)^s{n\choose s}k^{\underline{(m+1)s}}n^{\underline{k-(m+1)s}}$$
is the number of ways to arrange the books so that no shelf gets more than $m$.}

\itemi We have said that for nonnegative $i$ and positive $n$ we want to define
$-n\choose i$ to be $n+i-1\choose i$. If we want the Pascal recurrence to be
valid, how should we define $-n\choose -i$ when $n$ and $i$ are both positive?
\solution{The number $n\choose k$ is the number in row $n$ and column $k$ of
the Pascal (right) triangle.  We have said we want to have ${-n\choose
0}={n+0-1\choose 0}$, so we want ones everywhere in that column.  Now the Pascal
recurrence gives us that
${-n\choose 0}={-n-1\choose-1} +{-n-1\choose 0}$, so that ${-n\choose-1}=0$,
as does $0\choose -1$.  Applying the Pascal recurrence again gives us
${-n\choose -1}= {-n-1\choose -2} +{-n-1\choose -1}$, so we have ${-n-1\choose
-2}=0$ as well.  Following this pattern, we can prove by induction that
$-n\choose -k$ is zero whenever $k$ and $n$ are positive.}

\itemi Suppose that $n$ children join hands in a circle for a game at
nursery school. The game involves everyone falling down (and letting
go).  In how many ways may they join hands in  a circle again so that
nobody is to the right of the same child that was previously to his or
her right?
\solution{We use inclusion and exclusion, with property $i$ being the property
that child $i$ has the same child to the right the second time they join hands.
Given a set $S$ of properties, we can think of arranging units consisting of
individual children and strings of children holding hands in a circle.  If we
have $s$ properties, then we have $n-s$ of these units.  Each string of
children can be arranged in only one way, because our set specifies who has to
have the same child on the right.  Thus $N_{\mbox{a}}(S) = (n-s-1)!$.  This
gives us 
\begin{eqnarray*}N_{\mbox{e}}(\emptyset)=\sum_{S:S\subseteq P}
(-1)^{|S|}(n-|S|-1)!&=&
\sum_{s=0}^n (-1)^s {n\choose s} (n-s-1)!\\ &=&\sum_{s=0}^n (-1)^s {n!\over
s!(n-s)}
\end{eqnarray*}
 ways for the children to join hands the second time so that none
of them has the same child to the right.}

\itemih Suppose that $n$ people link arms in a folk-dance and dance in a
circle.  Later on they let go and dance some more, after which they
link arms in a circle again.  In how many ways can they link arms the
second time so that no-one is next to a person with whom he or she
linked arms before.\label{Hora}
\solution{We use the Principle of Inclusion and Exclusion.  Property $P_i$ will
be the property that person $i$ links arms with someone previously to his or
her right.  (Saying it is the person to the right gives us more control over
our formulas.)  Given a subset $S$ of the set $P$ of all properties, the number
of ways for the people in that set to link arms with the people previously on
their right is the number of ways to arrange $n-|S|$ strings of people around a
circle with strings of length more than 1 having two ways to arrange themselves. 
(Once we have two or more people linked, another person can be added to this
string only at one end or not at all, because this person must have been to the
right of one of the people on an end of the string. However a string of length
two or more can unlink and then link in the opposite order, and each person will
still be linked to exactly the same people.)  Thus $N_{\mbox{a}}(S)=
(n-|S|-1)!2^{m(S)}$, where $m(S)$ is the number of strings of length more than
one determined by $S$. The number $m(S)$ can be any number from 1 to $S$, so
long as
$S$ is not too big; namely so long as $|S|\le \lfloor n/1\rfloor$.  (This is
because if $m(S)=|S|$, then each person determined by a property in $S$ must be
adjacent to a person not determined by a property in $S$.)  In particular,
$N_{\mbox{a}}(S)$ is not completely determined by the size of $S$, as in all our
other inclusion-exclusion problems.  How do we compute $m(S)$?  Let us call a
subset $R$ of $S$ a run if 
\begin{enumerate}
\item the people determined by $R$ sit together in a
row in both seatings, and 
\item no other person in $S$ is in a row with these
people in both seatings. 
\end{enumerate} Some runs might determine just one person, but a run could
also equal all of $S$.  Each run will have one more person not in $S$ who was
originally to the right of the person in the run who was rightmost in the first
seating, and so this person will have to sit in a row with the people in $R$ in
the second seating as well.  Thus the number $r$ of runs in $S$ is the number of
strings
$m(S)$ that may be seated in two ways, and there are $n-|S|-r$ people who do not
have to be seated with runs.  Thus $N_{\mbox{a}}(S) = (n-|S|-1)!2^r$, because
the total number of strings of people (including strings of just one person) we
need to seat is $n-|S| -r$, and there are $(k-1)!$ ways to arrange $k$ objects
in a circle.. If we try to use the information we have so far to compute
$N_{\mbox{e}}(\emptyset)$, we get
$$N_{\mbox{e}}(\emptyset)=\sum_{S:S\subseteq P}(-1)^{|S|}(n-|S|_1)!2^r
=\sum_{s=0}^n\sum_{r=1}^{|S|}N(s,r)(n-s-1)!2^r,$$ in which $N(s,r)$ stands for
the number of property sets with size $s$ and $r$ runs.

Picking out runs in a circular arrangement adds a layer of difficulty, so to
compute $N(s,r)$, we first compute how many subsets of $[n]$ we have with $r$
runs and then adjust for putting $1$ through $n$ around a circle in order.
Imagine writing 1 through
$n$ in a straight line, each integer occupying one unit of distance along the
line. We now place
$r$  sticks whose lengths add to
$s$ (each stick has positive integer length) along that line. Each stick
picks out a set of consecutive integers, as many as its length, so the
sticks together pick out $s$ integers.  In order to be sure the sticks
correspond to runs, we need to make sure they do not touch each other, so we
place $n-s$ identical stones along the line too, making sure there is at
least one stone between any two sticks.  The stones thus pick out the integers
not in $S$.  The sticks are not quite identical, though the sticks of a given
length are.  In other words, which lengths of sticks are in which places is
what matters.  So the sticks give us a composition of $s$, a list of distinct
positive integers that add to $s$.  We know there are $s-1\choose r-1$ such
compositions. Once we have chosen an ordering for the sticks, we need to
distribute the stones among the sticks so that no two sticks are adjacent. 
Since the stones are identical, we can do this by putting one stone between
each pair of sticks in our composition, and then distribute the remaining
$n-s-r+1$  stones in any way we want among the $r-1$ places between the sticks
and the two places to the left and right of all the sticks..  We can do this in
$${r+1 + (n-s-r+1)-1\choose n-s-r+1}={n-s+1\choose n-s-r+1}={n-s+1\choose r}$$
ways. Thus there are
${s-1\choose r-1}{n-s+1\choose r}$ ways to choose a subset $S$ of $[n]$ that
has
$r$ runs.

Now we have to deal with the fact that our $n$ people (who we have replaced
with the integers 1 through $n$ in order) were arranged around a circle. That
means that a run is now a set of consecutive integers on the circle, where $n$
and 1 are considered consecutive.  Recall that the set $S$ is picked out by the
sticks. If we arrange 1 through
$n$ around a circle in order, the set
$S$ that originally had
$r$ runs will have $r-1$ runs if sticks covered both the first and last integer
(1 and $n$), but otherwise it will still have $n$ runs.  Thus the number of
subsets of $[n]$ that have $n$ runs when $1$ through $n$ are  arranged in a
circle is the number of subsets of $[n]$ with $r+1$ runs that have both 1 and
$n$ in $S$ plus the number of subsets of $[n]$ with $r$ runs that do not have
both $1$ and
$n$ in $S$. To compute the number of subsets $S$ that {\em do} contain both 1
and $n$, we compute the number of arrangements of  $r$ sticks and $n-s$ stones
that do start and end with a stick; that means that after we choose our
composition into $r$ parts to get our arrangement of sticks and place one stone
between each pair of previously adjacent sticks, we now place the remaining
$n-s-r+1$ stones in the $r-1$ places between previously adjacent sticks in
$${r-1 + (n-s-r+1)-1\choose n-s-r+1}={n-s-1\choose n-s-r+1}={n-s-1\choose r-2}$$
ways. For the sticks and stones to determine a subset we must assign lengths to
the sticks; the number of ways to do this is, as above, $s-1\choose r-1$, the
number of compositions of $s$ with $r$ parts. Thus there are
${s-1\choose r-1}{n-s-1\choose r-2}$ subsets of
$[n]$ that have
$r$ runs and include both 1 and $n$.  For our computation we will also want the
number of subsets of $[n]$ that have $r+1$ runs and contain both $1$ and $n$
this is
${s-1\choose r}{n-s-1\choose r-1}$.

On the other hand, the number of subsets of $[n]$ that have $r$ runs and do not
contain both $1$ and $n$ is the total number of subsets with $r$ runs minus the
number that do contain both $1$ and $n$; this is $${s-1\choose
r-1}\left({n-s+1\choose r}- {n-s-1\choose
r-2}\right).$$ 


This gives us $$N(s,r) ={s-1\choose r}{n-s-1\choose r-1}+{s-1\choose
r-1}\left({n-s+1\choose r}- {n-s-1\choose
r-2}\right)$$ 
ways to choose an $s$-element subset of $[n]$ that has $r$ runs when $[n]$ is
arranged around a circle.  Thus there are\\
$\tallstrut{24}\sum\limits_{s=0}^n\sum\limits_{r=1}^{s}(-1)^s\left[{s-1\choose
r}{n-s-1\choose r-1}+{s-1\choose r-1}\left({n-s+1\choose r}- {n-s-1\choose
r-2}\right)\right](n-s-1)!2^r$
\\\tallstrut{18} ways for people to arrange themselves in the
second circle so that no-one is adjacent to anyone he or she was previously
adjacent to.
}

\itemih (A challenge; the author has not tried to solve this one!)  Redo
Problem \ref{Hora} in the case that there are $n$ men and $n$ women and
when people arrange themselves in a circle they do so alternating gender.

\item  What is the generating function for the number of ways to pass out
 $k$ pieces of candy from
an unlimited supply of identical candy to  $n$ children (where
$n$ is fixed) so that each child gets between three and six pieces of
candy (inclusive)?  Use the fact that
$$(1+x+x+x^3)(1-x) = 1-x^4$$ to find a formula for the number of ways to
pass out the candy.  Reformulate this problem as an inclusion-exclusion
problem and describe what you would need to do to solve it.
\solution{$(x^3+x^4+x^5+x^6)^n$; 
\begin{eqnarray*}(x^3+x^4+x^5+x^6)^n&=&x^{3n}(1+x+x^2+x^3)^n\\
&=&x^{3n}\left({1-x^4\over 1-x}\right)^n\\
&=&x^{3n}\sum_{j=0}^n (-1)^j{n\choose j}x^{4j}\sum_{i=0}^\infty
{n+i-1\choose i}x^i
\end{eqnarray*}  The number of ways to pass out $k$ pieces of candy is the
coefficient of $x^k$ in this expression.  Thus the answer is zero if $k< 3n$
because of the $x^{3n}$ in front.  Otherwise the answer is the coefficient of
$^{k-3n}$ in $\sum_{j=0}^n (-1)^j{n\choose j}x^{4j}\sum_{i=0}^\infty
{n+i-1\choose i}x^i$, which is $$\sum_{i,j:4j+i=k-3n}(-1)^j{n\choose
j}{n+i-1\choose i}= \sum_{j=0}^{\lfloor (k-3n)/4\rfloor} (-1)^j{n\choose
j}{k-2n-4j-1\choose n-1}.$$

As an inclusion-exclusion problem, we would let property $i$ be that child $i$
gets more than six pieces of candy. We would then observe that the number of
ways to pass out the candy so that the children determined by a set $S$ of
properties all get more than six pieces, and everyone else gets at least 3, is
the number of ways to pass out the remaining candy after giving 7 pieces to each
child identified by $S$ and 3 pieces to each other child.  This number is
${k-7|S|-3(n-|S|)-1\choose n-1}={k-2n-4|S|-1\choose n-1}$.  From here we would
substitute into Equation
\ref{incexempty}, make any simplifications we could, and we would be done.}

\itemm
\begin{enumerate}
\item In paying off a mortgage loan with initial amount A,
annual interest rate
$p$\% on a monthly basis with a monthly payment of $m$, what recurrence
describes the amount owed after $n$ months of payments in terms of the
amount owed after $n-1$ months?  Some technical details:  You make the
first payment after one month.  The amount of interest included in your
monthly payment is $.01p/12$.  This interest rate is applied to the amount
you owed immediately after making your last monthly payment.  
\solution{$a_n=(1+{.01p\over 12})a_{n-1}-m$.}

\item Find a
formula for the amount owed after $n$ months.  
\solution{From Problem
\ref{firstordlinconst} or by applying generating functions we have 
\begin{eqnarray*}a_n &=& A(1+{.01p\over 12})^n-m{1-(1+{.01p\over
12})^n\over 1-(1+{.01p\over 12})}\\ &=& \left(A-{1200m\over
p}\right)\left(1+{.01p\over 12}\right)^n+{1200m\over p}\end{eqnarray*}}

\item Find a formula for the
number of months needed to bring the amount owed to zero.  Another
technical point:  If you were to make the standard monthly payment $m$ in
the last month, you might actually end up owing a negative amount of
money.  Therefore it is ok if the result of your formula for the number
of months needed gives a non-integer number of months.  The bank would
just round up to the next integer and adjust your payment so your balance
comes out to zero.
\solution{$\left(A-{1200m\over
p}\right)\left(1+{.01p\over 12}\right)^n+{1200m\over p}=0$ gives us the
equation $\left(1+{.01p\over 12}\right)^n={1200m\over 1200m-Ap}$. Taking
logarithms to any base we choose gives us $n\log (1+{.01p\over 12})=\log
1200m-\log (1200m-Ap)$ and so $n={\log
1200m-\log (1200m-Ap)\over \log (1+{.01p\over 12})}$.}

\item What should the monthly payment be to pay off the loan over a period of 30
years?
\solution{$360={\log
1200m-\log (1200m-Ap)\over \log (1+{.01p\over 12})}$ is the equation we need to
solve for $m$, the monthly payment.  We need to chose some base for the
logarithm so we can write its inverse function; suppose we use logs to the base
10.  Then 
\begin{eqnarray*}360\log (1+{.01p\over 12}) &=& \log
1200m-\log (1200m-Ap)\\
10^{360\log (1+{.01p\over 12})}&=&10^{\log
1200m/(1200m-Ap)}\\
 (1+{.01p\over 12})^{360} &=&1200m/(1200m-Ap)\\
\left(1200m-Ap\right)(1+{.01p\over 12})^{360}&=&1200m\\
1200m\left((1+{.01p\over 12})^{360}-1\right)&=& Ap(1+{.01p\over 12})^{360}\\
m &=& {Ap(1+{.01p\over 12})^{360}\over 1200\left((1+{.01p\over
12})^{360}-1\right)} 
\end{eqnarray*}
is our monthly payment.}
\end{enumerate}

\itemi Find a recurrence relation for the number of ways to divide a
convex $n$-gon into triangles by means of non-intersecting diagonals. 
How do these numbers relate to the Catalan numbers?\index{Catalan
Number!recurrence for}
\solution{Let $d_n$ be the number of ways to divide an $n$-gon into triangles
by means of nonintersecting diagonals.  Take an $n$-gon and label its vertices
cyclically from 1 to $n$.    Any triangulation must have a triangle containing
the edge $1n$ between vertex 1 and vertex $n$.  The third vertex can be any
number between 2 and
$n-1$. We consider two cases. First, if the third vertex is 2 or $n-1$, then we
have divided our polygon up into a triangle and an
$(n-1)$-gon, and any triangulation of that $(n-1)$-gon joins with our original
triangle to give us a triangulation of the $n$-gon. Second, if the third vertex
of our original triangle is vertex $i$ with $3\le i\le n-2$ then we have divided
our polygon into the polygon with the $i$ edges $12$, $23$, \ldots,
$(i-1)i$, $i1$, the polygon with $n-i+1$ edges $ni$, $i(i+1)$, \ldots, $(n-1)n$,
and the original triangle with edges
$n1$, $1i$, $in$. Triangulations of the first two of these polygons join with the
original triangle to give us a triangulation of the original polygon.

The number of triangulations of the original polygon that we get from case 1 is
$2d_{n-1}$.  The number of triangulations we get from the second case is
$\sum_{i=3}^{n-2} d_id_{n-i+1}$.  Thus the total number of triangulations is
$2d_{n-1}+d_{n-2}d_3 + d_{n-3}d_4 +\cdots + d_3d_{n-2}$.  If we take $d_2=1$,
then we may write $d_n = d_{n-1}d_2+d_{n-2}d_3 +\cdots+d_3d_{n-2} +
d_2d_{n-1}=\sum\limits_{i=2}^{n-1} d_id_{n-i+1}$.  This is very similar to the
recurrence in Problem \ref{CatalanRecurrence} for the Catalan numbers.  We could
apply the generating function method we used with the Catalan numbers to find a
formula for $d_n$.  We could also experiment with the first few Catalan numbers
and the first few ``triangulation'' numbers to see if they are related.  We
have $C_0=1$, $C_1=1$, $C_2=2$, $C_3=C_0C_2+C_1C_1 +C_2C_0=5$, and
$C_4=C_0C_3+C_1C_2+C_2C_1+C_3C_0=14$.  We have that $d_2=1$, $d_3=1$, $d_4=2$,
$d_5=d_4d_2+d_3d_3+d_2d_4=5$, and $d_6=d_5d_2+d_4d_3+d_3d_4+d_5d_2=14$.  This
makes pretty convincing evidence that $d_i= C_{i-2}$.  We have already done a
base case (and more) for an inductive proof. So assume inductively that
$d_{i}=C_{i-2}$ for $i<n$.  Then 
\begin{eqnarray*}d_n&=&\sum\limits_{i=2}^{n-1} d_id_{n-i+1}\\
&=&\sum\limits_{i=2}^{n-1} C_{i-2}C_{n-i+1-2}\\
&=&\sum\limits_{i=2}^{n-1} C_{i-2}C_{n-i-1}\\
&=&\sum\limits_{k=1}^{n-2} C_{k-1}C_{n-(k+1)-1}\\
&=&\sum\limits_{k=1}^{n-2} C_{k-1}C_{(n-2)-k}\\
&=&C_{n-2}
\end{eqnarray*}
Thus by the principle of mathematical induction, $d_n=C_{n-2}$ for all integers
$n\ge 2$.}

\itemi How does $\sum_{k=0}^n{n-k\choose k}$ relate to the Fibonacci
Numbers?
\solution{We begin by computing a few values of 
$a_n=\sum_{k=0}^n{n-k\choose
k}$.  We have $a_0=1$, $a_1=1$, $a_2=2$, $a_3=1+2=3$, $a_4=1+3+1=5$,
$a_5=1+4+3=8$ and $a_6=1+5+6+1=13$.  So far the sequence agrees with the
Fibonacci Numbers.  Each term of the sequence is the sum of the two preceding
terms, so it makes sense to try to prove that $a_n=a_{n-1}+a_{n-2}$.  We may
write
\begin{eqnarray*}
a_{n-1}+a_{n-2}
&=&\sum_{k=0}^{n-1} {n-1-k\choose k}
+\sum_{k=0}^{n-2} {n-2-k\choose k}\\
&=&\sum_{k=0}^{n-1} {n-1-k\choose k}+\sum_{j=1}^{n-1}{n-1-j\choose j-1}\\
&=&1+\sum_{k=1}^{n-1}{n-1-k\choose k}+{n-1-k\choose k-1}\\
&=&1+\sum_{k=1}^{n-1}{n-k\choose k} \>=\> \sum_{k=0}^{n-1}{n-k\choose k}
\>=\> a_n
\end{eqnarray*}Thus the sequence satisfies the same recurrence as the Fibonacci
numbers and its first two values are the same as the Fibonacci numbers.  This
lets us prove by induction that $a_n$ is the $n$th Fibonacci number.  More
generally, given any second order recurrence, if two sequences satisfy that
recurrence and have the same first two values, then they are equal.}

\item Let $m$ and $n$ be fixed.  Express the generating function for the
number of
$k$-element multisets of an $n$-element set such that no element appears
more than
$m$ times as a quotient of two polynomials.  Use this expression to get a
formula for the number of $k$-element
multisets of an $n$-element set such that no element appears more than
$m$ times.\solution{$(1+x+x^2+\cdots+x^m)^n={(1-x^{m+1})^n\over (1-x)^n}$. 
Expanding this gives us ${(1-x^{m+1})^n\over (1-x)^n}=\sum_{i=0}^n(-1)^i
{n\choose i}x^{(m+1)i} \sum_{j=0}^\infty {n+j-1\choose j}$.  The coefficient of
$x^k$ in this product is the number of $k$-element multisets chosen from an
$n$-element set in which no element appears more than $m$ times.  This
coefficient is
$\sum\limits_{i,j:(m+1)i+j=k}(-1)^i{n\choose i}{n+j-1\choose
j}=\sum\limits_{i=1}^{\lfloor{k\over m+1}\rfloor}(-1)^i{n\choose
i}{n+k-(m+1)i-1\choose n-1}$.}

\item One natural but oversimplified model for the growth of a tree is
that all new wood grows  from the previous year's growth and is
proportional to it in amount.  To be more precise, assume that the
(total) length of the new growth in a given year is the constant $c$
times the (total) length of new growth in the previous year.  Write down
a recurrence for the total length $a_n$ of all the branches of the tree
at the end of growing season
$n$.  Find the general solution to your recurrence relation.  Assume that
we begin with a one meter cutting of new wood (from the previous year) which
branches out and grows a total of two meters of new wood in the first year. 
What will the total length of all the branches of the tree be at the end of $n$
years?
\solution{$a_n= a_{n-1} +c(a_{n-1}-a_{n-2})=(1+c)a_{n-1}-ca_{n-2}$.
\begin{eqnarray*}\sum_{n=2}^\infty
a_nx^n&=&\sum_{n=2}^\infty(1+c)a_{n-1}x^n-c\sum_{n=2}^\infty a_{n-2}x^n\\
(1-(1+c)x +cx^2)\sum_{n=0}^\infty
a_nx^n&=&a_0+a_1x-a_0(1+c)x\\
\sum_{n=0}^\infty a_nx^n &=& {a_0+(a_1-a_0(1+c))x\over 1-(1+c)x +cx^2}\\
&=& {a_0+(a_1-a_0(1+c))x\over (1-x)(1-cx)}
\end{eqnarray*} Assuming $c\not=1$ and using the method of partial fractions
gives us 
\begin{eqnarray*}
&&{a_0+(a_1-a_0(1+c))x\over (1-x)(1-cx)}\\ &=& (a_0+(a_1-a_0(1+c))x)
\left[{1/(1-c)\over (1-x)} -{c/(1-c)\over 1-cx}\right]\\
&=& (a_0+(a_1-a_0(1+c))x)\left[{1\over 1-c}\sum_{i=0}^\infty x^i-{c\over
1-c}\sum_{i=0}^\infty c^ix^i\right]\\
&=& {(a_0+(a_1-a_0(1+c))x)\over 1-c}\sum_{i=0}^\infty
(1-c^{i+1})x^i.
\end{eqnarray*}
From this we get that $$a_i={a_0\over 1-c}(1-c^{i+1}) +{a_1-a_0(1+c)\over
1-c}(1-c^i).$$

Assuming that we begin with one meter of new wood means $a_0=1$, and assuming
we have a total of two meters of new wood at the end of the first year means
$c=2$ and $a_1=3$. Substituting these into our formula for $a_i$ gives us
$a_i=2^{i+1}-1$}

\itemi (Relevant to Appendix \ref{expogenfun}) We have some chairs which we are
going to paint with red, white, blue, green, yellow and purple paint.  Suppose
that we may paint any number of chairs red or white, that we may paint at most
one chair blue, at most three chairs green, only an even number of chairs
yellow, and only a multiple of four chairs purple.  In how many ways may we
paint $n$ chairs?
\solution{The generating function for the number of ways to paint $n$ chairs is 
{\small
\begin{eqnarray*}
&&\hspace*{-.5in}(1+x+x^2+\cdots)^2(1+x)(1+x+x^2+x^3)(1+x^2+x^4+\cdots)
(1+x^4+x^8+\cdots)\\ &=&
{(1+x)(1+x+x^2+x^3)\over(1-x)^2(1-x^2)(1-x^4)}\>=\> {1\over
(1-x)^4}\end{eqnarray*}} Thus the number of ways to paint $n$ chairs is
${n+4-1\choose n}={n+3\choose n}$}

\item What is the generating function for the number of partitions of an
integer in which  each part is used at most $m$ times?  Why is this also
the generating function for partitions in which consecutive parts (in a
decreasing list representation) differ by at most $m$?
\solution{
$$(1+q+\cdots+q^m)(1+q^2 +\cdots+q^{2m})\cdots 
=\prod_{i=1}^\infty\sum_{j=0}^m q^{ij}\\
=\prod_{i=1}^\infty {1-q^{i(m+1)}\over 1-q^i}
$$ This is also the generating function for the number of
partitions of an integer in which consecutive parts differ by at most $m$,
because when we conjugate a partition in which each part is used at most $m$
times, we get a partition in which each distinct column of the Young diagram
occurs at most $m$ times, which means that the difference between two
consecutive parts (in the decreasing list representation) is at most $m$.}

\itemi Suppose we take two graphs $G_1$ and $G_2$ with disjoint vertex sets, we
choose one vertex on each graph, and connect these two graphs by an edge $e$ to
get a graph $G_{12}$.  How does   the chromatic polynomial of $G_{12}$ relate to
those of
$G_1$ and
$G_2$?
\solution{By the deletion-contraction recurrence, $$\Chi_{G_{12}}(x) =
\Chi_{G_{12}-e}(x)-\Chi_{G_{12/e}}(x).$$ 
Now $\Chi_{G_{12}-e}(x)=\Chi_{G_1}(x)\times \Chi_{G_2}(x)$ because 
each ordered pair of proper colorings of $G_1$ and $G_2$ is a proper coloring of
$G_{12}-e$.  $G_{12}/e$ is the graph we get by identifying the endpoint of $e$
in $G_1$ with the endpoint of $e$ in $G_2$. Notice that if you fix one vertex
of a graph $G$, fix one color, and ask how many proper colorings with $x$
colors
$G$ has in which the fixed vertex is the fixed color, you get $\Chi_G(x)/x$. 
(By the quotient principle.)  Thus $\Chi_{G_2}(x)/x$
is the number of ways to extend a proper coloring of
$G_1$ to a proper coloring of $G_{12}/e$.  Then, by the product principle, the
number of proper colorings of $G_{12}/e$ with $x$ colors is
$\Chi_{G_1}(x)\Chi_{G_2}(x)/x$.  Therefore by the deletion-contraction
recurrence,
$\Chi_{G_{12}}(x) = \Chi_{G_1}(x)\Chi_{G_2}(x)(1-{1\over x})$.}

%%%%%%% These are repeats
\iffalse
\item  What is the number of proper colorings of a cycle on five vertices
in 4 colors?

\item What are the possible chromatic polynomials for a tree on $n$
vertices?
\fi
  
\end{enumerate}