\chapter{Distribution Problems}
\section{The idea of a distribution}
Many of the problems we solved in Chapter 1 may be thought of as problems
of distributing objects (such as pieces of fruit or ping-pong balls) to
recipients (such as children).  Some of the ways of viewing counting
problems as distribution problems are somewhat indirect.  For example, in
Problem
\ref{ping-pong} you probably noticed that the number of ways to pass out
$k$ ping-pong balls to $n$ children so that no child gets more than one is
the number of ways that we may choose a
$k$-element subset of
an $n$-element set.  We think of the children as recipients and objects we
are distributing as the identical ping-pong balls, distributed so that
each recipient gets at most one ball.  Those children who receive an
object are in our set.  It is helpful to have more than one way to think
of solutions to problems.  In the case of distribution problems, another
popular model for distributions is to think of putting balls in boxes
rather than distributing objects to recipients.  Passing out identical
objects is modeled by putting identical balls into boxes.  Passing out
distinct objects is modeled by putting distinct balls into boxes.


\subsection{The twenty-fold way}
When we are passing out objects to recipients, we may think of the
objects as being either identical or distinct.  We may also think of the
recipients as being either identical (as in the case of putting fruit
into plastic bags in the grocery store) or distinct (as in the case of
passing fruit out to children).  We may restrict the distributions to
those that give at least one object to each recipient, or those that give
exactly one object to each recipient, or those that give at most one
object to each recipient, or we may have no such restrictions.  If the
objects are distinct, it may be that the order in which the objects are
received is relevant (think about putting books onto the shelves in a
bookcase) or that the order in which the objects are received is
irrelevant (think about dropping a handful of candy into a child's trick
or treat bag).  If we ignore the possibility that the order in which
objects are received matters, we have created
$2\cdot2\cdot4=16$ distribution problems.  In the cases where a recipient
can receive more than one distinct object, we also have four more
problems when the order objects are received matters.  Thus we have 20
possible distribution problems.  
\begin{table}[t]
\caption{An incomplete table of the number of ways to
distribute $k$ objects to $n$ recipients, with
restrictions on how the objects are
received}\label{firstdistributiontable}
\begin{center}
\fbox{\hglue-3.5pt
{\footnotesize\begin{tabular}{|l||c |c|}\hline
\multicolumn{3}{|c|}{The Twentyfold Way: A Table of Distribution
Problems}
\\ %\multicolumn{3}{|c|}{\hrulefil}
\hline\hline
$k$ objects and conditions&\multicolumn{2}{|c|}{ $n$ recipients and
mathematical model for distribution}\\
\cline{2-3}
on how they are received&Distinct&Identical\\
\hline\hline 1.  Distinct &$n^k$& 
?\\
no conditions&functions&set partitions ($\le n$ parts)\\ \hline
2.  Distinct&$n^{\underline{k}}$ &1 if
$k\le n$; 0 otherwise\\ Each gets at most one&\kern -2pt $k$-element
permutations\kern -2 pt&\\
\hline
3.  Distinct&?& ?
\\ Each gets at least one&onto functions& set partitions
($n$ parts)\\\hline 
4. Distinct&$k!=n!$&1 if $k=n$; 0 otherwise\\ Each
gets exactly one&bijections&\\
\hline
5.  Distinct, order matters&?& ?\\
&?& ?\\
\hline
6.  Distinct, order
matters&?&?\\
Each gets at least one&?&?\\
\hline
7.  Identical  &?&?\\
no conditions&?& ?\\ \hline
8.  Identical&$n\choose k$&1 if $k\le n$; 0 otherwise\\
Each gets at most one&	subsets&\\ \hline
9.  Identical&?&? \\
Each gets at least one&? &? \\
\hline 10.  Identical&1 if $k=n$; 0 otherwise&1 if $k=n$; 0 otherwise\\
Each gets exactly one&&\\
\hline
\end{tabular}\hglue-3pt}}\end{center}\vspace*{-12 pt}
\end{table}

We describe these problems in Table
\ref{firstdistributiontable}.  Since there are twenty possible
distribution problems, we call the table the ``Twentyfold
Way,"\index{Twentyfold Way} adapting terminology suggested by Joel Spencer
for a more restricted class of distribution problems. In the first column
of the table we state whether the objects are distinct (like people) or
identical (like ping-pong balls) and then give any conditions on how the
objects may be received.  The conditions we consider are whether each
recipient gets at most one object, whether each recipient gets at least
one object, whether each recipient gets exactly one object, and whether
the order in which the objects are received matters.  In the second
column we give the solution to the problem and the name of the
mathematical model for this kind of distribution problem when the
recipients are distinct, and in the third column we give the same
information when the recipients are identical.  We use question marks as
the answers to problems we have not yet solved and models we have not yet
studied.   We give explicit answers to problems we solved in Chapter 1
and problems whose answers are immediate.  The goal of this chapter is to
develop methods that will allow us to fill in the table with formulas or
at least quantities we know how to compute, and we will give a completed
table at the end of the chapter.  We will now justify the answers that
are not question marks and replace some question
marks with answers as we cover relevant material. 

If we pass out  $k$ distinct objects (say pieces of fruit) to $n$ distinct
recipients (say children), we are saying for each object which recipient
it goes to.  Thus we are defining a function from the set of objects to
the recipients.  We saw the following theorem in Problem
\ref{numberoffunctionsconjecture}. 
\begin{theorem} There are
$n^k$\index{functions!number of} functions from a
$k$-element set to an $n$-element set.
\end{theorem} We proved it in Problem
\ref{provenumberoffunctionsconjecture}.   If we pass out $k$ distinct
objects (say pieces of fruit) to $n$ indistinguishable recipients (say
identical paper bags) then we are dividing the objects up into 
disjoint sets; that is we are forming a partition of the objects into
some number, certainly no more than the number $k$ of objects, of parts. 
Later in this chapter (and again in  the next chapter) we shall discuss
how to compute the number of partitions of a
$k$-element set into
$n$ parts.  This explains the entries in row one of our table.

If we pass out $k$ distinct objects to $n$ recipients so that each gets
at most one, we still determine a function, but the function must be
one-to-one.  The number of one-to-one functions from a $k$-element set to
an $n$ element set is the same as the number of one-to-one functions
from the set $[k] =\{1,2,\ldots,k\}$ to an $n$-element set.  In Problem
\ref{kelementpermutation} we proved the following theorem.
 \begin{theorem} If $0\le k\le n$, then the number of
$k$-element permutations of an $n$-element set is 
$$n^{\underline{k}} = n(n-1)\cdots(n-k+1) =
n!/(n-k)!.$$\index{functions!one-to-one!number
of}\label{numberofinjections}
 \end{theorem}
 If $k>n$ there are no one-to-one functions from a $k$ element set to an
$n$ element, so we define $n^{\underline{k}}$ to be zero in this case. 
Notice that this is what the indicated product in the middle term of our
formula gives us. If we are supposed to distribute
$k$ distinct objects to
$n$ identical recipients so that each gets at most one, we cannot do so
if $k>n$, so there are 0 ways to do so.  On the other hand, if $k\le n$,
then it doesn't matter which recipient gets which object, so there is
only one way to do so.  This explains the entries in row two of our table.

If we distribute $k$ distinct objects to $n$ distinct recipients so that
each recipient gets at least one, then we are counting functions again,
but this time functions from a $k$-element set {\em onto} an $n$-element
set.  At present we do not know how to compute the number of such
functions, but we will discuss how to do so later in this chapter and in
the next chapter.  If we distribute $k$ identical objects to $n$
recipients, we are again simply partitioning the objects, but the
condition that each recipient gets at least one means that we are
partitioning the objects into exactly $n$ blocks.  Again, we will discuss
how compute the number of ways of partitioning a set of $k$ objects into
$n$ blocks later in this chapter.  This explains the entries in row three
of our table.

If we pass out $k$ distinct objects to $n$ recipients so that each gets
exactly one, then $k=n$ and the function that our distribution gives us
is a bijection.  The number of bijections from an $n$-element set to an
$n$-element set is $n!$ by Theorem \ref{numberofinjections}.  If we pass
out $k$ distinct objects of $n$ identical recipients so that each gets
exactly 1, then in this case it doesn't matter which recipient gets which
object, so the number of ways to do so is 1 if $k=n$.  If $k\not=n$, then
the number of such distributions is zero.  This explains the entries in
row four of our table.

We now jump to row eight of our table.  We saw  in Problem
\ref{ping-pong} that the number of ways to pass out $k$ identical
ping-pong balls to $n$ children is simply the number of $k$-element
subsets of an $n$-element set.  In Problem \ref{formulanchoosek} we proved
the following theorem.
\begin{theorem}  If $0\le k \le n$, the number of $k$-element subsets of
an
$n$-element set is given by $${n\choose k} = {n^{\underline{k}}\over k!}
= {n!\over k!(n-k)!}.$$
\end{theorem}
We define $n \choose k$ to be 0 if $k>n$, because then there are no
$k$-element subsets of an $n$-element set.  Notice that this is what the
middle term of the formula in the theorem gives us.  This explains the
entries of row 8 of our table.  For now we jump over row 9.

In row 10 of our table, if we are passing out $k$ identical objects to
$n$ recipients so that each gets exactly one, it doesn't matter whether
the recipients are identical or not; there is only one way to pass out
the objects if $k=n$ and otherwise it is impossible to make the
distribution, so there are no ways of distributing the objects.    This
explains the entries of row 10 of our table.  Several other rows of our
table can be computed using the methods of Chapter 1.

\subsection{Ordered functions}
\bp 
\itemei Suppose we wish to place $k$ distinct books onto the shelves of
a bookcase with $n$ shelves.  For simplicity, assume for now that all of
the books would fit on any of the shelves.  Also, let's imagine pushing
the books on a shelf as far to the left as we can, so that we are only
thinking about how the books sit relative to each other, not about the
exact places where we put the books. Since the books are distinct, we can
think of a the first book, the second book and so on.  How many places
are there where we can place the first book?  When we place the second
book, if we decide to place it on the shelf that already has a book, does
it matter if we place it to the left or right of the book that is already
there?  How many places are there where we can place the second book? 
Once we have $i-1$ books placed, if we want to place  book $i$
on a shelf that already has some books, is sliding it in to the left of
all the books already there different from placing it to the right of
all the books already or between two books already there?  In how many
ways may we place the
$i$th book into the bookcase? In how many ways may we place all the
books?\label{bookcase}
\solution{There are $n$ places where we can place the first book.  Once
we have placed it, there are $n+1$ places where we can place the second
book, because on the shelf that has one book, we could put the second
book to the left or to the right of the book already there.  Once we have
$i-1$ books on the shelves the $i$th book could go on any shelf to the
left of all books there, if any, giving us $n$ places, or it could go to
the immediate right of any book already there, giving us another $i-1$
places.  Thus there are $n+i-1$ places where we could place book $i$. 
From this, we can see that the number of ways to place all the books is
$$\prod_{i=1}^k (n+i-1).$$
}
\item Suppose we wish to place the books in Problem \ref{bookcase}
(satisfying the assumptions we made there) so that each shelf gets at
least one book.  Now in how many ways may we place the books? (Hint: how
can you make sure that each shelf gets at least one book before you start
the process described in Problem
\ref{bookcase}?)\label{bookcaseeveryshelf}
\solution{Choose $n$ books from the $k$ books in $k\choose n$ ways, and
assign them to the $n$ places shelves in $n!$ ways, giving us $k!/(k-n)!$
ways to put a book on each shelf.  Now leaving these books at the far
left of each shelf, place the remaining books in $$\prod_{i=1}^{k-n}
(n+i-1)={(n+(k-n)-1)!\over (n-1)!}={(k-1)!\over (n-1)!}$$ ways.  Thus we
have $${k!(k-1)!\over(k-n)!(n-1)!}=k!{k-1\choose n-1}$$
ways to place the books.  Of course the right hand side of that equation
cries out for a combinatorial explanation.  Here it is.  Imagine lining
up the $k$ books in a row.  Then there are $k-1$ places in between them. 
Choose $n-1$ of these places, and slide a piece of paper in there as a
divider.  Now put the books before the first divider on shelf one, and
the books after divider $i$ on shelf $i+1$.  This gives an arrangement of
the books on the shelves so that every shelf has a book!
 }


\ep
The assignment of which books go to which shelves of a bookcase is simply
a function from the books to the shelves.  But a function doesn't
determine which book sits to the left of which others on the shelf, and
this information is part of how the books are arranged on the shelves.  In
other words, the order in which the shelves receive their books matters. 
Our function must thus assign an ordered list of books to each shelf.  We
will call such a function an ordered function.  More precisely, an {\bf
ordered function}\index{ordered function}\index{function!ordered} from a
set $S$ to a set $T$ is a function that assigns an (ordered) list of
elements of
$S$ to some, but not necessarily all, elements of $T$ in such a way that
each element of $S$ appears on one and only one of the lists.\footnote{The phrase ordered function is not a
standard one, because there is as yet no standard name for the result of
an ordered distribution problem.}  (Notice that although it is not the
usual definition of a function from $S$ to $T$, a
function\index{function!alternate definition} can be described as an
assignment of  subsets of $S$ to some, but not necessarily all, elements
of $T$ so that each element of $S$ is in one and only one of these
subsets.) Thus the number of ways to place the books into the bookcase is
the entry in the middle column of row 5 of our table.  If in addition we
require each shelf to get at least one book, we are discussing the entry
in the middle column of row 6 of our table.  An {\em ordered onto
function}\index{function!ordered!onto}\index{ordered onto
function}\index{onto function!ordered} is one which assigns a list to
each element of $T$.  \label{orderedfunctionsection}

In Problem \ref{bookcase} you showed that the number of ordered functions
from a $k$-element set to an $n$-element set is $\displaystyle
\prod_{i=1}^n (n+i-1)$.  This product occurs frequently enough that it
has a name; it is called the $k$\/th {\em rising factorial
power}\index{factorial power!rising}\index{rising factorial power} of
$n$ and is denoted by $n^{\overline{k}}$.\index{$n^{\overline{k}}$}  It is
read as ``$n$ to the
$k$ rising." (This notation is due to Don Knuth, who also suggested the
notation for falling factorial powers.)\index{factorial}  We can summarize
with a theorem that adds two more
formulas for the number of ordered
functions.\index{factorial!falling}\index{falling factorial power}
\begin{theorem} The number of ordered functions from a $k$-element set to
an $n$-element set is
$$n^{\overline{k}}=\prod_{i=1}^n (n+i-1) = {(n+i-1)!\over (n-1)!} =
(n+k-1)^{\underline{k}}.$$\end{theorem}

\subsection{Broken permutations and Lah numbers}  
\bp  
\itemesi In how many ways may we stack $k$ distinct books into $n$
identical boxes so that there is a stack in every box? (Hint:  Imagine
taking a stack of
$k$ books, and breaking it up into stacks to put into the boxes in the
same order they were originally stacked.  If you are going to use $n$
boxes, in how many places will you have to break the stack up into smaller
stacks, and how many ways can you do this?)  (Alternate hint: How many
different bookcase arrangements correspond to the same way of stacking
$k$ books into $n$ boxes so that each box has at least one book?).  The
hints may suggest that you can do this problem in more than one way!
\label{brokenpermutation}
\solution{We can make a list of the $k$ distinct books in $k!$ ways. 
Then we have to choose $n-1$ of the $k-1$ places between the lists as the
places where we will break the list.  However the order in which we
list the boxes is irrelevant, so we have equivalence classes of $n!$
arrangements for each way of putting the books into boxes.  Thus we can
put  the books in boxes in
$k!{k-1\choose n-1}/n!$ ways.

Alternately, we can take the number of ways to put $k$ books onto $n$
bookshelves so that each shelf gets at least one, and then divide by the
number of shelves factorial.  That gives us $k!{k-1\choose n-1}/n!$ ways
to arrange the books.}
\ep

We can think of stacking books into identical boxes as partitioning the
books and then ordering the blocks of the partition.  This turns out not
to be a useful computational way of visualizing the problem because the
number of ways to order the books in the various stacks depends on the
sizes of the stacks and not just the number of stacks.  However this way
of thinking actually led to the first hint in Problem
\ref{brokenpermutation}.  Instead of dividing a set up into
nonoverlapping parts, we may think of dividing a {\em permutation}
(thought of as a list) of our $k$ objects up into $n$ ordered blocks.  We
will say that a set of ordered lists of elements of a set $S$ is a {\bf
broken permutation}\index{broken permutation}\index{permutation!broken}
of $S$ if each element of $S$ is in one and only one of these
lists.\footnote{The phrase broken permutation is not standard,
because there is no standard name for the solution to this kind  of
distribution problem.}   The number of broken permutations of a
$k$-element set with
$n$ blocks is denoted by $L(k,n)$.  The number $L(k,n)$ is called a
{\em Lah Number} and, from our solution to Problem
\ref{brokenpermutation}, is equal to $k!{k-1\choose n-1}/n!$.\index{Lah
number}

The Lah numbers are the solution to the question ``In how many ways may we
distribute $k$ distinct objects to $n$ identical recipients if order
matters and each recipient must get at least one?"  Thus they give the
entry in row 6 and column 6 of our table.  The entry in row 5 and column
6 of our table will be the number of broken permutations with less than
or equal to $n$ parts.  Thus it is a sum of Lah numbers.

We have seen that ordered functions and broken permutations explain the
entries in rows 5 and 6 of our table.
\subsection{Compositions of integers}
\bp 
\itemes In how many ways may we put $k$ identical books onto $n$ shelves
if each shelf must get at least one book?
\solution{In problem \ref{bookcaseeveryshelf} we showed that with $k$
distinct books we could place the books in $k!{k-1\choose n-1}$ ways.  
We can partition these arangements of distinct books into blocks, where
each block consists of all arrangements that we get just by permuting the
books among themselves.  Thus each block has $k!$ arrangements in it, and
each arrangement corresponds to an arrangement of identical books.  Thus
there are $k-1\choose n-1$ ways to arrange identical books.}

\itemes A {\bf composition} of the integer $k$ into $n$ parts is 
 a list of $n$ positive integers that add to $k$. 
How many compositions are there of an integer $k$ into $n$
parts?\label{compositionagian}
\solution{There is a bijection between compositions of $k$ into $n$ parts
and arrangements of $k$ identical books on $n$ shelves so that each shelf
gets a book.  Namely, the number of books on shelf $i$ is the $i$th
element of the list.  Thus the number of compositions of $k$ into $n$
parts is $k-1 \choose n-1$.}

\itemi Your answer in Problem \ref{compositionagian} can be expressed as a
binomial coefficient.  This means it should be possible to interpret a
composition as a subset of some set.  Find a bijection between
compositions of $k$ into $n$ parts and certain subsets of some set. 
Explain explicitly how to get the composition from the subset and the
subset from the composition.
\solution{If we line up $k$ identical books, there are $k-1$ places in
between two books.  If we choose $n-1$ of these places and slip dividers
into those places, then we have a first clump of books, a second clump of
books, and so on.  The $i$th element of our list is the number of books
in the $i$th clump.  Clearly using books is irrelevant; we could line up
any $k$ identical objects and make the same argument.  Our bijection is
between compositions and $n-1$-element subsets of the set of $k-1$ spaces
between our objects.}

\itemes Explain the connection between compositions of $k$ into $n$ parts
and the problem of distributing $k$ identical objects to $n$ recipients
so that each recipient gets at least one.  
\solution{Since the recipients are distinct, we can think of them as
a first recipient, a second, and so on.  Given a composition of
$k$ into
$n$ parts, let the
$i$th element of the list be the number of objects given to recipient
number
$i$.}
\ep
The sequence of problems you just completed should explain the entry in
the middle column of row 9 of our table of distribution problems.

\subsection{Multisets}
In the middle column of row 7 of our table, we are asking for the number
of ways to distribute $k$ identical objects (say ping-pong balls) to $n$
distinct recipients (say children).   
\bp
\iteme In how many ways may we distribute $k$ identical books on the
shelves of a bookcase with $n$ shelves, assuming that any shelf can hold
all the books?  \label{identicalbooks}
\solution{We saw that we could arrange $k$ distinct books on $n$ shelves
in $\prod_{i=1}^k (n+i-1)$ ways.  We partition these arrangements into
blocks by putting two arrangements  in the same block if we can get one
from the other by permuting the books among themselves.  Then the number
of blocks is the number of ways to place identical books on the shelves. 
However, there are $k!$ arrangements per block, so there are
$${\prod_{i=1}^k (n+i-1)\over k!}=(n+k-1)^{\underline{k}}= {n+k-1\choose
k}$$ ways to arrange identical books.}

\iteme A {\em multiset}\index{multiset} chosen from a set $S$ may be
thought of as a subset with repeated elements allowed.  For example the
multiset of letters of the word Mississippi is
$\{i,i,i,i,m,p,p,s,s,s,s\}$.  To determine a multiset we must say how
many times (including, perhaps, zero) each member of $S$ appears in the
multiset. The number of times an element appears is called its
{\em multiplicity}.\index{multiplicity in a multiset} The size of a
multiset chosen from
$S$ is the total number of times any member of $S$ appears.  For example,
the size of the multiset of letters of Mississippi is 11.  What is the
number of multisets of size
$k$ that can be chosen from an $n$-element set? \label{multiset}
\solution{There is a bijection between arrangements of identical books on
$n$ shelves and  multisets chosen from an $n$-element set: the
multiplicity of element $i$ is the number of books on shelf $i$.  Thus we
have $n+k-1\choose k$ ways to choose a $k$-element multiset from an
$n$-element set by Problem \ref{identicalbooks}.}

\itemi Your answer in the previous problem should be expressible as a
binomial coefficient.  Since a binomial coefficient counts subsets, find
a bijection between subsets of something and multisets chosen from a set
$S$.
\solution{Note that ${n+k-1\choose k} = {n+k-1 \choose n-1}$.  We will
show a bijection between ways of choosing $n-1$ things our of $n+k-1$
things and multisets.  Namely,  take $n+k-1$ objects and line them up in
a row.  Choose $n-1$ of them.  Now let the multiplicity of element 1 of
our multiset be the number of objects before the first thing we chose. 
if $1<i<n$, let the multiplicity of element $i$ of our multiset be the
number of things between the $i-1$th thing we chose and the $i$th thing
we choose.  Let the multiplicity of the $n$th element of our multiset be
the number of objects after the last one we chose.  Another way to say
the essentially same thing is to make a list of $n+k-1$ blank spaces.  We
choose $k$ of them in which we put ones and $n-1$ of them in which we put
plus signs.  Then the multiplicity of element 1 is the number of ones
before the first plus sign, the multiplicity of element $n$ is the number
of ones after the last plus sign and if $1<i<n$, the multiplicity of
element $i$ is the number of ones between the $(i-1)$th plus sign and the
$i$th plus sign.}

\item How many solutions are there in nonnegative integers to the
equation $x_1+x_2+ \cdots +x_m = r$, where $m$ and $r$ are constants?
\solution{We can think of $x_i$ as the multiplicity of element $i$ of a
multiset chosen from among $m$ things.  The total number of elements of
the multiset will be
$r$.  Thus we have $m+r-1\choose r$ solutions.
}
\ep 

A more precise definition of a {\bf multiset}\index{multiset} chosen from
a set
$S$ is that it is a function $m$, called a {\em multiplicity
function}\index{multiplicity function}, from
$S$ to the nonnegative integers.  For each $x$ in $S$, $m(x)$ specifies
how many times $x$ appears in the multiset.  In our example of the word
Mississippi above, our set $S$ can be taken to be the set of alphabet
letters and the multiplicity function $m$ is given by $m({\rm i})=4$,
$m({\rm m}) =1$, $m({\rm p})=2$, $m({\rm s}) =4$, and $m$ of any other
member of $S$ is 0.  When we list the members of a multiset in braces, it
will be clear from context that we are thinking of a multiset.  However
when we use braces in another way, it may not be clear what we mean.  For
example, when we write
$$\{x|x \mbox{~is a letter of Mississippi}\},$$
do we mean the set $\{i,m,p,s\}$ or the multiset
$\{i,i,i,i,m,p,p,s,s,s,s\}$?  To resolve this, whenever it is not clear
from context whether we are talking about a set or multiset we will use
the subscript multi on the right brace enclosing the multiset to
distinguish a multiset. Thus we write
$$\{x|x \mbox{\ is a letter of
Mississippi}\}_{\mbox{\scriptsize multi}}=\{i,i,i,i,m,p,p,s,s,s,s\}.$$
In this case it is probably clear from the right-hand side of the equation
that we are thinking of the left-hand side as a multiset, but we will
always try to err in the direction of clarity rather than brevity.

The
sequence of problems you just completed should explain the entry in the
middle column  of row 7 of our table of distribution problems.  In the
next two sections we will give ways of computing the remaining entries.


\section{Partitions and Stirling Numbers} We have seen how the number of
partitions of a set of $k$ objects into $n$ blocks corresponds to the
distribution of $k$ distinct objects to $n$ identical recipients.  While
there is a formula that we shall eventually learn for this number, it
requires more machinery than we now have available.  However there is a
good method for computing this number that is similar to Pascal's
equation.  Now that we have studied recurrences in one variable, we will
point out that Pascal's equation is in fact a {\em recurrence in two
variables}\index{recurrence!two
variable}; that is it lets us compute $n\choose k$ in terms of values of
$m\choose i$ in which either $m<n$ or $i<k$ or both.  It was the fact
that we had such a recurrence and knew $n\choose 0$ and $n\choose n$ that
let us create Pascal's triangle.
\subsection{Stirling Numbers of the second kind}
We use the notation $S(k,n)$ to stand for the number of partitions of a
$k$ element set with $n$ blocks.  For historical reasons, $S(k,n)$ is
called a {\em Stirling number of the second kind}. \index{partition!of a
set}\index{partition!of a set!!Stirling Numbers}\index{Stirling
Number!second kind}\index{$S(k,n)$}
\bp
\itemei In a partition of the set $[k]$, the number $k$ is either in a
block by itself, or it is not.  How does the number of partitions of
$[k]$ with $n$ parts in which $k$ is in a block with other elements of
$[k]$ compare to the number of partitions of $[k-1]$ into $n$ blocks? 
Find a two variable recurrence for $S(n,k)$, valid for  $n$ and
$k$ larger than one.\label{secondstirlingrecurrence}
\solution{The number of partitions of $[k]$ into $n$ parts in which $k$ is
in a block with other elements of $[k]$ is equal $n$ times the number
of partitions of $[k-1]$ into $n$ blocks, because $k$ could be in any
of the
$n$ parts, and since it is in a block with other elements of $[k-1]$,
removing it leaves a partition of $[k-1]$ into $n$ blocks.  The number of
partitions of $[k]$ into $n$ blocks in which $k$ is in a block by itself
is the number of partitions of $[k]$ into $n-1$ blocks, because you can
get any such partition in by deleting the block containing $k$ from a
partition of $[k]$ in which $k$ is in a block by itself.  Thus
$S(k,n))=S(k-1,n-1) +nS(k-1,n)$.}

\item What is $S(k,1)$?  What is $S(k,k)$?  Create a table of values of
$S(k,n)$ for $k$ between 1 and 5 and $n$ between 1 and $k$.  This table
is sometimes called {\em Stirling's Triangle (of the
second kind)}\index{Stirling's triangle!second kind}  How would you define
$S(k,n)$ for the nonnegative values of
$k$ and
$n$ that are not both positive?  Now for what values of $k$ and $n$ is
your two variable recurrence valid?
\solution{$S(k,1)=1$ and $S(k,k)=1$. 

\begin{tabular}{|c|c|c|c|c|c|c|c|}
$k\backslash n$&0&1&2&3&4&5&6\\
\hline
0&1&0&0&0&0&0&0\\
1&0&1&0&0&0&0&0\\
2&0&1&1&0&0&0&0\\
3&0&1&3&1&0&0&0\\
4&0&1&7&6&1&0&0\\
5&0&1&15&25&10&1&0\\
\end{tabular}

As you see in the table, we
define $S(0,0)=1$, and $S(0,n)$ or $S(k,0)$ to be 0 otherwise.  This
makes sense because for $n>0$ there is no partition of an empty set
into $n$ parts, and for $k>0$ there is no partition of a $k$-element
set into no parts, but saying there is one partition of the empty set
into no parts allows us to use our recurrence to compute $S(1,1)$. 
This makes our recurrence valid for all nonnegative values of $k$ and
$n$.
}

\item  Extend Stirling's triangle enough to allow you to answer the
following question and answer it.  (Don't fill in the rows all the way;
the work becomes quite tedious if you do.  Only fill in what you need to
answer this question.)  A caterer is preparing three bag lunches for
hikers.  The caterer has nine different sandwiches.  In how many ways can
these nine sandwiches be distributed into three identical lunch bags so
that each bag gets at least one?\label{sandwiches}
\solution{We need $S(9,3)$.  Thus we need to extend our table for four
more rows, but only out to the column labelled 3.  These rows are
6,0,1,31,90, 7,0,1,63,301, 7,0,1,127,966, 8,0,1,255,3025,
9,0,1,511,9330.  Thus there are 9330 ways to distribute the sandwiches
into the lunch bags.}

\item  The question in Problem \ref{sandwiches} naturally suggests a more
realistic question; in how many ways may the caterer distribute the nine
sandwich's into three identical bags so that each bag gets exactly
three?  Answer this question.  (Hint, what if the question asked about
six sandwiches and two distinct bags?  How does having identical bags
change the answer?)\label{caterer2}
\solution{${9\choose 3}{6\choose 3}{3\choose 3}/3!$.  First we choose
three sandwiches for bag 1, then three for bag 2, and put the remainder in
bag 3.  However, it doesn't matter which bags the sandwiches
are in so we have counted each partition $3!$ times.}

\iteme In how many ways can we partition $k$ items into $n$ blocks so that
we have $k_i$ blocks of size $i$ for each $i$?  (Notice that
$\sum_{i=1}^k k_i = n$ and $
\sum_{i=1}^k ik_i = k$.)\label{partitionsgivenpartsize}  The sequence
$k_1,k_2,\ldots,k_n$ is called the {\em type vector} of the
partition.\index{partition of a set! type vector}\index{type vector of a
partition of a set}
\solution{$n!\over \prod_{i=1}^n (i!)^{k_i}{k_i!}$.  We can make a
list in $n!$ ways, and then break it into first $k_1$ blocks of size 1,
then $k_2$ blocks of size 2, $k_3$ blocks of size 3 up to $k_n$ blocks
of size $n$.  But then we realize that we get the same partition if we
permute the $i!$ elements of a block of size $i$ and we get the same
partition if we permute the $k_i$ blocks of size $i$ so we apply the
quotient principle.}

\item Describe how to compute $S(n,k)$ in terms of quantities given by
the formula you found in Problem \ref{partitionsgivenpartsize}.
\solution{We can find $S(n,k)$ by summing $n!\over \prod_{i=1}^n
(i!)^{k_i}{k_i!}$ over all type vectors $(k_1,k_2,\ldots,k_n)$ such
that
$k_1+k_2+\cdots+k_n=k$.}

\itemi Find a recurrence similar to the one in Problem
\ref{secondstirlingrecurrence} for the Lah numbers $L(k,n)$.
\solution{$L(k,n)$ is the number of broken permutations of a
$k$-element set into $n$ parts.  Either $k$ is in an ordered block by
itself or it is not.  If it is, it can go after any of the $k-1$ other
elements, or it can go at the beginning of any of the $n$ blocks.  If
it is not, deleting it gives a broken permutation of a $k-1$-element
set into $n-1$ blocks.  Thus $L(k,n)=L(k-1,n-1) + (n+k-1)L(k,n)$.}

\itemes (Relevant in Appendex \ref{expogenfun}.) The total number of
partitions of a
$k$-element set is denoted by
$B(k)$ and is called the $k$-th {\em Bell number}\index{Bell
Number}\index{partitions of a set!number of}.  Thus
$B(1)=1$ and
$B(2) =2$.\label{BellNumberIntro}
\begin{enumerate}\item Show, by explicitly exhibiting the partitions, that
$B(3)=5$.
\solution{The five partitions of $[3]$ are the sets $\{\{1\},
\{2\},\{3\}\}$,
$\{\{1,2\},\{3\}\}$, $\{\{1,3\},\{2\}\}$, $\{\{1\},\{2,3\}\}$, and
$\{\{1,2,3\}\}$.}
\item Find a recurrence that expresses $B(k)$ in terms of $B(n)$ for
$n< k$ and prove your formula correct in as many ways as you can.
\solution{If we delete the block containing $k$, we get a partition of
a subset of $[k-1]$. Thus $B(k)$ is the sum over all subsets of
$[k-1]$ of the number of partitions of that subset.  This gives us
$B(k)= \sum_{n=0}^{k-1}{k-1\choose n}B(n)$.

Alternatively, we can show by the same sort of argument that
$S(k,n)=\sum_{i=0}^{k-1} {k-1\choose i}S(i,n-1)$ and then use the fact
that $B(k)=\sum_{n=0}^k S(k,n)$ to get the recurrence for $B(k)$.}

\item Find $B(k)$ for $k=4,5,6$.
\solution{$$B(4) ={3\choose 0}B_0 +{3\choose1}B_1 +{3\choose2}B_2 +
{3\choose3}B_3=1 +3+3\cdot2 +5=15$$
$$B(5) = \sum_{n=0}^4 {4\choose n}B_n = 1 +4+6\cdot2 +4\cdot5 + 15=52$$
$$B(6) = \sum_{n=0}^5 {5\choose n}B_n =1+5 +10\cdot2 +10\cdot 5
+5\cdot 15 +52=203$$}
\end{enumerate}
\ep

\subsection{Stirling Numbers and onto functions}
\bp 

\itemm Given a function $f$ from a $k$-element set $K$ to an $n$-element
set, we can define a partition of $K$ by putting $x$ and $y$ in the same
block of the partition if and only if $f(x)=f(y)$.  How many blocks does
the partition have if $f$ is onto?  How is the number of functions from a
$k$-element set onto an $n$-element set related to a Stirling number?   Be
as precise in your answer as you can.\index{function!onto!and Stirling
Numbers}\index{onto function!counting}
\solution{If $f$ is onto, the number of blocks of the partition is
$n$.  The number of onto functions from a $k$-element set onto an
$n$-element set is $S(k,n)n!$, because we have a one-to-one function
from the blocks to the $n$-element set.}
\iteme Each function from a $k$-element set $K$ to an $n$-element set $N$
is a function from $K$ onto {\em some} subset of $N$.  If $J$ is a subset
of $N$ of size $j$, you know how to compute the number of functions that
map onto $J$ in terms of Stirling numbers.  Suppose you add the number of
functions mapping onto $J$ over all possible subsets $J$ of $N$.  What
simple value should this sum equal?  Write  the equation this gives
you.\label{Stirlingfalling}
\solution{The sum should equal the number of functions, $n^k$.  Thus
we get $\sum_{j=0}^n {n\choose j}S(k,j)j! = n^k$.  By using the fact
that ${n\choose j}= n^{\underline{j}}/j!$, this may be rewritten as
$\sum_{j=0}^n n^{\underline{j}}S(k,j) = n^k.$}
\itemm In how many ways can the sandwiches of Problem \ref{sandwiches} be
placed into three distinct bags so that each bag gets at least one?
\solution{$S(9,3)\cdot3!= 55,980$.}
\itemm In how many ways can the sandwiches of Problem \ref{caterer2} be
placed into distinct bags so that each bag gets exactly three?
\solution{Choose three sandwiches for bag one in $9\choose3$ ways,
three for bag two in $6\choose3$ ways and put the remainder in bag 3. 
This gives us ${9\choose3}{6\choose3}={9!\over3!3!3!}=1680$ ways.  

The $9!\over3!3!3!$ suggests another solution.  We can line up the
sandwiches in 9! ways.  We take the first three for bag one, the
second three for bag two and the last three for bag 3.  The order of
the sandwiches in the bag does not matter though, so each there are
$3!3!3!$ listings corresponding to each way of putting sandwiches in
bags, giving us $9!\over3!3!3!$ ways to put the sandwiches in bags.}

\iffalse
\item  If $f:X\rightarrow Y$, we say that $y$ is the {\em
image}\index{image of an element under a function} of $x$ under $f$ if
$f(x)=y$. How many functions are there from a set
$K$ with
$k$ elements to a set $N$ with $n$ elements so that for each $i$ from 1
to
$n$, $k_i$ elements of
$N$ are each the images of $i$ different elements of $K$?  (Said
differently, we have $k_1$ elements of $N$ that are images of one
element of $K$, we have $k_2$ elements of $N$ that are images of two
elements of $K$, and in general $k_i$ elements of $N$ that are images
of $i$ elements of $K$.)  (We say
$y$ is the {\em image}\index{image (under a function} of $x$ if
$y=f(x)$.) 
\solution{We can think of this as using the elements of $K$ to label
the elements of $N$ so that $k_1$ elements of $K$ label one element
each of $N$, $k_2$ elements of $K$ label two elements each of $N$, and
in general, $k_i$ elements of $K$ label $i$ elements of $N$.  We can
think of lining up the elements of $N$ in $n!$ ways, and taking the
first $k_1$ of them to be labelled by the 
\fi
 
\iteme How many functions are there from a $k$-element set $K$ to a set
$N=\{y_1,y_2,\ldots y_n\}$ so that $y_i$ is the image of $j_i$ elements of
$K$ for each $i$ from 1 to $n$?  This number is called a {\em multinomial
coefficient}\index{multinomial
coefficient}\index{coefficient!multinomial} and denoted by $${k\choose
j_1,j_2,\ldots, j_n}.$$ 
\solution{${k\choose j_1,j_2,\ldots, j_n} =
{k!\over j_1!j_2!\cdots j_n!}$.  We get this either as the
product of binomial coefficients $${k\choose j_1}{k-j_1\choose
j_2}{k-j_1-j_2\choose j_3}\cdots{j_n\choose j_n},$$ or more elegantly,
by lining up the elements of the domain in $k!$ ways, taking the first
$j_1$ elements to $y_1$, the next $j_2$ elements to $y_2$ and so on. 
However the order of the $j_i$ elements that go to $y_i$ is
irrelevant, so $j_1!j_2!\cdots j_n!$ lists all correspond to the same
function, giving us $k!\over j_1!j_2!,\cdots j_n!$ functions.}

\item Explain how to compute the number of functions from a $k$-element
set $K$ onto an $n$-element set $N$ by using multinomial coefficients.
\solution{Add the multinomial coefficients $k\choose
j_1,j_2,\ldots,j_n$ in which each $j_i$ is different from zero.  To
see why, let $N=\{y_1,y_2,\ldots,y_n\}$ and note that we are counting
functions that send at least one element of $K$ to each element $y_i$.}

\iteme What do multinomial coefficients have to do with expanding the
$k$th power of a multinomial $x_1+x_2+\cdots+x_n$?  This result is called
the {\em multinomial theorem}.
\solution{When we use the distributive law to multiply out
$(x_1+x_2+\cdots +x_n)^k$, we will get a sum of a bunch of terms of the
form
$x_1^{i_1}x_2^{i_2}\cdots x_n^{i_n}$ where $i_1+i_2+\cdots+ i_n=k$. 
The terms with a given sequence $i_1,i_2,\ldots, i_n$ of
exponents will arise from choosing, as we apply the distributive law
over and over again, 
$x_1$ from
$i_1$ of the factors,
$x_2$ from $i_2$ of the factors, and so on. Thus the number of terms
$x_1^{i_1}x_2^{i_2}\cdots x_n^{i_n}$ will be the number of ways to
label $i_1$ of the factors with a 1, $i_2$ of the factors with a 2,
\ldots, and $i_n$ of the factors with an $n$.  The number of ways to do
this is a multinomial coefficient, as we now explain.    This labeling
gives us a function from
$[k]$ to $[n]$ as follows.  If factor $i$ is labelled $j$ we let $f(i)
=j$.  Further each function $f$ from $[k]$ to $[n]$ gives us that maps
$i_j$ elements of $[k]$ to $j$ will give us such a labelling.  Thus the
coefficient of
$x_1^{i_1}x_2^{i_2}\cdots x_n^{i_n}$ will be the multinomial
coefficient $k\choose i_1,i_2,\ldots, i_n$.}

\ep
\subsection{Stirling Numbers and bases for polynomials}
\bp 
\itemes Find a way to express $n^k$ in terms of 
$k^{\underline{j}}$ for appropriate values $j$.  You may use Stirling
numbers if they help you.  Notice that
$x^{\underline{j}}$ makes sense for a numerical variable $x$ (that could
range over the rational numbers, the real numbers, or even the complex
numbers instead of only the nonnegative integers, as we are implicitly
assuming $n$ does), just as
$x^j$ does.  Find a way to express the power $x^k$ in terms of the
polynomials $x^{\underline{j}}$ for appropriate values of $j$ and explain 
why your formula is correct.\label{powersfromfalling}
\solution{In Problem \ref{Stirlingfalling}, we saw that $\sum_{j=0}^n
{n\choose j}S(k,j)j! = n^k$.  Using the relationship between binomial
coefficients and falling factorials, we may rewrite this as 
$\sum_{j=0}^n n^{\underline{j}}S(k,j) = n^k$.  This expresses $n^k$
in terms of $k^{\underline{j}}$.  To be precise, we define
$x^{\underline{j}}$ to be $ x(x-1)\cdots (x-j+1)$. At first glance it
looks like we could express $x^{\underline{j}}$ in terms of powers of
$x$ by simply substituting $x$ for $n$   in the equation $\sum_{j=0}^n
n^{\underline{j}}S(k,j) = n^k$.  However this gives us $\sum_{j=0}^x
x^{\underline{j}}S(k,j) = x^k$, and we have never defined what we mean
by a sum whose upper limit is a variable $x$.  Thus we need to examine
the equation $\sum_{j=0}^n n^{\underline{j}}S(k,j) = n^k$ to see if we
can replace the $n$ that is the upper limit of the sum with something
else.    Notice that
$S(k,j)=0$ when $j>k$.  This means that if $k\le j$, then $\sum_{j=0}^n
n^{\underline{j}}S(k,j)=\sum_{j=0}^k n^{\underline{j}}S(k,j)$.
Notice also that $n^{\underline{j}}= n(n-1)\cdots(n-j+1)$ is zero when
$j>n$ because one of its factors is zero then.  This implies that if
$k>j$, then $\sum_{j=0}^n
n^{\underline{j}}S(k,j)=\sum_{j=0}^k n^{\underline{j}}S(k,j)$. 
Therefore, regardless of the relative size of $k$ and $n$,  we have
that
$\sum_{j=0}^n n^{\underline{j}}S(k,j)=\sum_{j=0}^k
n^{\underline{j}}S(k,j)$.  Therefore 
\begin{equation}
\sum_{j=0}^k
n^{\underline{j}}S(k,j)=n^k.\label{changeupperlimit}
\end{equation}
  It
makes sense to write the polynomial
$\sum_{j=0}^k
x^{\underline{j}}S(k,j)$; this is simply a polynomial of degree $k$ in
the variable $x$.  The expression $\sum_{j=0}^k
x^{\underline{j}}S(k,j)-x^k$ is also a polynomial in $x$, but it
might not be of degree $k$ since we are subtracting a degree $k$ term
from a degree $k$ polynomial.  In fact for every positive integer
value $n$ of $x$, this polynomial is zero.  That is, $\sum_{j=0}^k
n^{\underline{j}}S(k,j)-n^k=0$, which is just a restatement of
Equation \ref{changeupperlimit}.  But it is a fact of algebra that the
number of solutions of a nontrivial polynomial equation is no more
than the degree of the polynomial. Since the polynomial equation 
$\sum_{j=0}^k x^{\underline{j}}S(k,j)-x^k$ has infinitely many
different solutions, it must be a trivial equation; that is
$\sum_{j=0}^k
x^{\underline{j}}S(k,j)-x^k$ must be zero for every real (and even
every complex) number $x$.  Thus $\sum_{j=0}^k
x^{\underline{j}}S(k,j)=x^k$, and we have expressed $x^k$ in terms of
$x^{\underline{j}}$ for $j\le k$.
}

\ep
You showed in Problem \ref{powersfromfalling} how to
get each power of $x$ in terms of the falling factorial powers
$x^{\underline{j}}$.  Therefore every polynomial in $x$ is expressible in
terms of a sum of numerical multiples of falling factorial powers.  Using
the language of linear algebra, we say that the ordinary powers of $x$ and
the falling factorial powers of
$x$ each form a  basis for the
``space'' of polynomials, and that the numbers
$S(k,n)$ are ``change of basis coefficients.'' If you are not familiar
with linear algebra, a {\em basis}\index{basis (for polynomials)} for the
{\em space of polynomials}\footnote{The space of polynomials is just
another name for the set of all polynomials.\index{space of polynomials}}
is a set of polynomials such that each polynomial, whether in that set or
not, can be expressed in one and only one way as a sum of numerical
multiples of polynomials in the set.

\bp 
\itemm Show that every power of $x+1$ is expressible as a sum of
numerical multiples of powers of $x$.  Now show that every power of $x$
(and thus every polynomial in
$x$) is a sum of numerical multiples (some of which could be negative) of
powers of
$x+1$.  This means that the powers of
$x+1$ are a basis for the space of polynomials as well.  Describe the
change of basis coefficients that we use to express the binomial powers
$(x+1)^n$ in terms of the ordinary $x^j$ explicitly.  Find the change of
basis coefficients we use to express the ordinary powers $x^n$ in terms
of the binomial powers $(x+1)^k$.
\solution{We know that 
\begin{equation}
(x+1)^n=\sum_{i=0}^n {n\choose
i}x^{i}\label{usingbinomialtheorem}
\end{equation}
 from the binomial
theorem. (In the way we stated the binomial theorem, instead of
${n\choose i}x^i$ we would have gotten
${n\choose i}x^{n-i}$. There are two ways to fix this.  One is to
observe that the coefficient of
$x^i$ in that expansion is $n\choose n-i$, which equals $n\choose
i$.  The other is to observe that when we expand $(1+x)^n$ according
to the binomial theorem we get exactly what we wrote on the right hand
side in Equation
\ref{usingbinomialtheorem}.)  Therefore every power of $x+1$ is
expressible in terms of powers of $x$.

How do we express powers of $x$ in terms of powers of $x+1$?  Some
experimentation would help us guess how to do so; however there is a
really nice trick that also isn't hard to see.  Namely, we can write
\begin{eqnarray*}
x^n =(x+1-1)^n= [(x+1) -1]^n&=& \sum_{i=0}^n {n\choose
i}(x+1)^{n-i}(-1)^i\\ &=&
\sum_{i=0}^n {n\choose i}(x+1)^i(-1)^{n-i}%\label{x+1tox}
\end{eqnarray*}
This means that every power of $x$ is expressible in terms of powers
of $x+1$ and the change of basis coefficients to express powers of $x$
in terms of powers of $x+1$ are $(-1)^{n-i}{n\choose i}$ while the
change of basis coefficients used to express powers of $x+1$ in terms
of powers of $x$ are $n\choose i$.
}

\itemesi By multiplication, we can see that every falling factorial
polynomial can be expressed as a sum of numerical multiples of powers of
$x$.  In symbols, this means that there are numbers $s(k,n)$ (notice that
this $s$ is lower case, not upper case) such that we may write
$x^{\underline{k}} =
\sum_{n=0}^k s(k,n)x^n$.  These numbers $s(k,n)$\index{$s(k,n)$} are
called Stirling Numbers of the first kind.\index{Stirling Number!first
kind}  By thinking algebraically about what the formula 
 \begin{equation} 
x^{\underline{k}} =
x^{\underline{k-1}}(x-k+1)\label{stirling1}
 \end{equation} 
means, we can
find a recurrence for Stirling numbers of the first kind that gives us
another triangular array of numbers called Stirling's triangle of the
first kind.\index{Stirling's triangle!first kind}  Explain why Equation
\ref{stirling1} is true and use it to derive a recurrence for $s(k,n)$ in
terms of $s(k-1,n-1)$ and $s(k-1,n)$.\label{Stirlingfirst}
\solution{Equation \ref{stirling1} is effectively the inductive step
of an inductive definition of $x^{\underline{k}}$.  With this
equation we can write
\begin{eqnarray*} \sum_{n=0}^k s(k,n)x^n&=&x^{\underline{k}} =
x^{\underline{k-1}}(x-k+1)\\
&=&\left(\sum_{n=0}^{k-1}s(k-1,n)x^n\right)(x-k+1)\\
&=&\sum_{n=0}^{k-1}s(k-1,n)x^{n+1} - \sum_{n=0}^{k-1}(k-1)s(k-1,n)x^n\\
&=&\sum_{n=1}^k s(k-1,n-1)x^n-\sum_{n=0}^{k-1}(k-1)s(k-1,n)x^n.
\end{eqnarray*}
Equating the coefficients of $x^n$ in the first and last line of this
equation, we get
$s(k,n) = s(k-1,n-1) -(k-1)s(k-1,n)$, for $n$ between 1 and $k-1$..}
\item Write down the  rows of Stirling's triangle of the first kind
 for $k=0$ to~6.
\solution{
\begin{tabular}{|c|c|c|c|c|c|c|c|}
$k\backslash n$&0&1&2&3&4&5&6\\
\hline
0&1&0&0&0&0&0&0\\
1&0&1&0&0&0&0&0\\
2&0&-1&1&0&0&0&0\\
3&0&2&-3&1&0&0&0\\
4&0&-6&11&-6&1&0&0\\
5&0&24&-50&35&-10&1&0\\
6&0&-120&274&-225&85&-15&1
\end{tabular}
}
\ep

By definition, the Stirling numbers of the first kind are also change of
basis coefficients.
The Stirling numbers of the first and second kind are change of basis
coefficients from the falling factorial powers of $x$ to the ordinary
factorial powers, and vice versa. 
\bp 
\itemi Explain why every rising factorial polynomial $x^{\overline{k}}$
can be expressed in terms of the falling factorial polynomials
$x^{\underline{n}}$.  Let $b(k,n)$ stand for the change of basis
coefficients that allow us to express 
 $x^{\overline{k}}$  in terms of the falling factorial polynomials
$x^{\underline{n}}$; that is, define $b(k,n)$ by the equations
$$x^{\overline{k}}=\sum_{n=0}^k b(k,n) x^{\underline{n}}.$$ 
\begin{enumerate}
\item Find a recurrence
for $b(k,n)$. 
\solution{
\begin{eqnarray*}
&&\sum_{n=0}^k b(k,n) x^{\underline{n}} = x^{\overline{k}} =
x^{\overline{k-1}}(x+k-1)\\
&=&\left(\sum_{n=0}^{k-1} b(k-1,n) x^{\underline{n}}\right)(x+k-1)\\
&=&\sum_{n=0}^{k-1} b(k-1,n) x^{\underline{n}}(x+k-1)\\
&=&\sum_{n=0}^{k-1} b(k-1,n) x^{\underline{n}}(x-n+n+k-1)\\
&=&\sum_{n=0}^{k-1} b(k-1,n) x^{\underline{n+1}} +(n+k-1)b(k-1,n)
x^{\underline{n}}\\
&=&\sum_{n=1}^k b(k-1,n-1)x^{\underline{n}} +\sum_{n=0}^{k-1}
(n+k-1)b(k-1,n)x^{\underline{n}}
\end{eqnarray*}
Thus if $n$ is not 0 or $k$, we equate the coefficient of
$x^{\underline{n}}$ in the first line and last line to get $$b(k,n) =
b(k-1,n-1) + (n+k-1)b(k-1,n).$$  The trick of subtracting $n$ and
adding $n$ in the middle of the computation was the result of wanting
to mimic the way in which we increased the power on $x$ in the
solution to Problem \ref{Stirlingfirst}.}
\item Find a formula for $b(k,n)$ and prove the correctness of what you
say in as many ways as you can.
\solution{We will answer the next part of the problem here!  The
recurrence for $b(k,n)$ is exactly the same as the recurrence for
$L(k,n)$.  Further, $b(0,0)=1=L(0,0)$, $b(0,n) = 0=L(0,n)$ for $n>0$,
and
$b(k,k)=L(k,k) =1$.  Thus $b(k,n)$ and $L(k,n)$ are identical.  This and
the formula from Problem \ref{brokenpermutation} gives one proof that
$b(k,n)=k!{k-1\choose n-1}/n!$.

A second proof that the change of basis coefficients are Lah numbers
goes as follows.  $n^{\overline{k}}$ counts the number of ordered
functions from a $k$-element set to an $n$-element set.  One way to
determine such an ordered function is to take a broken permutation of
the $k$-element set into $n$ or fewer parts, and then take a
one-to-one function from the parts to the $n$-element set.  More
informally we assign the parts of the broken permutation to distinct
elements of the $n$-element set.  If the broken permutation has $i$
parts, the number of ways to do this assignment is the number of
$i$-element permutations of an
$n$-element set, $n^{\underline{i}}$.  Thus
$n^{\overline{k}}=\sum_{i=0}^n L(k,i)n^{\underline{i}}$.  However we
can change the upper limit of the sum to $k$ because
$L(k,i)$ is zero when $i>k$ and $n^{\underline{i}}$ is zero when $i>n$. 
Now we change
$n$ to
$x$ because we have a polynomial equality which is valid for infinitely
many of the values of the variable.  This gives us
$x^{\overline{k}}=\sum_{i=0}^k L(k,i)x^{\underline{i}}$.  Thus $b(k,i) =
L(k,i)$.}
\item Is $b(k,n)$ the same as any of the other families of numbers
(binomial coefficients, Bell numbers, Stirling numbers, Lah numbers,
etc.) we have studied?
\solution{As we said in our solution to the previous part, $b(k,n)$ is
the Lah number $L(k,n)$.}
\item Say as much as you can (but say it precisely) about the change of
basis coefficients for expressing $x^{\underline{k}}$ in terms of
$x^{\overline{n}}$.
\solution{There are several ways of finding this relationship, but the
most concise way is to observe that
$(-x)^{\underline{k}}=(-1)^kx^{\overline{k}}$ and
$(-x)^{\overline{k}}= (-1)^k x^{\underline{k}}$.  This lets us write
\begin{eqnarray*}(-x)^{\overline{k}}&=&\sum_{n=0}^k
b(k,n)(-x)^{\underline{n}}\\
(-1)^kx^{\underline{k}}&=&\sum_{n=0}^k(-1)^nb(k,n)x^{\overline{n}}\\
x^{\underline{k}}&=&\sum_{n=0}^k (-1)^{n-k}b(k,n)x^{\overline{n}}. 
\end{eqnarray*}
Therefore the change of basis coefficients are $(-1)^{n-k}b(k,n)$.}

\end{enumerate}
\ep
\section{Partitions of Integers}  We have now completed all our
distribution problems except for those in which both the objects and the
recipients are identical.  For example, we might be putting identical
apples into identical paper bags.  In this case all that matters is how
many bags get one apple (how many recipients get one object), how many
get two, how many get three, and so on.  Thus for each bag we have a
number, and the multiset of numbers of apples in the various bags is
what determines our distribution of apples into identical bags.  A
multiset of positive integers that add to $n$ is called a {\bf
partition}\index{partition of an integer} of
$n$.  Thus the partitions of 3 are 1+1+1, 1+2 (which is the same as 2+1)
and 3.  The number of partitions of $k$ is denoted by $P(k)$; in
computing the partitions of 3 we showed that $P(3) = 3$.  It is
traditional to use Greek letters like $\lambda$ (the Greek letter
$\lambda$ is pronounced LAMB duh)  to stand for partitons;
we might write $\lambda= 1,1,1$, $\gamma= 2,1$ and $\tau = 3$
to stand for the three partitions we just described.  We also write
$\lambda = 1^3$ as a shorthand for $\lambda = 1,1,1$, and we write
$\lambda \dashv 3$ as a shorthand for ``$\lambda$ is a partition of
three."

\bp
\itemm Find all partitions of 4 and find all partitions of 5, thereby
computing $P(4)$ and $P(5)$.
\solution{$4=1+1+1+1$, $4=2+1+1$, $4=2+1$, $4=3+1$, $4=4$, so that
$P(4)=5$.  $5=1+1+1+1+1$, $5=2+1+1+1$, $5=2+2+1$, $5=3+1+1$, $5=3+2$,
$5=4+1$, $5=5$, so that $P(5)=7$.}
\ep

\subsection{The number of partitions of $k$ into $n$ parts}
 A {\em partition of the integer $k$ into $n$ parts}\index{partition
of an integer!into $n$ parts} is a multiset of
$n$ positive integers that add to $k$.  We use $P(k,n)$ to denote the
number of partitions of $k$ into $n$ parts.  Thus $P(k,n)$ is the number
of ways to distribute $k$ identical objects to $n$ identical recipients
so that each gets at least one.  
\bp
\itemm Find
$P(6,3)$ by finding all partitions of 6 into 3 parts.  What does this say
about the number of ways to put six identical apples into three identical
bags so that each bag has at least one apple? 
\solution{$6=4+1+1$, $6=3+2+1$, $6=2+2+2$, so $P(6,3)=3$.  This says
there are three ways to put six identical apples into three identical
bags so that each bag gets at least one apple.}
\ep

\subsection{Representations of partitions}
\bp
\itemm  How many solutions are there in the positive integers to the
equation $x_1+x_2+x_3 =7$ with $x_1\ge x_2\ge x_3$?
\solution{This problem is asking for $P(7,3)$ and suggests an
organized way to go about finding it: list the partitions starting
with the largest part and work down.  $7=5+1+1$, $7=4+2+1$,
$7=3+3+1$, $7=3+2+2$, and if we have three numbers that add to seven,
one must be larger than two, so there are four such solutions.}
\item Explain the relationship between partitions of $k$ into $n$ parts
and lists $x_1,x_2$,\ldots, $x_n$ of positive integers with
$x_1\ge x_2\ge\ldots
\ge x_n$.  Such a representation of a partition is called a {\em
decreasing list}\index{partition of an integer!decreasing list}
representation of the partition. 
\solution{There is a bijection between partitions of $k$ into $n$
parts and lists, in nonincreasing order, of $n$ positive integers that
add to $k$, because each multiset of numbers that adds to $k$ can be
listed in nonincreasing order in exactly one way.}

\itemm Describe the relationship between partitions of $k$ and lists or
vectors $(x_1,x_2,\ldots,x_n)$ such that $x_1+2x_2+\ldots kx_k = k$.
Such a representation of a partition is called a {\em type vector}
representation of a partition, and it is typical to leave the trailing
zeros out of such a representation; for example $(2,1)$ stands for the
same partition as $(2,1,0,0)$.  What is the decreasing list
representation for this partition, and what number does it
partition?\index{partition of an integer!type vector}\index{type vector
for a partition of an integer}
\solution{The type vector of a partition of $k$ is a way of
representing the multiplicity function of the multiset of integers
that adds to $k$.  Thus there is a bijection between type vectors and
partitions.}

\item How does the number of partitions of $k$ relate to the number of
partitions of $k+1$ whose smallest part is one?
\solution{They are equal, because if we take two different partitions
of $k-1$ and increase the multiplicity of 1 in each (by one), they are
still different; also if we take two different partitions of $k$ that
have parts of size one, and decrease the multiplicity of 1 in each (by
one), they are still different.}
\ep

When we write a partition as
$\lambda =
\lambda_1,\lambda_2,\ldots,\lambda_n$, it is customary to write the list of
$\lambda_i$s as a decreasing list.  When we have a type vector
$(t_1,t_2,\ldots,t_m)$ for a partition, we write either $\lambda =
1^{t_1}2^{t_2}\cdots m^{t_m}$ or $\lambda =  m^{t_m}(m-1)^{t_{m-1}}\cdots
2^{t_2}1^{t_1}$.  Henceforth we will use the second of these.  When we
write $\lambda=\lambda_1^{i_1}\lambda_2^{i_2}\cdots\lambda_n^{i_n}$, we
will assume that $\lambda_i>\lambda_i+1$.



\subsection{Ferrers and Young Diagrams and the conjugate of a partition}
The decreasing list representation of partitions leads us to a handy way
to visualize partitions.  Given a decreasing list
$(\lambda_1,\lambda_2,\ldots \lambda_n)$, we draw a figure made up of
rows of dots that has $\lambda_1$ equally spaced dots in the first row,
$\lambda_2$ equally spaced dots in the second row, starting out right
below the beginning of the first row and so on.  Equivalently, instead of
dots, we may use identical squares, drawn so that a square touches each
one to its immediate right or immediately below it along an edge.  See
Figure
\ref{FerrersYoung} for examples.\index{Ferrers diagram}\index{Young
diagram}\index{partition of an integer!Ferrers diagram}\index{partition
of an integer!Young diagram}\index{diagram!of a
partition!Ferrers}\index{diagram!of a partition!Young} The figure we draw
with dots is called the Ferrers diagram of the partition; sometimes the
figure with squares is also called a Ferrers diagram; sometimes it is
called a Young diagram.  At this stage it is irrelevant which name we
choose and which kind of figure we draw; in more advanced work the squares
are handy because we can put things like numbers or variables into them. 
From now on we will use squares and call the diagrams Young diagrams.

\begin{figure}[htb]\caption{The Ferrers
and Young diagrams of the partition
(5,3,3,2)}\label{FerrersYoung}
\begin{center}\psfig{figure=FerrersYoung.eps%,height=1.0in
} \end{center}
\end{figure}
\bp
\iteme Draw the Young diagram of the partition (4,4,3,1,1).  Describe the
geometric relationship between the Young diagram of (5,3,3,2) and the
Young diagram of (4,4,3,1,1).
\solution{\begin{center}\vspace*{-24 pt} 
\psfig{figure=Young44311.eps}
\end{center}We get the Young diagram of $(5,3,3,2)$ by flipping the
Young diagram of $(4,4,3,1,1)$ around a line that includes the diagonal
of the upper left box; if we think of the top left corner of the
diagram as being at the origin, we flip around the line $y=-x$.}
\iteme The partition $(\lambda_1,\lambda_2,\ldots, \lambda_n)$ is called
the {\em conjugate}\index{conjugate of an integer
partition}\index{partition of an integer!conjugate of} of the partition
$(\gamma_1,\gamma_2,\ldots, \gamma_m)$ if we obtain the Young diagram of
one from the Young diagram of the other by flipping one around the line
with slope -1 that extends the diagonal of the top left square.  See Figure
\ref{conjugateYoung} for an example.
\begin{figure}[htb]\caption{The Ferrers
diagram the partition
(5,3,3,2) and its conjugate.}\label{conjugateYoung}
\begin{center}\psfig{figure=conjugateYoung.eps%,height=1.0in
} \end{center}
\end{figure}  
What is the conjugate of
(4,4,3,1,1)?  How is the largest part of a partition related to the
number of parts of its conjugate?  What does this tell you about the
number of partitions of a positive integer $k$ with largest part $m$?
\solution{$(5,3,3,2)$.  The largest part of a partition equals the
number of parts of its conjugate.  The number of partitions of $k$
with largest part $m$ equals the number of partitions of $k$ with $m$
parts.}

\itemi A partition is called {\em self-conjugate}\index{self-conjugate
partition}\index{partition of an integer!self conjugate} if it is equal
to its conjugate.  Find a relationship between the number of
self-conjugate partitions of $k$ and the number of partitions of $k$ into
distinct odd parts.
\solution{The number of self-conjugate partitons of $k$ equals the
number of partitions of $k$ with distinct odd parts.  Here is a
geometric description of a bijection from self conjugate partitions of
$k$ to partitions into distinct odd parts. 
\begin{center} 
\psfig{figure=selfconjugate.eps}
\end{center}
 Take the top row and left
column of squares of the Young diagram, and make them into one row in  a
new diagram.  (Only include the square that is in both the row and column
once.) Now take the remaining squares in the next row and column and make
a new row of the Young diagram of the second partition with them.
Continue this process with succeeding rows and columns, not using any
squares you have already used. Because the first partition is self
conjugate, the diagram has the same number of rows as columns and row
$i$ and column
$i$ have the same length.  Because row $i$ and column $i$ share one
square, and we only use that square once when we create a new row, each row
we create has odd length.  Thus we get a partition with the same number of
squares, so it is a partition of $k$ and each part is odd.  The parts are
distinct because when we take off the squares of a row and column, we
reduce the number of squares in each row and column that remains.  Given
a partition of $k$ into distinct odd parts, we use the fact that each
row has a unique middle element, and each is shorter than the one
above (by at least two squares) to reverse the process.  Thus we have a
bijection.}

\item Explain the relationship between the number of partitions of $k$
into even parts and the number of partitions of $k$ into parts of even
multiplicity, i.e. parts which are each used an  even number of times as
in (3,3,3,3,2,2,1,1).
\solution{The number of partitions of $k$ into even parts equals the
number of partitions of parts of even multiplicity, because if we take
the Young diagram of a partition of $k$ into even parts and conjugate
it, the resulting diagram has columns of even length.  Thus the
difference in heights of two successive columns is an even number,
but this difference is the multiplicity of one of the parts of the
conjugate. Further the height of the last column of a partition is
the multiplicity of the first part. Since the multiplicity of any part
of a partition is either the difference in height of two successive
columns of the Young diagram or the height of the last column, then
each part of the conjugate has even multiplicity.  This bijection can
be reversed, because if all the differences in height of the columns are
even and the height of the last column is even, then when we
conjugate this partition, the last row will be an even length, and
all differences in length of the rows will be even, so all the parts
of the resulting partition will be even.}
  
\itemi Show that the number of partitions of $k$ into $4$ parts equals
the number of partitions of $3k$ (or $3k+4$ or $3k-4$) into $4$ parts.
\solution{Think about putting the Young diagram of the partition into
the upper left corner of a rectangle that is $k$ units wide and four
units high.  Subdivide the rectangle into $4k$ squares of unit area. 
The Young diagram covers $k$ of these squares.  The uncovered squares
are in rows of length $r_1\le r_2\le r_3\le r_4$.  Thus if we list
these lengths in the opposite order, we have a decreasing list
representation of a partition of $3k$. Even $r_1$ will have to be
positive, because the first part of the original partition will be at 
most $k-3$. To get partitions of $3k+4$, use a rectangle of width
$k+1$, and to get partitions of $3k-4$, use a rectangle of width
$k-1$.  Since the first row of the Young diagram has at most $k-3$
squares, we will still have four nonzero parts in the partition that
results.}

\item The idea of conjugation of a partition could be defined without the
geometric interpretation of a Young diagram, but it would seem far less
natural without the geometric interpretation.  Another idea that seems
much more natural in a geometric context is this.  Suppose we have a
partition of $k$ into $n$ parts with largest part $m$.  Then the Young
diagram of the partition can fit into a rectangle that is $m$ or more
units wide (horizontally) and $n$ or more units deep.  Suppose we place
the Young diagram of our partition in the top left-hand corner of an $m'$
unit wide and
$n'$ unit deep rectangle with $m'\ge m$ and $n' \ge n$, as in Figure
\ref{complementpartition}.
\begin{figure}[htb]\caption{To complement the partition
(5,3,3,2) in a 6 by 5 rectangle: enclose it in
the rectangle, rotate, and cut out the
original Young diagram.}\label{complementpartition}
\begin{center}\psfig{figure=complementpartition.eps%,height=1.0in
} \end{center}
\end{figure}  Why can we interpret the part of the rectangle
not occupied by our Young diagram, rotated in the
plane, as the Young diagram of another partition?  This is called the
{\em complement}\index{complement of a partition} of our partition in the
rectangle.  What integer is being partitioned by the complement?  What
conditions on
$m'$ and $n'$ guarantee that the complement has the same number of parts
as the original one?  What conditions on $m'$ and $n'$ guarantee that the
complement has the same largest part as the original one?  Is it possible
for the complement to have both the same number of parts and the same
largest part as the original one?  If we complement a partition in an
$m'$ by $n'$ box and then complement that partition in an $m'$ by $n'$
box again, do we get the same partition that we started
with?\label{rectanglecomplement}
\solution{If we fill the rectangle with unit squares, those not in the
Young diagram of the original partition $\lambda$ will fall into rows. 
The lengths of the rows are nonnegative, and are nondecreasing as we
move down.  Therefore, after we rotate through 180 degrees, these same
rows will be listed in the opposite order, lined up along the left sides, 
and will have nonincreasing length.  Thus they will be the Young
diagram of a partition.  The integer being partitioned will be
$m'n'-k$.  If
$m'>m$ and $n'=n$. the two partitions will have the same number of
parts, because we will have a nonzero number of empty squares at the end
of each row of the Young diagram of $\lambda$.  If $m'=m$ and $n'-n$
is the multiplicity of the largest part of $\lambda$, they
will have the same number of parts.  Otherwise, their numbers of
parts will differ.  If
$n'>n$ and
$m=m'$, then the two partitions will have the same largest part.  If
$n'=n$ and $m'-m$ is the smallest part of $\lambda$,
then they will have the same largest part.  Otherwise, their largest
parts will differ.  Thus for the two partitions to have the same
number of parts, either $m'=m$ or $n'=n$.  If $m'=m$ and they have the
same largest part, then $n'>n$.  But this is consistent with $n'-n$
being the multiplicity of the largest part of $\lambda$.  Thus they can
have the same number of parts and the same largest part if $m'=m$ and
$n'-n$ is the multiplicity of the largest part of $\lambda$, or
similarly if
$n=n'$ and $m'-m$ is the smallest part of $\lambda$.}


\itemi  Suppose we take a partition of $k$ into $n$ parts with largest
part $m$, complement it in the smallest rectangle it will fit into,
complement the result in the smallest rectangle it will fit into, and
continue the process until we get the partition 1 of one into one part. 
What can you say about the partition with which we started?
\solution{Let us call the process of enclosing $\lambda$ in the smallest
rectangle possible and then forming the complement in that rectangle
{\em encomplementation} (This is short for {\em en}\/closure
and {\em complementation} and is not a standard term---there is no
standard term for this operation.) and call the result of it the {\em
encomplement}\index{encomplement of a partition} of
$\lambda$.
The result of two encomplementations on the Young diagram
of a partition is to remove all rows of maximum length and all columns
of maximum length from the Young diagram.  Thus the description of the
result of an even number $2j$ of encomplementations is straightforward;
we remove all the rows of the $j$ largest distinct lengths and all
columns of the $j$ largest distinct lengths.  So if an even number of
encomplementations brings us to a partition with one block of size
one, we should be able to describe the original partition fairly
easily.  To deal with the result of an odd number of
encomplementations, we ask what happens if we encomplement just once.
If the complement of
$\lambda$ in the smallest rectangle in which if fits has one square, then
$\lambda =\lambda_1^{n_1}\lambda_1-1$.  Thus we are asking for the
partitions which, after an even number of encomplementations, give us
either the partition with one block or a partition of the form
$\lambda_1^{n_1}(\lambda_1-1)$.  First we  ask what
kind of partition results in the second one after two
encomplementations.  If we get
$\lambda_1^{n_1}(\lambda_1-1)$ from two
encomplementations, the partition we started with had the form
$$\lambda_0^{n_0}(\lambda_1+\lambda_{2})^{n_1}(\lambda_1+
\lambda_2-1)\lambda_2^{n_2}.$$  If
we get $\lambda_1^{n_1}(\lambda_1-1)$ from four
encomplementations, then we started with a partition of the form
$$\lambda_{-1}^{n_{-1}}(\lambda_0+\lambda_{3})^{n_0}(\lambda_1+
\lambda_2 +
\lambda_{3})^{n_1}(\lambda_1+\lambda_2 +\lambda_3-1)(\lambda_2+
\lambda_{3})^{n_3}
\lambda_3^{n_3}.$$
From this pattern we see that a partition that results in
$\lambda_1^{n_1}(\lambda_1-1)$ after $2j$ encomplementations has the
form
  \begin{equation}
\lambda_{1-j}^{n_{1-j}}\lambda_{2-j}^{n_{2-j}}\cdots
\lambda_0^{n_0}
{\lambda'_1}^{n_1}
(\lambda'_1-1)\lambda_2^{n_2}\cdots
\lambda_{j+1}^{n_{j+1}},\label{form1}
  \end{equation}
where $\lambda_i>\lambda_{i+1}$ and $\lambda_0>\lambda'_1>\lambda_2+1$.

On the other hand, a partition $\lambda$ that results in $1$ after
two encomplementations has the form
$\lambda_0^{n_0}(\lambda_1+1)\lambda_1^{n_1}$, and so a partition that
results in 1 after $j$ encomplementations is of the form
  \begin{equation}
\lambda_{1-j}^{n_{1-j}}\lambda_{2-j}^{n_{2-j}}\cdots
\lambda_0^{n_0}(\lambda_1+1)\lambda_1^{n_1}\lambda_2^{n_2}\cdots
\lambda_j^{n_j},\label{form2}
  \end{equation} 
where $\lambda_i>\lambda_{i+1}$ and $\lambda_0>\lambda_1+1$.  Thus a
partition results in a single part of size 1 after some number of
encomplementations if and only if it has the form of Equation
\ref{form1} or Equation \ref{form2}.}



\item Show that $P(k,n)$ is at least ${1\over n!}{k-1\choose n-1}$.
\solution{The number of compositions of $k$ into $n$ parts is
$k-1\choose n-1$.  We can divide the compositions into blocks,
where two compositions are in the same block if and only if one is a
rearrangement of the other.  Then the blocks correspond bijectively to
partitions of $k$ into $n$ parts.  However we cannot compute the
number of blocks by dividing by the number of compositions per block
since the number of compositions per block ranges from $1$ to $n!$. 
But then if we divide the number of compositions by $n!$ we will get a
number less than the number of blocks because $n!$ times the number of
blocks would be, by the sum principle, greater than the number of
partitions.}
\ep

 With the binomial coefficients, with Stirling numbers of the
 second kind, and with the Lah numbers, we were able to find a
recurrence by asking what happens to our subset, partition, or broken
permutation of a set $S$ of numbers if we remove the largest element of
$S$.  Thus it is natural to look for a recurrence to count the number of
partitions of $k$ into $n$ parts by doing something similar. 
Unfortunately, since we are counting distributions in which all the
objects are identical, there is no way for us to identify a largest
element. However if we think geometrically, we can ask what we could
remove from a Young diagram to get a Young diagram. Two natural ways to
get a partition of a smaller integer from a partition of
$n$ would be to remove the top row of the Young diagram of the partition
and to remove the left column of the Young diagram of the partition. 
These two operations correspond to removing the largest part from the
partition and to subtracting 1 from each part of the partition
respectively.  Even though they are symmetric with respect to
conjugation, they aren't symmetric with respect to the number of parts. 
Thus one might be much more useful than the other for finding a recurrence
for the number of partitions of
$k$ into $n$ parts.

\bp \itemesi In this problem we will study the two operations and see
which one seems more useful for getting a recurrence for
$P(k,n)$.\label{numberpartitionrecurrence}
\begin{enumerate}

\item How many parts does the remaining partition
have when we remove the largest part (more precisely, we reduce its
multiplicity by one) from a partition of
$k$ into
$n$ parts?  What can you say about the number of parts of the remaining
partition if we remove one from each part?
\solution{Reducing the multiplicity of the largest part by one reduces
the number of parts by one.  Removing 1 from each part reduces the
number of parts by the multiplicity of the smallest part, so it
strictly reduces the number of parts, perhaps even to one.}

\item If the largest part of a partition is $j$ and we remove it,
what integer is being partitioned by the remaining parts of the
partition? If we remove one from each part of a partition of $k$ into
$n$ parts, what integer is being partitioned by the remaining parts? 
\solution{If we remove the largest part, the integer being partitioned
is $k$ minus the largest part.  Thus it is a number less than $k$ and
at least $n-1$. If we remove one from each part of the partition, the
integer being partitioned is $k-n$.}

\item The last two questions are designed to get you thinking about how
we can get a bijection between the set of  partitions of $k$ into
$n$ parts and some other set of partitions that are partitions of a
smaller number.  These questions describe two different strategies for
getting that set of partitions of a smaller number or of smaller numbers. 
Each strategy leads to a bijection between partitions of $k$ into $n$
parts and a set of partitions of a smaller number or numbers.  For each
strategy, use the answers to the last two questions to find and describe 
this set of partitions into a smaller number and a bijection between
partitions of
$k$ into
$n$ parts and partitions of the smaller integer or integers into
appropriate numbers of parts.
\solution{Removing the largest part of a partition of $k$ into $n$
parts gives us a bijection between partitions of $k$ into $n$ parts and
and partitions of numbers $k'$ between
$n-1$ and $k-1$ into $n-1$ parts of size at most $k-k'$.  (Removing
the largest part gives us such a partition, and adjoining a part of
size $k-k'$ to such a partition gives us a partition of $k$ with $n$
parts.)

Removing one from each part of a partition of $k$ into $n$
parts gives us a bijection between partitions of $k$ into $n$ parts
and and partitions $k-n$ into $n$ or fewer parts.  (Removing one from
each part of a partition of $k$ into $n$ parts  gives us such a
partition, and, given such a partition, we get a partition of $k$ into
$n$ parts by adding one to each part and then creating enough parts of
size 1 to have $n$ parts.)}
\item Find a recurrence (which need not have just two terms on the
right hand side) that describes how to compute $P(k,n)$ in terms of the
number of partitions of smaller integers into a smaller number of
parts. (Hint: One of the two sets of partitions of smaller numbers
from the previous part is more amenable to finding a recurrence than the
other.)
\solution{The second bijection is to the set of partitions of $k-1$
into $n$ or fewer parts, and this makes the second bijection sound
easier to work with.  We get $P(k,n)=\sum_{i=1}^n P(k-n,i)$.  The
proof is the bijection we already described; in particular a partition
of $k-n$ into $i$ parts corresponds to the partition of $k$ we get by
adding one to each of the $i$ parts and then creating $n-i$ parts of
size one.}

\item What is $P(k,1)$ for a positive integer $k$?
\solution{$P(k,1)=1$.}

\item What is $P(k,k)$ for a positive integer $k$?
\solution{$P(k,k)=1$.}

\item Use your recurrence to compute a table with the values of
$P(k,n)$ for values of $k$ between 1 and 7.
\solution{
\begin{tabular}{|c|c|c|c|c|c|c|c|}
$k\backslash n$&1&2&3&4&5&6&7\\
\hline
1&1&0&0&0&0&0&0\\
2&1&1&0&0&0&0&0\\
3&1&1&1&0&0&0&0\\
4&1&2&1&1&0&0&0\\
5&1&2&2&1&1&0&0\\
6&1&3&3&2&1&1&0\\
7&1&3&4&3&2&1&1\\
\end{tabular}}

\item What would you want to fill
into row 0 and column 0 of your table in order to make it consistent with
your recurrence.  What does this say $P(0,0)$ should be?  We usually
define a sum with no terms in it to be zero. Is that consistent with the
way the recurrence says we should define $P(0,0)$?
\solution{We would want to have $P(0,0)=1$ and $P(k,0)=P(0,n)=0$ for
positive integer $k$ or $n$.  Since the sum of the empty multiset of
positive integers is zero, this gives us one partition of the number
zero, namely the empty multiset of positive integers.}
\end{enumerate}
\ep 
It is remarkable that
there is no known formula for $P(k,n)$, nor is there one for $P(k)$.  This
section was are devoted to developing
methods for computing values of $P(n,k)$ and finding properties of
$P(n,k)$ that we can prove even without knowing a formula.  Some future
sections will attempt to develop other methods.

We have seen that the number of partitions of $k$ into $n$ parts is equal
to the number of ways to distribute $k$ identical objects to $n$
recipients so that each receives at least one.  If we relax the condition
that each recipient receives at least one, then we see that the number of
distributions of $k$ identical objects to $n$ recipients is $\sum_{i=1}^n
P(k,i)$ because if some recipients receive nothing, it does not matter
which recipients these are.  This completes rows 7 and 8 of our table of
distribution problems.  The completed table is shown in Figure
\ref{lastdistributiontable}.  There are quite a few theorems that you
have proved which are summarized by Table \ref{lastdistributiontable}. 
It would be worthwhile to try to write them all down!


\begin{table}[htb]\caption{The number of ways to
distribute $k$ objects to $n$ recipients, with
restrictions on how the objects are
received}\label{lastdistributiontable}
\begin{center}
\fbox{\hglue-3.5pt
{\footnotesize\begin{tabular}{|l||c |c|}\hline
\multicolumn{3}{|c|}{The Twentyfold Way: A Table of Distribution
Problems}
\\ %\multicolumn{3}{|c|}{\hrulefil}
\hline\hline
$k$ objects and conditions&\multicolumn{2}{|c|}{ $n$ recipients and
mathematical model for distribution}\\
\cline{2-3}
on how they are received&Distinct&Identical\\
\hline\hline 1.  Distinct &$n^k$& 
$\sum_{i=1}^kS(n,i)$\tallstrut{12}\\
no conditions&functions&set partitions ($\le n$ parts)\\ \hline
2.  Distinct&$n^{\underline{k}}\tallstrut{12}$ &1 if
$k\le n$; 0 otherwise\\ Each gets at most one&\kern -2pt $k$-element
permutations\kern -2 pt&\\
\hline
3.  Distinct&$S(k,n)n!$& $S(k,n)$
\\ Each gets at least one&onto functions& set partitions
($n$ parts)\\\hline 
4. Distinct&$k!=n!$&1 if $k=n$; 0 otherwise\\ Each
gets exactly one&permutations&\\
\hline
5.  Distinct, order matters&$(k+n-1)^{\underline{k}}\tallstrut{12}$&
$\sum_{i=1}^n L(k,i)$\tallstrut{11}\\
&ordered functions&\hglue -3 pt broken permutations ($\le n$
parts)\kern -3 pt\\
\hline
6.  Distinct, order
matters&$(k)^{\underline{n}}(k-1)^{\underline{k-n}}$&\tallstrut{12}$L(k,n)=
{ k\choose n}(k-1)^{\underline{k-n}}$\\ Each
gets at least one&ordered onto functions&broken permutations ($n$ parts)\\
\hline
7.  Identical  & $n+k-1\choose k$\tallstrut{12}&$\sum_{i=1}^nP(k,i)$\\
no conditions&multisets& number partitions ($\le n$ parts)\\ \hline
8.  Identical&$n\choose k$\tallstrut{12}&1 if $k\le n$; 0 otherwise\\
Each gets at most one&	subsets&\\ \hline
9.  Identical&$k-1\choose n-1$\tallstrut{12}& $P(k,n) $\\
Each gets at least one&compositions ($n$ parts) &number partitions ($n$
parts)
\\
\hline 10.  Identical&1 if $k=n$; 0 otherwise&1 if $k=n$; 0 otherwise\\
Each gets exactly one&&\\
\hline
\end{tabular}\hglue-3pt}}\end{center}
\end{table}

\subsection{Partitions into distinct parts}
 Often $Q(k,n)$ is used to denote the number of partitions of $k$ into
distinct parts, that is, parts that are different from each other.  
\bp 
\item Show that $$Q(k,n) \le {1\over n!}{k-1\choose n-1}.$$
\solution{The number of compositions of $k$ into $n$ parts is
$k-1\choose n-1$.  Thus the number of compositions of $k$ into $n$
distinct parts is less than $k-1\choose n-1$.  Divide the compositions
of $k$ into $n$ distinct parts into blocks with two compositions in
the same block if one is a rearrangement of the other.  Because the
parts are distinct, each block has $n!$ members.  Further, there is a
bijection between the blocks of this partition and the partitions of
$k$ into $n$ distinct parts.  Since the number of compositions of $k$
into $n$ distinct parts is less than $k-1 \choose n-1$, the number of
partitions of $k$ into $n$ distinct parts is less than ${1\over n!}
{k-1\choose n-1}$.}

\itemi Show that the number of partitions of 7 into 3 parts equals the
number of partitions of 10 into three distinct parts.  
\solution{Given a
partition $\lambda$ of 7 in decreasing list form
$\lambda_1,\lambda_2,\lambda_3$, if we add 0 to $\lambda_3$, $1$ to
$\lambda_2$ and 2 to $\lambda_1$ the resulting partition of 10 has
distinct parts.  If we take a partition $\lambda'$ of 10 with distinct
parts, then $\lambda'_1\ge\lambda'_2+1$,  $\lambda'_1\ge\lambda'_2+2$,
and
$\lambda'_2\ge \lambda'_3+1$. Therefore if we subtract 2 from 
$\lambda'_1$ to get $\lambda_1$, subtract 1 from $\lambda'_2$ to get
$\lambda_2$ and let $\lambda_3= \lambda'_3$, then
$\lambda_1,\lambda_2,\lambda_3$ is the decreasing list representation
of a partition of $10-3=7$.  Thus there is a bijection between
partitions of $7$ into three parts and partitions of $10$ into three
distinct parts.}

\itemesi There is a relationship between $P(k,n)$ and $Q(m,n)$ for some
other number $m$.  Find the number $m$ that gives you the nicest possible
relationship.
\solution{The number of partitions of $k$ into $n$ parts is equal to
the number of partitions of $k+{n\choose2}$ into n distinct parts. 
The bijection from partitions of $k$ with $n$ parts to partitions of
$k+{n\choose2}$ with $n$ distinct parts that proves this is the one
that takes a partition $\lambda_n\lambda_{n-1}\cdots\lambda_1$ of $k$ with
$\lambda_i>\lambda_{i+1}$ and adds $i-1$ to $\lambda_i$ to get
$\lambda'_i$.  Then
$\lambda'$ is a partition into distinct parts, and the number it
partitions is
$k+1+2+\cdots+n-1=k+{n\choose 2}$.  The proof that it is a bijection is
the fact that subtracting $n-i$ from the $i$\/th part of a partition of
$k$ into distinct parts yields a partition of $k$, because part $i+j$
is at least $j$ smaller than part $i$.}

\itemes Find a recurrence that expresses $Q(k,n)$ as a sum  of $Q(k-n,m)$
for appropriate values of $m$.
\solution{Suppose $\lambda$ is a partition of $k$ into $n$ distinct
parts.  Either 1 is one of those parts or not.  Thus if we subtract
1 from each part, we either get a partition of $k-n$ into $n-1$ parts
or a partition of $k-n$ into $n$ parts. If $\lambda$ and $\lambda'$
are different partitions of $k$ into $n$ distinct parts, they go to
different partitions.  Each partition of
$k-n$ into
$n-1$ parts or $n$ parts can be gotten in this way from a
corresponding partition of $k$ into $n$ parts. Thus we have a
bijective correspondence and $Q(k,n)=Q(k-n,n-1) +Q(k-n,n)$.}

\itemih Show that the number of partitions of $k$ into distinct
parts equals the number of partitions of $k$ into odd parts.
\solution{We start by giving a function from the set of partitions of
$k$ to the set of partitions of $k$ with (only) odd parts.  Clearly
such a function cannot be one to one.  Then we show that when
restricted to the partitions with distinct parts it is one-to-one and
onto by constructing an inverse.  Given a partition
$\lambda_1^{i_1}\lambda_2^{i_2}\cdots\lambda_n^{i_n}$, write
$\lambda_i=\gamma_i2^{k_i}$, where $\gamma_i$ is odd.  (Thus $2^{k_i}$
is the highest power of 2 that is a factor of $\lambda_i$, so it is 1 if
$\lambda_i$ is odd.).  It is possible that $\gamma_i=\gamma_j$, for
example if $\lambda_i=36$ and $\lambda_j=18$, then
$\gamma_i=\gamma_j=9$.  We construct a new partition $\pi$ whose parts
are the numbers $\gamma_j$ as follows:  Given an odd number $p$, let
the multiplicity $m(p)$ of $p$ in $\pi$ be $\sum_{j: \gamma_j=m}
2^{k_j}$. Thus $\sum_{p: m(p)\not=0}m(p)p = k$.  Therefore, $\pi$ is a
partition of $k$ whose parts are all odd.

Now consider a partition $\pi$ of $k$ whose parts are all odd.  Let
$\pi=\pi_1^{r_1}\pi_2^{r_2}\cdots \pi_t^{r_t}$, with
$\pi_i>\pi_{i+1}$.  (In terms of the multiplicity function $m$,
$m(\pi_i) =r_i$, and $\sum_{i=1}^t r_i\pi_i = k$.)  We are going to
write the binary expansion of each $r_i$  as $r_i = \sum_{j=
0}^{\lfloor
\log_2 r_i\rfloor} 2^{ja_{ij}}$, where $a_{ij}$ is 1 if $2^j$ appears in
the binary expansion of $r_i$, and 0 otherwise.  All of the numbers
$\pi_i2^{ja_{ij}}$ are distinct, because a power of two times one odd
number cannot equal a  power of two times another odd
number.  The numbers $\pi_i2^{ja_{ij}}$ add to $k$, so they are the
parts of a partition $\pi'$ of $k$ into distinct parts.  When we apply
the function constructed in the first part of the solution to $\pi'$,
we get $\pi$, so the correspondence between $\pi$ and $\pi'$ is a
bijection.}

\itemih  Euler showed that if $k\not= {3j^2+j\over 2}$, then the
number of partitions of $k$ into an even number of distinct parts is the
same as the number of partitions of $k$ into an odd number of distinct
parts. Prove this, and in the exceptional case find out how the two
numbers relate to each other.
\solution{This solution is taken largely from the book Introduction to
Combinatorics by Ioan Tomescu (published in London by Collet's in
1975).  Tomescu calls a collection of rows in a Young diagram a
``trapezoid" if each row contains one less cell than the row above and
the number of cells in the rows above and below the trapezoid differ by
two or more from the number of cells in rows of the trapezoid.  Thus in
(8,6,5,4,2,1) we have 3 trapezoids, the first row, the next three rows,
and the last two.  Since we are dealing with partitions with distinct
parts, we don't have to worry about how two equal rows  affect the
definition of a trapezoid.   We will describe a way to transform a
partition with an even number of distinct parts into a partition with
an odd number of distinct parts and vice versa.

First we describe a transformation on Young
diagrams.  Here is the first part of the description. Suppose the smallest
 part
$m$ of $\lambda$ is less than or equal to the number $j$ of rows in the
top trapezoid.  Suppose further that if we have only one trapezoid,
then $j>m$.  Then we construct a partition with one less part by
adding 1 to each of the
$m$ largest parts and discarding the part
$m$.  We still have a diagram for a partition of the same integer,
but now the parity of the number of parts has changed, and we {\it may}
have increased the number of trapezoids by 1.  The smallest part
will now be larger than the number (now $m$) of rows in the top
trapezoid.  (Notice that the construction would not work if we had
only one trapezoid and $j=m$ because we would first remove one
row of the trapezoid and thus have no row to which to attach one of
our squares.)

Here is the second part of the description of the
transformation.  Suppose now that
$m$ is larger than the number $j$ of rows of the top trapezoid in
the Young diagram.   Suppose also that
the Young diagram has at least two trapezoids or  it
has one trapezoid and $j\ge m-2$. Take one square from each of the
$j$ rows  of the top trapezoid (which is the whole diagram if there
is only one trapezoid) and also add a row of
$j$ squares at the bottom of the diagram.  (Since $m>j$, this gives us
a Young diagram of a partition of the same integer into distinct
parts.)  The parity of the number of rows has changed, and now the
number of rows of the top trapezoid is at least as large as the
smallest part of the partition.  (Note, two previously distinct
trapezoids may have joined together to form one on top.) 
(Notice that if we have one trapezoid and $j= m+1$, then the
construction yields a partition with two equal parts, which is why
we made the special assumption above.)  Now let $T$ be the
transformation  described by the two constructions above.  Its
domain is all Young diagrams except those with one trapezoid and
$m\le j\le m+1$.  $T^2$ is the identity, and so $T$ is a bijection. 
When restricted to partitions with an odd number of parts, $T$
gives partitions with an even number of parts, so on its domain it
gives a bijection between partitions with an even number of parts
and partitions with an odd number of parts.

If $m=j$ and the diagram has just one trapezoid, then the diagram has
$3j^2-j\over 2$ squares, and if
$m=j+1$ and the diagram has just one trapezoid, then the diagram has
$3j^2+j\over 2$ squares.   Thus if
$k\ne {3j^2\pm j\over 2}$, the number of partitions of $k$ into distinct
even parts equals the number of partitions of $k$ into distinct odd
parts.  

If
$k= {3j^2\pm j\over 2}$ and $j$ is even, then there is one diagram of
a partition of $k$ that is not in the domain of the bijection and has
an even number of rows, so in this case there will be one more
partition with an even number of parts than with an odd number.  If
 $k=
{3j^2\pm j\over 2}$ and $j$ is odd, there is one diagram with an
odd number of rows not in the domain and so in this case there is
one more partition with an odd number of parts than with an even
number.  This completes the exceptional cases of the problem.}
\ep



\subsection{Supplementary Problems}  
\begin{enumerate}
\item Answer each of the following questions with $n^k$,
$k^n$, $n!$,
$k!$,
$n \choose k$, $k \choose n$, $n^{\underline{k}}$, $k^{\underline{n}}$,
$n^{\overline{k}}$, $k^{\overline{n}}$, $n+k-1\choose k$, 
$n+k-1\choose n$, $n-1\choose k-1$, $k-1\choose n-1$, or ``none of the
above".
\begin{enumerate} 
\item In how many ways may we pass out $k$ identical pieces of candy to
$n$ children?
  \solution{$n+k-1\choose k$
  }

\item  In how many ways may we pass out $k$ distinct pieces of candy to
$n$ children?


    \solution{$n^k$}



\item  In how many ways may we pass out $k$ identical pieces of candy to
$n$ children so that each gets at most one?  (Assume $k\le n$.) 
\solution{$n\choose k$.}


\item  In how many ways may we pass out $k$ distinct pieces of candy to
$n$ children so that each gets at most one?  (Assume $k\le n$.)
\solution{$n^{\underline{k}}$}

\item In how many ways may we pass out $k$ distinct pieces of
candy to $n$ children so that each gets at least one?  (Assume $k\ge n$.)
\solution{ None of the above.
 }

\item  In how many ways may we pass out $k$ identical pieces of candy to
$n$ children so that each gets at least one?  (Assume $k\ge n$.)
\solution{
$k-1\choose n-1$}
\end{enumerate}

\item The neighborhood betterment committee has been given $r$ trees to
distribute to $s$ families living along one side of a street.
\begin{enumerate}
\item In how many ways can they distribute all of
them if the trees are distinct, there are more families than trees, and
each family can get at most one? 
\solution{$s^{\underline{r}}$}

\item In how many ways can they
distribute all of them if the trees are distinct, any family can get any
number, and a family may plant its trees where it chooses?
\solution{$s^r$}

\item In how many ways can they distribute all the trees if the trees are
identical, there are no more trees than families,   and any family
receives at most one?   
\solution{$s\choose r$}

\item In how many ways can they distribute them if the trees are
distinct, there are more trees than families, and each family
receives at most one (so there could be some leftover trees)?
\solution{ $\sum_{k=0}^s {s\choose k}r^{\underline{k}}$ \quad or\quad
$\sum_{k=0}^s s^{\underline{k}}{r\choose k}$}

\item In how many ways can they distribute all the trees if they are
identical and anyone may receive any number of trees?
\label{multisetproblem}
\solution{$r+s-1\choose r$}

\item  In how many ways can all the trees be distributed and planted if
the trees are distinct, any family can get any number, and a family must
plant its trees in an evenly spaced row along the road? 
\label{orderedfunctionproblem}
\solution{$s^{\overline{r}}=(r+s-1)^{\underline{r}}$}

\item  Answer the question in Part \ref{orderedfunctionproblem}
assuming that every family must get a tree.
\solution{$r!{r-1\choose s-1}$}

\item  Answer the question in Part \ref{multisetproblem} assuming that
each family must get at least one tree.
\solution{$r-1\choose s-1$}
\end{enumerate}

\item  In how many ways can $n$ identical chemistry books, $r$ identical
mathematics books, $s$ identical physics books, and $t$ identical astronomy books
be arranged on three bookshelves?
(Assume there is no limit on the number of books per shelf.)
\solution{${(n+r+s+t+2)! \over n!r!s!t!2!}$}


\itemi One formula for the Lah numbers is
$$L(k,n) = {k\choose n}(k-1)^{\underline{k-n}}$$ 
Find a proof that explains this product.
\solution{
 First choose the $n$ elements which will be the first
member of the part they lie in.  (This, in effect, labels the $n$
parts.)  Then assign the remaining $k-n$ elements to their parts
by making an ordered function of $n-k$ objects to $n$
recipients in $(n + (k-n) - 1)^{{k-n}} = (k-1)^{{k-n}}$ ways.}

\item What is the number of partitions of $n$ into two parts?
\solution{$n/2$ if $n$ is even and $(n-1)/2$ if $n$ is odd, equivalently,
$\lfloor n/2\rfloor$}

\item Show that the number of partitions of $k$ into $n$ parts of size at
most $m$ equals the number of partitions of $mn-k$ into no more than $n$
parts of size at most $m-1$.
\solution{If we take the complement of the Young diagram of a partition
of $k$ into $n$ parts of size at most $m$ in an rectangle with $n$ rows
and $m$ columns, the number we partition will be $mn-k$, and we will have
no more than $n$ parts, each of size at most $m-1$.  And if we take the
complement of a partition of this second kind in the same rectangle, we
will get a partition of the first kind.}

\item Show that the number of partitions of $k$ into parts of size at
most $m$ is equal to the number of partitions of of $k+m$ into $m$ parts.
\solution{Given the first kind of partition, take the conjugate (giving a
partition of $k$ into at most $m$ parts), add one to each part, and then
add enough parts of size 1 to get a total of $m$ parts.  It is
straightforward that this process can be reversed.}

\item  You can say something pretty specific about 
self-conjugate partitions of $k$ into distinct parts.  Figure out what
it is and prove it.  With that, you should be able to find a relationship
between these partitions and partitions whose parts are consecutive
integers, starting with 1.  What is that relationship?
\solution{In a self-conjugate partition, the number of parts is the size
of the largest part.  If these parts are distinct, this means that each
number between 1 and the largest part appears once as a part.  That is,
the parts are a list of consecutive integers, starting with 1.}

\item What is $s(k,1)$?
\solution{Since s$(k,1)$ is the coefficient of $x^1$ in
$$x^{\underline{k}} = x(x-1)
(x-2)\cdot (x-(k-1)),$$ it is $(-1)^{k-1}(k-1)!$.}

\item Show that the Stirling numbers of the second kind satisfy the
recurrence 
$$S(k,n) = \sum_{i=1}^kS(k-i,n-1){n-1\choose i-1}.$$
\solution{A partition of $[k]$ into $n$ blocks has a block containing
$k$.  If this block has size $i$, when you remove it, you get a partition
of a set of size $k-i$ into $n-1$ blocks.  The number of possible sets of
size $i$ containing $k$ is $k-1\choose i-1$, and $i$ can be any number
between 1 and $k$.  Each partition of $k$ into $n$ blocks may be
constructed exactly once by first choosing the block containing $k$ and
then partitioning the remaining elements into $n-1$ blocks.  This proves
the formula.}

\itemi Let $c(k,n)$ be the number of ways for $k$ children to hold hands
to form $n$ circles, where one child clasping his or her hands together
and holding them out to form a circle is considered a circle.  Find a
recurrence for $c(k,n)$.  Is the family of numbers $c(k,n)$ related to any
of the other families of numbers we have studied? If so, how?
\solution{The $k$th child is either in a circle by him/her self, and
there are $c(k-1,n-1)$ ways for this to happen, or is in a circle with
some other children.  In the second case child $i$ can be to the
immediate right of any of the other $k-1$ children, so there are
$(k-1)c(k-1,n)$ ways for this to happen.  Thus $c(k,n)=c(k-1,n-1)
+(k-1)c(k-1,n)$.  This recurrence is almost the same as the recurrence for
$s(k,n)$, except it has a plus sign where the recurrence for the Stirling
numbers of the first kind has a minus sign.  Further $c(k,1)=(k-1)!$ and
$c(k,k)=1$, which agrees, except for sign, with the Stirling numbers of
the first kind.  If we experiment with applying the recurrence, we see
that whenever we use it to compute $c(k,n)$, we get that
$c(k,n)=|s(k,n)|$.  It is now straightforward to prove by induction that
$c(k,n)=|s(k,n)|$.}



\end{enumerate}